\chapter{Recursos}\label{cap:recursos}

\section{Recursos humanos}

Para el desarrollo del proyecto se ha contado con un equipo formado por cuatro integrantes, todos ayudando a los demas en sus apartados pero cada uno con un rol específico para asegurar una adecuada distribución del trabajo y una correcta ejecución de todas las fases del proyecto.

\begin{itemize}
    \item \textbf{Raúl} 
    \item[] Responsable principal del backend y la lógica del servidor. Su labor incluye la creación de las API, la comunicación con la base de datos y la implementación de la seguridad del sistema.

    \item \textbf{Javier}
    \item[] Responsable del análisis y la documentación técnica. Se encarga de la especificación de requisitos, la estructura de la documentación y la supervisión del seguimiento metodológico del proyecto.

    \item \textbf{Jose}
    \item[] Responsable de frontend y diseño de la interfaz. Se encarga de la maquetación, la usabilidad, el diseño responsive y la experiencia de usuario.

    \item \textbf{Antonio}
    \item[] Labores híbridas entre front y back apoyando a Raul y a Jose en gran medida con sus labores. Se encarga de unir backend y frontend, realizar pruebas, gestionar el despliegue en el servidor y solucionar incidencias del entorno.
\end{itemize}

\section{Recursos de hardware}

Para el desarrollo de la aplicación en el entorno local...:

\begin{itemize}
    \item Ordenador: PC de sobremesa / portátil de uso personal.
    \item Sistema operativo: Windows 10 / Windows 11 / MacOs / Linux Ubuntu 22.04.
    \item RAM instalada: 16 GB.
    \item Procesador: Intel Core i5 / i7 de 10ª generación o superior, o equivalente AMD Ryzen.
\end{itemize}

Para el despliegue de la aplicación web, se hará uso de un servidor VPS con las características siguientes:
\begin{itemize}
    \item Platform: Servidor VPS en la nube.
    \item OS Package: Linux Ubuntu Server 22.04 LTS.
    \item Memory: 2 GB – 4 GB de RAM.
    \item SSD: 40 GB – 80 GB de almacenamiento SSD.
\end{itemize}


\section{Recursos de software}

Se van a utilizar los siguientes recursos de software para el desarrollo de la aplicación web: 

\begin{itemize}
    \item Editor de código fuente: Visual Studio Code \cite{visualStudioCode}.
    \item Lenguaje backend: PHP 8.4 con framework Laravel.
    \item Lenguaje frontend: HTML5, CSS, JavaScript y React.
    \item Sistema gestor de bases de datos: MySQL.
    \item Control de versiones: Git y plataforma GitHub.
    \item Servidor local de desarrollo: XAMPP.
    \item Navegadores para pruebas: Google Chrome y Mozilla Firefox.
    \item Herramientas adicionales: Postman para pruebas de API, Figma para prototipado de interfaz, Composer para gestión de dependencias, plesk para el servidor.
\end{itemize}


