\chapter{Diseño de datos} \label{cap:diseño_datos}

\section{Introducción}
En este capítulo se definirán las estructuras de datos que conforman el sistema a partir de los elementos identificados durante el análisis de datos del Capítulo 7 (tipos de entidad y tipos de interrelación). Para ello, se llevarán a cabo los siguientes pasos:

\begin{itemize}
    \item Obtención del modelo relacional. Definición de la estructuras (tablas) del modelo de datos.
    \item Normalización del modelo. Refinamiento del modelo, para la eliminación de errores de integridad.
    \item Obtención del esquema relacional.
    \item Diagrama relacional.
\end{itemize}

 \section{Modelo Relacional}\label{sec:modelo-relacional}
A partir del Modelo Entidad - Interrelación descrito en el capítulo 7, se pueden obtener las tablas o relaciones del Modelo Relacional utilizando las reglas de transformación (RTECAR - véase el capítulo 5 de Bases de Datos. Desde Chen hasta Codd con ORACLE.\cite{reglasBD}). En concreto, se han aplicado las siguientes reglas de transformación:

\begin{itemize}
    \item RTECAR-1: Todos los tipos de entidad presentes en el esquema conceptual se transformarán en tablas o relaciones en el esquema relacional manteniendo el número y tipo de atributos, así como la característica de identificador de esos atributos.
    \item RTECAR-3.1: En las relaciones 1:N, la clave primaria de la entidad del lado 1 se convierte en una clave foránea en la tabla de la entidad del lado N.
\end{itemize}

Para cada tabla se mostrará la siguiente información:

\begin{itemize}
    \item \textbf{Descripción}. Se describirá el origen de la tabla, indicando los elementos del modelo Entidad-Interrelación desde los que se ha obtenido.
    \item \textbf{Nombre de la tabla}
    \item \textbf{Atributos}. Se describirán los atributos que componen la tabla, distinguiendo su rol en cada caso con la siguiente notación:
        \begin{itemize}
            \item Clave \underline{primaria}
            \item Clave ALTERNA, si existe
            \item Claves \textbf{foráneas}, si existen
            \item Resto de atributos
        \end{itemize}
    \item \textbf{Esquema relacional}. Definición formal de la tabla, de acuerdo con el Modelo Relacional.
\end{itemize}

\subsection{Tabla Usuario}
\begin{itemize}
    \item \textbf{Descripción}: la tabla \textit{Usuario} se obtiene a partir de la entidad \textit{Usuario}
    (modelo \texttt{User}). Almacena las credenciales y datos básicos necesarios para la autenticación y
    la identificación del usuario dentro del sistema. Además, se relaciona con la entidad \textit{Profesional/Trabajador}
    mediante el atributo \texttt{worker\_id}.

    \item \textbf{Nombre de la tabla}: Usuario (\texttt{users})

    \item \textbf{Atributos}:
    \begin{itemize}
        \item Clave primaria: \underline{id}
        \item Claves alternas: email (único)
        \item Claves foráneas: \textbf{worker\_id} $\rightarrow$ Trabajador(id) \ (si existe la tabla de trabajadores/profesionales)
        \item Resto de atributos: name, password, avatarUrl, email\_verified\_at, remember\_token, created\_at, updated\_at
    \end{itemize}

    \item \textbf{Esquema relacional}:
    Usuario(\textbf{\underline{id}}, name, \textit{email}, password, avatarUrl, \textbf{worker\_id}, email\_verified\_at, remember\_token, created\_at, updated\_at)
\end{itemize}


 \section{Normalización del modelo}\label{sec:normalizacion}
La normalización del modelo descrito en la sección anterior pretende detectar y corregir redundancias e inconsistencias en la información representada, para lo cual se aplicarán las medidas correctoras que garanticen que las tablas obtenidas cumplen las siguientes formas normalizadas \cite{reglasBD}.

\begin{itemize}
    \item La tabla Usuario cumple la FN1 al almacenar valores atómicos en todos sus atributos.
    \item Cumple FN2 y FN3 al depender todos los atributos no clave de forma completa y no transitiva de la clave primaria.
    \item Además, se encuentra en FNBC, ya que los determinantes funcionales relevantes son claves candidatas (id y email).
\end{itemize}

Todas las tablas definidas se encuentran en la Primera Forma Normal, puesto que en ninguna de ellas existen atributos múltiples.

\subsection{Tabla Usuario}
    \begin{itemize}
        \item \textbf{Claves candidatas}: id
        \item \textbf{Normalización}: La tabla Usuario presenta una dependencia funcional formada por la clave primaria y el resto de atributos. La tabla se encuentra en FNBC: el único determinante funcional es la clave primaria, por lo que la dependencia funcional con el resto de atributos es completa.
    \end{itemize}

\begin{figure}[H]
\centering
\includegraphics[scale=0.75]{img/diagramas/Datos/User.png}
\caption{Tabla Usuario en FNBC}\label{fig:Tabla Usuario en FNBC}   
\end{figure}


\subsection{Tabla Trabajador}
\begin{itemize}
    \item \textbf{Descripción}: la tabla \textit{Trabajador} se obtiene a partir de la entidad \textit{Trabajador}
    (modelo \texttt{Worker} de Laravel). Almacena la información laboral necesaria para la planificación de guardias,
    incluyendo rango/categoría, fechas de alta y baja, y la especialidad a la que pertenece.

    \item \textbf{Nombre de la tabla}: Trabajador (\texttt{worker})

    \item \textbf{Atributos}:
    \begin{itemize}
        \item Clave primaria: \underline{id}
        \item Claves alternas: ---
        \item Claves foráneas: \textbf{id\_speciality} $\rightarrow$ Especialidad(id)
        \item Resto de atributos: name, rank, registration\_date, discharge\_date
    \end{itemize}

    \item \textbf{Esquema relacional}:
    Trabajador(\textbf{\underline{id}}, name, rank, registration\_date, discharge\_date, \textbf{id\_speciality})
\end{itemize}

\subsection{Tabla Trabajador}
    \begin{itemize}
        \item \textbf{Claves candidatas}: id
        \item \textbf{Normalización}: La tabla Trabajador presenta una dependencia funcional formada por la clave primaria y el resto de atributos. La tabla se encuentra en FNBC: el único determinante funcional es la clave primaria, por lo que la dependencia funcional con el resto de atributos es completa.
    \end{itemize}

\begin{figure}[H]
\centering
\includegraphics[scale=0.75]{img/diagramas/Datos/trabajador.png}
\caption{Tabla Trabajador en FNBC}\label{fig:Tabla Trabajador en FNBC}   
\end{figure}

\subsection{Tabla Especialidad}
\begin{itemize}
    \item \textbf{Descripción}: la tabla \textit{Especialidad} se obtiene a partir de la entidad \textit{Especialidad}
    (modelo \texttt{Speciality} de Laravel). Almacena las secciones/especialidades del hospital a las que pertenecen
    los trabajadores y sobre las que se organizan las guardias.

    \item \textbf{Nombre de la tabla}: Especialidad (\texttt{speciality})

    \item \textbf{Atributos}:
    \begin{itemize}
        \item Clave primaria: \underline{id}
        \item Claves alternas: ---
        \item Claves foráneas: ---
        \item Resto de atributos: name, active
    \end{itemize}

    \item \textbf{Esquema relacional}:
    Especialidad(\textbf{\underline{id}}, name, active)
\end{itemize}
\subsection{Tabla Especialidad}
    \begin{itemize}
        \item \textbf{Claves candidatas}: id
        \item \textbf{Normalización}: La tabla Especialidad presenta una dependencia funcional formada por la clave primaria y el resto de atributos. La tabla se encuentra en FNBC: el único determinante funcional es la clave primaria, por lo que la dependencia funcional con el resto de atributos es completa.
    \end{itemize}

\begin{figure}[H]
\centering
\includegraphics[scale=0.75]{img/diagramas/Datos/especialidad.png}
\caption{Tabla Especialidad en FNBC}\label{fig:Tabla Especialidad en FNBC}   
\end{figure}

\subsection{Tabla Guardia}
\begin{itemize}
    \item \textbf{Descripción}: la tabla \textit{Guardia} se obtiene a partir de la entidad \textit{Guardia}
    (modelo \texttt{Duty} de Laravel). Representa cada guardia asignada en una fecha concreta, indicando su tipo,
    la especialidad asociada, el trabajador asignado y, cuando corresponda, el trabajador que actúa como jefe de guardia.

    \item \textbf{Nombre de la tabla}: Guardia (\texttt{duties})

    \item \textbf{Atributos}:
    \begin{itemize}
        \item Clave primaria: \underline{id}
        \item Claves alternas: ---
        \item Claves foráneas: \textbf{id\_speciality} $\rightarrow$ Especialidad(id), \textbf{id\_worker} $\rightarrow$ Trabajador(id), \textbf{id\_chief\_worker} $\rightarrow$ Trabajador(id)
        \item Resto de atributos: date, duty\_type
    \end{itemize}

    \item \textbf{Esquema relacional}:
    Guardia(\textbf{\underline{id}}, date, duty\_type, \textbf{id\_speciality}, \textbf{id\_worker}, \textbf{id\_chief\_worker})
\end{itemize}
\subsection{Tabla Guardia}
    \begin{itemize}
        \item \textbf{Claves candidatas}: id
        \item \textbf{Normalización}: La tabla Guardia presenta una dependencia funcional formada por la clave primaria y el resto de atributos. La tabla se encuentra en FNBC: el único determinante funcional es la clave primaria, por lo que la dependencia funcional con el resto de atributos es completa.
    \end{itemize}

\begin{figure}[H]
\centering
\includegraphics[scale=0.75]{img/diagramas/Datos/guardia.png}
\caption{Tabla Guardia en FNBC}\label{fig:Tabla Guardia en FNBC}   
\end{figure}


%%
 \section{Esquema relacional}\label{sec:esquema-relacional}

\begin{table}[H]
\begin{center}
    \begin{tabular}{|l|p{8cm}|}
    \hline
    Tabla                 &   Atributos
    \\ \hline

    Usuario             &   (\textbf{\underline{id}}, name, email, password, avatarUrl, \textbf{worker\_id}, email\_verified)
    \\ \hline

    Trabajador          &   (\textbf{\underline{id}}, name, rank, registration\_date, discharge\_date, \textbf{id\_speciality})
    \\ \hline

    Especialidad        &   (\textbf{\underline{id}}, name, active)
    \\ \hline

    Guardia             &   (\textbf{\underline{id}}, date, duty\_type, \textbf{id\_speciality}, \textbf{id\_worker}, \textbf{id\_chief\_worker})
    \\ \hline

    \end{tabular}
\end{center}
\end{table}



\section{Diagrama relacional}\label{sec:diagrama-relacional}
El diagrama relacional del modelo propuesto se muestra en la figura \ref{fig:Diagrama Relacional}.

\begin{landscape}
\begin{figure}[H]
\centering
%\includegraphics[scale=0.3]{img/diagramas/Datos/Diagrama-relacional.png}
\caption{Diagrama relacional}\label{fig:Diagrama Relacional}   
\end{figure}
\end{landscape}