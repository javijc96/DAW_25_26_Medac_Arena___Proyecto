\chapter{Objetivos}

\section{Objetivo principal}

El objetivo principal de este proyecto es desarrollar una aplicación web que permita la gestión de servicios de suscripción para el alquiler de material de gimnasia. La plataforma ofrecerá un sistema flexible y accesible que facilite a gimnasios, otras empresas y usuarios particulares el acceso a equipamiento deportivo sin necesidad de realizar grandes inversiones iniciales. La aplicación deberá garantizar una experiencia de uso intuitiva y segura, permitiendo la administración del catálogo de productos, la gestión de usuarios y el control de los planes de suscripción.

\section{Objetivos específicos}

Los objetivos específicos de este proyecto son los siguientes:

\begin{itemize}

    \item \textbf{Tipos de usuarios}
    \item[] Se deberán permitir los siguientes tipos de usuarios:
    \begin{itemize}

        \item \textbf{Administrador}
        \begin{itemize}
            \item Este usuario tendrá acceso completo al sistema y será responsable de la gestión del catálogo de material, los niveles de suscripción, los usuarios registrados y los aspectos administrativos de la plataforma.
            \item Podrá realizar tareas avanzadas como la actualización del inventario, la supervisión de pedidos, la resolución de incidencias y la administración de la información general del sistema.
        \end{itemize}

        \item \textbf{Usuario registrado}
        \begin{itemize}
            \item Podrá acceder al sistema mediante registro y gestionar su suscripción.
            \item Tendrá acceso al catálogo de material disponible para su nivel de suscripción.
            \item Podrá consultar su historial, modificar su plan de suscripción o gestionar sus datos personales.
        \end{itemize}

        \item \textbf{Usuario invitado}
        \begin{itemize}
            \item Podrá consultar información general, niveles de suscripción y catálogo básico.
            \item Tendrá funcionalidades limitadas hasta que complete el registro.
        \end{itemize}

    \end{itemize}

    \item \textbf{Base de datos relacional}
    \item[] Se deberá diseñar una base de datos relacional que permita gestionar de forma eficiente toda la información relacionada con la aplicación. En particular:
    \begin{itemize}
        \item Gestión de usuarios registrados (administrador y usuarios suscritos).
        \item Gestión del catálogo de material deportivo disponible para alquiler.
        \item Gestión de niveles o planes de suscripción.
        \item Gestión de pedidos, renovaciones, devoluciones y estado del material.
        \item Registro de movimientos, operaciones y actividad dentro de la plataforma.
    \end{itemize}

    \item \textbf{Módulos}
    \item[] Se deberán diseñar los siguientes módulos principales:
    \begin{itemize}

        \item \textbf{Módulo del administrador}
        \begin{itemize}
            \item Gestión completa de usuarios.
            \item Gestión del catálogo de material deportivo.
            \item Administración de los niveles de suscripción.
            \item Revisión del estado del inventario y control del material disponible.
            \item Gestión de incidencias, solicitudes especiales y comunicación con usuarios.
            \item Consulta del registro de actividad del sistema.
        \end{itemize}

        \item \textbf{Módulo del usuario registrado}
        \begin{itemize}
            \item Consulta del catálogo según su nivel de suscripción.
            \item Realización de solicitudes de alquiler y renovación.
            \item Gestión de su cuenta y suscripción.
            \item Acceso a soporte y documentación.
        \end{itemize}

        \item \textbf{Módulo público}
        \begin{itemize}
            \item Consulta de información general sobre la plataforma.
            \item Consulta de los diferentes planes de suscripción disponibles.
            \item Acceso a preguntas frecuentes e información de ayuda.
        \end{itemize}

    \end{itemize}

    \item \textbf{Diseño de la interfaz}
    \item[] Se deberá diseñar una interfaz intuitiva, sencilla y compatible con los distintos navegadores web. Además, será responsiva para permitir su uso en móviles, tablets y ordenadores de escritorio. La interfaz deberá facilitar la navegación y ofrecer una buena experiencia para cualquier usuario.

\end{itemize}

\section{Objetivos personales}

Se proponen los siguientes objetivos personales:
\begin{itemize}
    \item Aprender sobre el desarrollo de aplicaciones web.
    \item Profundizar en la construcción de bases de datos relacionales.
    \item Mejorar las habilidades de análisis, diseño y programación.
    \item Desarrollar buenas prácticas en seguridad, mantenimiento y escalabilidad de software.
    \item Lidiar con un proyecto complejo en su totalidad, desde el principio hasta el final.
    \item Tener a Raúl contento.
\end{itemize}
