\chapter{Objetivos}
\section{Objetivo principal}
El objetivo principal de este proyecto es el desarrollo de una aplicación web que permita la 
gestión automatizada y centralizada de las guardias del hospital de Pozoblanco, 
sustituyendo el actual sistema manual basado en hojas de cálculo por una solución informática 
fiable, segura y fácil de usar.

La aplicación web se desarrollará utilizando el framework Laravel y estará orientada a mejorar 
la eficiencia en la planificación de turnos, reducir errores humanos y facilitar el acceso a 
la información tanto para los administradores como para el personal del hospital.

\section{Objetivos específicos}

Los objetivos específicos de este proyecto son los siguientes:

\begin{itemize}
\item \textbf{Tipos de usuarios}
\item[] Se deberán permitir los siguientes tipos de usuarios:
\begin{itemize}
    \item \textbf{Administrador}
    \begin{itemize}
        \item Este tipo de usuario estará registrado en el sistema y dispondrá de control completo sobre la aplicación.
        \item Tendrá competencias exclusivas para la gestión de usuarios, creación y modificación de guardias, asignación de turnos, consulta de estadísticas y generación de informes.
    \end{itemize}

    \item \textbf{Usuario estándar}
    \begin{itemize}
        \item Podrá consultar las guardias que tiene asignadas, así como su histórico de horas trabajadas y días libres.
        \item No dispondrá de permisos para modificar información crítica del sistema.
    \end{itemize}
\end{itemize}

\item \textbf{Base de datos relacional}
\item[] Se deberá diseñar una base de datos relacional que permita gestionar toda la información relacionada con el sistema de guardias, incluyendo:
\begin{itemize}
    \item Usuarios registrados y sus roles.
    \item Guardias y turnos asignados.
    \item Fechas, horarios y tipos de guardia.
    \item Históricos de asignaciones y modificaciones.
\end{itemize}

\item \textbf{Módulos del sistema}
\item[] Se deberán diseñar e implementar los siguientes módulos principales:
\begin{itemize}
    \item \textbf{Módulo del administrador}
    \begin{itemize}
        \item Gestión de usuarios registrados.
        \item Creación, modificación y eliminación de guardias.
        \item Asignación y reorganización de turnos.
        \item Generación de informes y consultas.
    \end{itemize}

    \item \textbf{Módulo del usuario estándar}
    \begin{itemize}
        \item Consulta de guardias asignadas.
        \item Visualización de calendario de turnos.
        \item Acceso a información personal relacionada con horas trabajadas.
    \end{itemize}
\end{itemize}

\item \textbf{Diseño de la interfaz}
\item[] Se deberá diseñar una interfaz de usuario intuitiva, robusta, amigable y adaptable a distintos dispositivos y navegadores web, garantizando una experiencia de uso satisfactoria.

\item \textbf{Seguridad y control de acceso}
\item[] Se deberán implementar mecanismos de autenticación y autorización que garanticen que cada usuario solo pueda acceder a las funcionalidades correspondientes a su rol.

\end{itemize}

\section{Objetivos personales}

Se proponen los siguientes objetivos personales:

\begin{itemize}
\item Aprender y afianzar conocimientos en el desarrollo de aplicaciones web utilizando el framework Laravel y React.
\item Profundizar en el uso del patrón Modelo-Vista-Controlador (MVC).
\item Mejorar habilidades en el diseño de bases de datos relacionales.
\item Aplicar buenas prácticas de desarrollo de software y documentación técnica.
\item Adquirir experiencia en el análisis, diseño, implementación y pruebas de un proyecto completo.
\item Conseguir que todos los miembros del grupo mejoren y se adapten al trabajo en equipo y tener a Raul contento.
\end{itemize}