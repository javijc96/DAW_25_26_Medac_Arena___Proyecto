\chapter{CONTEXTUALIZACIÓN EMPRESARIAL}

\section{Denominación}

    El proyecto consiste en el desarrollo de una aplicación web denominada \textbf{GuardiApp},
    orientada a la gestión automatizada y centralizada de las guardias de un hospital, con el objetivo
    de sustituir los procesos manuales actuales basados en hojas de cálculo.

    GuardiApp permitirá la planificación automática de los cuadrantes mensuales, la consulta de
    guardias por parte del personal sanitario y la gestión administrativa de turnos, cambios y
    jefaturas, proporcionando una solución especializada para centros hospitalarios.

    A nivel conceptual, GuardiApp se plantea como una herramienta software adaptable a las
    necesidades específicas de hospitales de tamaño medio, priorizando la eficiencia, la
    automatización y la facilidad de uso.

    
\section{Justificación de la idea del proyecto}

    La idea del proyecto surge a partir de la detección de una necesidad real en el entorno
    hospitalario: la gestión de guardias continúa realizándose, en muchos casos, mediante procesos
    manuales apoyados en hojas de cálculo y documentos generados de forma artesanal.

    Este enfoque conlleva una elevada carga administrativa, un alto riesgo de errores humanos y una
    escasa trazabilidad de los cambios realizados, especialmente en lo relativo a modificaciones de
    guardias, asignación de jefaturas y validaciones.

    GuardiApp pretende dar respuesta a esta problemática mediante la automatización del proceso de
    planificación de guardias, ofreciendo una solución centralizada que facilite tanto el trabajo
    del personal administrativo como el acceso a la información por parte de los profesionales
    sanitarios.
    
\section{Estudio de la competencia}

    Como competidores directos se pueden considerar aplicaciones de gestión hospitalaria que
    incluyen módulos de planificación de personal. Estas soluciones suelen ser completas, pero
    presentan un elevado coste económico y una complejidad de implantación que dificulta su adopción
    en hospitales de tamaño medio.

    Por otro lado, existen competidores indirectos como herramientas genéricas de gestión de turnos
    o el uso continuado de hojas de cálculo, que, aunque no están diseñadas específicamente para la
    gestión de guardias hospitalarias, siguen utilizándose por su bajo coste y facilidad de acceso.

    GuardiApp se diferencia de estas alternativas al ofrecer una solución específica, flexible y
    adaptada a la realidad de los hospitales, centrada en la automatización de guardias y en la
    simplificación de los procesos administrativos.

\section{Oportunidad de negocio}

    La transformación digital del sector sanitario es una tendencia en crecimiento, impulsada por
    la necesidad de mejorar la eficiencia operativa, reducir costes y garantizar la trazabilidad de
    la información.

    La gestión de guardias representa un proceso crítico dentro de cualquier hospital y, sin
    embargo, no siempre cuenta con herramientas específicas que se adapten a las necesidades reales
    de cada centro.

    GuardiApp cubre una necesidad existente en el mercado, ofreciendo una solución que puede ser
    implantada de forma progresiva y con un coste reducido en comparación con sistemas comerciales
    complejos, lo que la convierte en una opción viable para hospitales de tamaño medio.

    Por todo ello, el proyecto presenta una oportunidad de negocio clara dentro del ámbito de la
    digitalización de procesos administrativos en el sector sanitario.

\section{Obligaciones fiscales}


    El desarrollo de GuardiApp, en el contexto académico del proyecto, no conlleva inicialmente la
    generación de obligaciones fiscales, laborales ni de prevención de riesgos laborales, más allá
    de las propias del desarrollo del software.

    En un escenario de explotación comercial de la aplicación, sería necesario cumplir con las
    obligaciones fiscales correspondientes a la actividad empresarial, así como con la normativa
    laboral vigente en caso de contratación de personal.

    Asimismo, dado que la aplicación gestionaría información sensible del ámbito sanitario, sería
    imprescindible garantizar el cumplimiento de la normativa vigente en materia de protección de
    datos personales, en especial el Reglamento General de Protección de Datos (RGPD).

\section{Financiación, ayudas y subvenciones}

    El desarrollo inicial de GuardiApp no requiere una inversión económica significativa, ya que se
    basa en el uso de tecnologías de software libre y herramientas de desarrollo gratuitas.

    En caso de evolucionar el proyecto hacia un producto comercial, podrían contemplarse distintas
    fuentes de financiación, como subvenciones públicas destinadas a la digitalización de servicios
    sanitarios, ayudas a la innovación tecnológica o programas de apoyo al emprendimiento.

    Del mismo modo, podrían establecerse acuerdos de colaboración con centros hospitalarios
    interesados en la implantación de soluciones tecnológicas adaptadas a sus necesidades específicas.

    



