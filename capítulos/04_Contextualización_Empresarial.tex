\chapter{CONTEXTUALIZACIÓN EMPRESARIAL}

\section{Denominación}    
    RentFit tiene la intención de cubrir las necesidades de particulares y empresas de disponer de material de gimnasia que puede llegar
    a resultar muy costoso a traves de servicios de alquiler llevados a traves de un sistema de subscripciones de distintos rangos 

\section{Justificación de la idea del proyecto}
    El origen de este proyecto surge de la evidente necesidad que presentan numerosos gimnasios, centros deportivos y entrenadores de acceder a material actualizado, de buena calidad y en óptimas condiciones. A esto se suma el elevado coste que suelen tener estos productos y la situación económica de muchos establecimientos, que dificulta su adquisición de manera permanente. Como resultado, estos negocios no siempre pueden asumir el impacto financiero que supone la compra constante de nuevo equipamiento.
    RentFit propone una alternativa innovadora frente a la compra definitiva de equipamiento deportivo, ofreciendo un servicio de alquiler flexible y accesible. Este modelo permite que gimnasios, centros deportivos y entrenadores puedan disponer de material actualizado y en buen estado sin tener que asumir el elevado coste de adquisición. De esta manera, RentFit facilita el acceso a recursos de calidad, reduce la carga económica de los establecimientos y contribuye a que puedan renovar o ampliar su equipamiento de forma más rápida, eficiente y sostenible. Además, el alquiler permite adaptar el material a las necesidades cambiantes de cada negocio, evitando inversiones innecesarias y mejorando su capacidad de respuesta ante variaciones en la demanda.
\section{Estudio de la competencia}

Procederéis a identificar quiénes son las empresas con proyectos en marcha que puedan considerarse competidores (hay que tener en cuenta el concepto de competencia en su sentido más amplio, es decir, no sólo existen competidores directos, sino que también podemos tener competidores indirectos). 
    La competencia real a RentFit resulta componerse mas por competencia indirecta que por competencia directa ya que existen pocas empresas que se dediquen
    al alquiler de este tipo de material que operen en Córdoba. 

Competencia directa

Aunque limitada, sí existen algunas empresas que ofrecen servicios similares al modelo planteado por RentFit. Una de las más destacadas es FitnessRent, empresa dedicada al alquiler de equipamiento deportivo, principalmente para hoteles, anuncios publicitarios y eventos temporales.

Si bien su público objetivo difiere del de RentFit —centrado más en entornos corporativos, hoteleros o promocionales—, el hecho de ofrecer un servicio de alquiler de material deportivo la convierte en un competidor directo relevante. Representa un ejemplo claro de que existe demanda para un modelo basado en el acceso temporal al equipamiento sin necesidad de realizar una compra definitiva.

Competencia indirecta

La mayoría de competidores de RentFit pertenecen a la categoría de competencia indirecta. Se trata de empresas que no alquilan material, pero que ofrecen una alternativa válida mediante la venta directa de equipamiento deportivo, obligando al cliente a realizar una inversión inicial mayor.

Entre estas empresas destacan:

Decathlon

Sprinter

ProFit

Otros distribuidores especializados en material deportivo y fitness

Estas compañías disponen de una amplia variedad de productos y cuentan con presencia consolidada en el mercado. Sin embargo, su modelo de negocio exige a los gimnasios y entrenadores asumir el coste completo del equipamiento, lo que RentFit busca precisamente evitar ofreciendo un sistema más flexible y económicamente accesible.

\section{Oportunidad de negocio}

Oportunidad de negocio

El sector del fitness y la actividad física en España se encuentra en crecimiento constante, tanto en número de usuarios como en demanda de servicios profesionales. RentFit surge como respuesta directa a una necesidad que va en aumento: la de disponer de material deportivo actualizado y en buen estado sin tener que asumir el elevado coste que implica la compra de equipamiento.

Demanda del mercado

Actualmente, muchos gimnasios, entrenadores personales y centros deportivos, especialmente los de menor tamaño, tienen limitaciones económicas que dificultan la renovación o ampliación de su equipamiento. Esta situación genera una demanda creciente de soluciones más flexibles y accesibles.
El alquiler de material deportivo se presenta como una alternativa que:

Reduce la inversión inicial necesaria.

Permite adaptar el equipamiento a las necesidades del momento.

Evita la obsolescencia del material.

Facilita el acceso a productos de mayor calidad.

Aunque el modelo de alquiler en este sector aún no está ampliamente implantado en Córdoba, esta ausencia de competidores directos representa una oportunidad clara para introducir un servicio innovador con un mercado potencial significativo.

Viabilidad económica

Desde el punto de vista económico, RentFit resulta viable por varias razones:

Bajo nivel de competencia directa, lo que permite posicionarse como la alternativa principal en la ciudad.

Elevada demanda potencial, derivada del crecimiento del entrenamiento personal, centros boutique y gimnasios independientes.

Ingresos recurrentes mediante modelos de suscripción o alquiler mensual.

Rentabilidad del equipamiento, ya que el material puede generar beneficios de forma continua durante toda su vida útil.

Flexibilidad para escalar: es posible comenzar con un inventario reducido y ampliarlo según la demanda.

Conclusión

La combinación de una necesidad real, una competencia directa limitada y un modelo económico basado en ingresos recurrentes hace que RentFit represente una oportunidad de negocio sólida y con margen de crecimiento. Su propuesta responde a una carencia del mercado cordobés, ofreciendo una solución innovadora, rentable y adaptable a las necesidades cambiantes del sector fitness.

\section{Obligaciones fiscales}

Se debe especificar si el desarrollo del proyecto conlleva la necesidad de cumplimiento de obligaciones fiscales, laborales y/o de prevención de riesgos y sus condiciones de aplicación.
    
\section{Financiación, ayudas y subvenciones}

Habría que identificar las posibles necesidades de financiación para llevar a cabo el desarrollo del proyecto, así como la búsqueda de ayudas o subvenciones para la incorporación de las nuevas tecnologías de producción o de servicio propuestas.


    



