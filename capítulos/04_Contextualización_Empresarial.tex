\chapter{CONTEXTUALIZACIÓN EMPRESARIAL}

\section{Denominación}

En este primer apartado debéis explicar brevemente en qué consiste vuestra idea de negocio del proyecto y cuál es su nombre comercial (logo).
    
\section{Justificación de la idea del proyecto}
    
En este punto se deben exponer cuáles son las razones que os han llevado a poner en marcha vuestro proyecto.
 
\section{Estudio de la competencia}

Procederéis a identificar quiénes son las empresas con proyectos en marcha que puedan considerarse competidores (hay que tener en cuenta el concepto de competencia en su sentido más amplio, es decir, no sólo existen competidores directos, sino que también podemos tener competidores indirectos). 

\section{Oportunidad de negocio}

 En este apartado se deberían valorar las oportunidades de negocio previsibles en el sector. ¿Es un producto o servicio que está demandando el mercado? ¿Económicamente es viable? Demuéstralo brevemente.

\section{Obligaciones fiscales}

Se debe especificar si el desarrollo del proyecto conlleva la necesidad de cumplimiento de obligaciones fiscales, laborales y/o de prevención de riesgos y sus condiciones de aplicación.
    
\section{Financiación, ayudas y subvenciones}

Habría que identificar las posibles necesidades de financiación para llevar a cabo el desarrollo del proyecto, así como la búsqueda de ayudas o subvenciones para la incorporación de las nuevas tecnologías de producción o de servicio propuestas.


    



