\chapter{Modelo de datos} \label{cap:modelo_de_datos}

\section{Introducción}

En este capítulo se describirá conceptualmente el modelo de datos que empleará el sistema. Se identificarán las entidades que intervienen en el problema, sus atributos y las interrelaciones entre dichas entidades.

Para la confección del modelo, se empleará la notación del Modelo Entidad - Interrelación (modelo E-R) propuesto por Peter Chen%\cite{peter}. 
El esquema E-R describe la información representando los distintos elementos que la componen mediante un conjunto limitado de símbolos y reglas de relación entre ellos. Básicamente, los elementos principales del modelo son “tipo de entidad” y “tipo de interrelación”.

En las siguientes secciones se especifican con detalle los tipos de entidad y los tipos de interrelación que intervienen en el modelo, y se mostrará el diagrama E-R completo, que ofrecerá una visión global del problema.

\section{Tipos de entidad} \label{sec:entidades}

De acuerdo con el modelo E-R, un tipo de entidad representa a una serie de entes, objetos o personas reales o abstractos que forman parte del universo del problema a describir. Los tipos de entidad pueden ser fuertes o débiles.

\begin{itemize}

    \item Tipo de entidad fuerte: su existencia no depende de la de otro tipo de entidad.

    \item Tipo de entidad débil: su existencia depende de la de otro tipo de entidad. La debilidad puede ser por existencia o por identificación.

    \begin{itemize}
        \item Tipo de entidad débil por existencia: puede ser identificado por sí mismo a partir de sus atributos propios, pero requiere de la existencia de otro tipo de entidad del que depende.

        \item Tipo de entidad débil por identificación: es un tipo de entidad débil por existencia que, además, requiere de algún atributo identificativo del tipo de entidad del que depende para poder ser identificado y diferenciado.
    \end{itemize} 
\end{itemize}    

Las entidades de un determinado tipo de entidad se describen mediante un conjunto de atributos que representan cada una de las características o propiedades que lo describen.

Cada atributo tiene asociado un dominio de valores permitidos. Cada entidad se identifica y diferencia de forma inequívoca mediante un atributo identificador, que toma un valor único para cada entidad.

En esta sección se describirán todos los tipos de entidad que se han identificado que forman parte del problema descrito, indicando para cada uno de ellos la siguiente información:

\begin{itemize}
    \item Descripción: definición de la entidad y función dentro del universo del problema.
    
    \item Restricción: indicación de si tiene alguna debilidad por identificación o existencia respecto de otra entidad.
    
    \item Características: se indicarán las siguientes: 
    \begin{itemize}
        \item Nombre del tipo de entidad.
        \item Tipo: Fuerte o débil.
        \item Atributos heredados.
        \item Atributo identificador primario.
        \item Atributo identificador alternativo.
        \item Número de atributos, incluyendo los heredados.
    \end{itemize}
    
    \item Atributos propios: por cada atributo, además del nombre del mismo, se indicará:
    \begin{itemize}
        \item Definición: descripción del atributo.
        \item Dominio: tipo de dato o valores que puede tomar el atributo.
        \item Tipo: indicación de si es clave primaria o alterna, en su caso, atributo simple, etc.
        \item Opcional: indicación de si el atributo puede contener un valor nulo o no.
        \item Ejemplo: valor de muestra.
    \end{itemize}
    
    \item Diagrama: representación gráfica del tipo de entidad, de acuerdo con la notación E-R.
    
    \item Ejemplo de entidad.
\end{itemize}    

Los tipos de entidad que se han identificado y que se describirán a continuación son los siguientes:
\begin{itemize}
    \item Tipo de entidad: Usuario
    \item Tipo de entidad: Rol
    \item Tipo de entidad: Dirección
    \item Tipo de entidad: Envio\_Servicio
    \item Tipo de entidad: Contrato
    \item Tipo de entidad: Subscricion
    \item Tipo de entidad: Tipo
    \item Tipo de entidad: Producto
    \item Tipo de entidad: Categorias
    \item Tipo de entidad: Contrato\_productos
\end{itemize}


\subsection{Entidad Usuario}
\begin{itemize}
    \item \textbf{Descripción:} este tipo de entidad representa a los usuarios registrados en el sistema que pueden contratar servicios de suscripción, gestionar productos y realizar operaciones dentro de la plataforma. Cada usuario dispone de un rol que determina sus permisos de acceso y funcionalidades disponibles.

    \item \textbf{Restricciones:} es una entidad fuerte, por lo que no depende de ningún otro tipo de entidad para existir.


    \item \textbf{Características:}
    \begin{itemize}
        \item Nombre del tipo de entidad: Usuario.
        \item Tipo: Fuerte.
        \item Atributos heredados: Ninguno.
        \item Atributo identificador primario: id.
        \item Atributo identificador alternativo: email.
        \item Número de atributos: 8 propios.
    \end{itemize}

    \item Atributos propios:
    \begin{itemize}
        \item id
        \begin{itemize}
            \item Definición: código numérico secuencial e incremental que identifica el usuario.
            \item Dominio: números enteros mayores que 0.
            \item Tipo: clave primaria.
            \item Opcional: no
            \item Ejemplo: 13
        \end{itemize}

        \item \textbf{nombre}
        \begin{itemize}
            \item Definición: nombre del usuario.
            \item Dominio: cadena de hasta 150 caracteres.
            \item Tipo: atributo simple.
            \item Opcional: no.
            \item Ejemplo: Javier.
        \end{itemize}

        \item \textbf{apellidos}
        \begin{itemize}
            \item Definición: apellidos del usuario.
            \item Dominio: cadena de hasta 150 caracteres.
            \item Tipo: atributo simple.
            \item Opcional: no.
            \item Ejemplo: Martínez López.
        \end{itemize}

        \item \textbf{telefono}
        \begin{itemize}
            \item Definición: número de teléfono de contacto del usuario.
            \item Dominio: cadena numérica de entre 9 y 15 dígitos.
            \item Tipo: atributo simple.
            \item Opcional: no.
            \item Ejemplo: 678945321.
        \end{itemize}

        \item \textbf{email}
        \begin{itemize}
            \item Definición: dirección de correo electrónico del usuario.
            \item Dominio: cadena de hasta 250 caracteres con formato email válido.
            \item Tipo: atributo simple / clave alternativa.
            \item Opcional: no.
            \item Ejemplo: usuario@example.com.
        \end{itemize}

        \item \textbf{contraseña}
        \begin{itemize}
            \item Definición: contraseña encriptada utilizada para autenticar al usuario.
            \item Dominio: cadena de hasta 250 caracteres.
            \item Tipo: atributo simple.
            \item Opcional: no.
            \item Ejemplo: \$2y\$10\$XHfj23bd.... (hash).
        \end{itemize}

        \item \textbf{fecha\_alta}
        \begin{itemize}
            \item Definición: fecha en la que el usuario se registró en la plataforma.
            \item Dominio: tipo fecha.
            \item Tipo: atributo simple.
            \item Opcional: no.
            \item Ejemplo: 2025-02-10.
        \end{itemize}

        \item \textbf{id\_rol}
        \begin{itemize}
            \item Definición: identificador del rol que tiene asignado el usuario.
            \item Dominio: número entero positivo.
            \item Tipo: clave foránea.
            \item Opcional: no.
            \item Ejemplo: 1.
        \end{itemize}

    \end{itemize}

  \item \textbf{Diagrama (Figura \ref{fig:E-Usuario}):}

    \begin{figure}[H]
        \centering
    \includegraphics[scale=0.3]{img/diagramas/usuario.png}
        \caption{Entidad Usuario}
        \label{fig:E-Usuario}
    \end{figure}

    \item \textbf{Ejemplo de entidad (Tabla \ref{table:T-Usuario}):}

    \begin{table}[H]
    \centering
        \begin{tabular}{ |p{6cm}||p{6cm}|  }
             \hline
                \multicolumn{2}{|c|}{\textbf{Usuario}} \\ \hline
                 \textbf{Atributo} & \textbf{Valor} \\ \hline
                 id & 5 \\ \hline
                 nombre & Antonio \\ \hline
                 apellidos & Núñez Ortega \\ \hline
                 telefono & 665874321 \\ \hline
                 email & antonio@ejemplo.com \\ \hline
                 contraseña & \$2y\$10\$7sd... \\ \hline
                 fecha\_alta & 2025-01-12 \\ \hline
                 id\_rol & 2 \\ \hline
        \end{tabular}
        \caption{Ejemplo de la entidad \textit{Usuario}}
        \label{table:T-Usuario}
    \end{table}

\end{itemize}

\subsection{Entidad Rol}
\begin{itemize}
    \item \textbf{Descripción:} este tipo de entidad representa los distintos roles que pueden tener los usuarios dentro del sistema (por ejemplo, administrador o usuario estándar). El rol determina los permisos y funcionalidades a las que puede acceder cada usuario.

    \item \textbf{Restricciones:} es una entidad fuerte.

    \item \textbf{Características:}
    \begin{itemize}
        \item Nombre del tipo de entidad: Rol.
        \item Tipo: Fuerte.
        \item Atributos heredados: Ninguno.
        \item Atributo identificador primario: id.
        \item Atributo identificador alternativo: ninguno.
        \item Número de atributos: 2 propios.
    \end{itemize}

    \item Atributos propios:
    \begin{itemize}
        \item \textbf{id}
        \begin{itemize}
            \item Definición: identificador único del rol.
            \item Dominio: números enteros mayores que 0.
            \item Tipo: clave primaria.
            \item Opcional: no.
            \item Ejemplo: 1.
        \end{itemize}

        \item \textbf{nombre}
        \begin{itemize}
            \item Definición: nombre del rol asignado a los usuarios.
            \item Dominio: cadena de hasta 100 caracteres.
            \item Tipo: atributo simple.
            \item Opcional: no.
            \item Ejemplo: Administrador.
        \end{itemize}
    \end{itemize}

    \item \textbf{Diagrama (Figura \ref{fig:E-Rol}):}

    \begin{figure}[H]
        \centering
        \includegraphics[scale=0.3]{img/diagramas/rol.png}
        \caption{Entidad Rol}
        \label{fig:E-Rol}
    \end{figure}

    \item \textbf{Ejemplo de entidad (Tabla \ref{table:T-Rol}):}

    \begin{table}[H]
    \centering
        \begin{tabular}{ |p{6cm}||p{6cm}|  }
             \hline
                \multicolumn{2}{|c|}{\textbf{Rol}} \\ \hline
                 \textbf{Atributo} & \textbf{Valor} \\ \hline
                 id & 1 \\ \hline
                 nombre & Administrador \\ \hline
        \end{tabular}
        \caption{Ejemplo de la entidad \textit{Rol}}
        \label{table:T-Rol}
    \end{table}

\end{itemize}



\subsection{Entidad Dirección}
\begin{itemize}
    \item \textbf{Descripción:} este tipo de entidad almacena las direcciones asociadas a los usuarios, utilizadas para la entrega de material o prestación de servicios.

    \item \textbf{Restricciones:} es una entidad fuerte.

    \item \textbf{Características:}
    \begin{itemize}
        \item Nombre del tipo de entidad: Dirección.
        \item Tipo: Fuerte.
        \item Atributos heredados: Ninguno.
        \item Atributo identificador primario: id.
        \item Atributo identificador alternativo: ninguno.
        \item Número de atributos: 5 propios.
    \end{itemize}

    \item Atributos propios:
    \begin{itemize}

        \item \textbf{id}
        \begin{itemize}
            \item Definición: identificador único de la dirección.
            \item Dominio: entero positivo.
            \item Tipo: clave primaria.
            \item Opcional: no.
            \item Ejemplo: 3.
        \end{itemize}

        \item \textbf{id\_usuario}
        \begin{itemize}
            \item Definición: identificador del usuario al que pertenece la dirección.
            \item Dominio: entero positivo.
            \item Tipo: clave foránea.
            \item Opcional: no.
            \item Ejemplo: 5.
        \end{itemize}

        \item \textbf{via}
        \begin{itemize}
            \item Definición: nombre de la vía y número (calle, avenida, etc.).
            \item Dominio: cadena de hasta 200 caracteres.
            \item Tipo: atributo simple.
            \item Opcional: no.
            \item Ejemplo: Calle Mayor 15, 3ºB.
        \end{itemize}

        \item \textbf{cod\_postal}
        \begin{itemize}
            \item Definición: código postal de la dirección.
            \item Dominio: cadena de 5 a 10 caracteres.
            \item Tipo: atributo simple.
            \item Opcional: no.
            \item Ejemplo: 14004.
        \end{itemize}

        \item \textbf{Ciudad}
        \begin{itemize}
            \item Definición: ciudad en la que se encuentra la dirección.
            \item Dominio: cadena de hasta 100 caracteres.
            \item Tipo: atributo simple.
            \item Opcional: no.
            \item Ejemplo: Córdoba.
        \end{itemize}

    \end{itemize}

    \item \textbf{Diagrama (Figura \ref{fig:E-Direccion}):}

    \begin{figure}[H]
        \centering
        \includegraphics[scale=0.3]{img/diagramas/direccion.png}
        \caption{Entidad Dirección}
        \label{fig:E-Direccion}
    \end{figure}

    \item \textbf{Ejemplo de entidad (Tabla \ref{table:T-Direccion}):}

    \begin{table}[H]
    \centering
        \begin{tabular}{ |p{6cm}||p{6cm}|  }
             \hline
                \multicolumn{2}{|c|}{\textbf{Dirección}} \\ \hline
                 \textbf{Atributo} & \textbf{Valor} \\ \hline
                 id & 3 \\ \hline
                 id\_usuario & 5 \\ \hline
                 via & Calle Mayor 15, 3ºB \\ \hline
                 cod\_postal & 14004 \\ \hline
                 Ciudad & Córdoba \\ \hline
        \end{tabular}
        \caption{Ejemplo de la entidad \textit{Dirección}}
        \label{table:T-Direccion}
    \end{table}

\end{itemize}



\subsection{Entidad Envio\_Servicio}
\begin{itemize}
    \item \textbf{Descripción:} este tipo de entidad representa los servicios de envío disponibles para una determinada dirección, incluyendo su coste y tiempo de envío.

    \item \textbf{Restricciones:} entidad fuerte.

    \item \textbf{Características:}
    \begin{itemize}
        \item Nombre del tipo de entidad: Envio\_Servicio.
        \item Tipo: Fuerte.
        \item Atributos heredados: Ninguno.
        \item Atributo identificador primario: id.
        \item Atributo identificador alternativo: ninguno.
        \item Número de atributos: 6 propios.
    \end{itemize}

    \item Atributos propios:
    \begin{itemize}

        \item \textbf{id}
        \begin{itemize}
            \item Definición: identificador único del servicio de envío.
            \item Dominio: entero positivo.
            \item Tipo: clave primaria.
            \item Opcional: no.
            \item Ejemplo: 2.
        \end{itemize}

        \item \textbf{nombre}
        \begin{itemize}
            \item Definición: nombre comercial del servicio de envío.
            \item Dominio: cadena de hasta 150 caracteres.
            \item Tipo: atributo simple.
            \item Opcional: no.
            \item Ejemplo: Envío estándar 48h.
        \end{itemize}

        \item \textbf{tempo\_envio}
        \begin{itemize}
            \item Definición: tiempo estimado de envío.
            \item Dominio: cadena de hasta 50 caracteres.
            \item Tipo: atributo simple.
            \item Opcional: no.
            \item Ejemplo: 48 horas.
        \end{itemize}

        \item \textbf{coste\_base}
        \begin{itemize}
            \item Definición: coste base asociado al servicio de envío.
            \item Dominio: número real positivo.
            \item Tipo: atributo simple.
            \item Opcional: no.
            \item Ejemplo: 9.99.
        \end{itemize}

        \item \textbf{activo}
        \begin{itemize}
            \item Definición: indica si el servicio de envío está disponible.
            \item Dominio: valor booleano (true/false).
            \item Tipo: atributo simple.
            \item Opcional: no.
            \item Ejemplo: true.
        \end{itemize}

        \item \textbf{id\_direccion}
        \begin{itemize}
            \item Definición: identificador de la dirección asociada al servicio de envío.
            \item Dominio: entero positivo.
            \item Tipo: clave foránea.
            \item Opcional: no.
            \item Ejemplo: 3.
        \end{itemize}

    \end{itemize}

    \item \textbf{Diagrama (Figura \ref{fig:E-EnvioServicio}):}

    \begin{figure}[H]
        \centering
        \includegraphics[scale=0.3]{img/diagramas/envio_servicio.png}
        \caption{Entidad Envio\_Servicio}
        \label{fig:E-EnvioServicio}
    \end{figure}

    \item \textbf{Ejemplo de entidad (Tabla \ref{table:T-EnvioServicio}):}

    \begin{table}[H]
    \centering
        \begin{tabular}{ |p{6cm}||p{6cm}|  }
             \hline
                \multicolumn{2}{|c|}{\textbf{Envio\_Servicio}} \\ \hline
                 \textbf{Atributo} & \textbf{Valor} \\ \hline
                 id & 2 \\ \hline
                 nombre & Envío estándar 48h \\ \hline
                 tempo\_envio & 48 horas \\ \hline
                 coste\_base & 9.99 \\ \hline
                 activo & true \\ \hline
                 id\_direccion & 3 \\ \hline
        \end{tabular}
        \caption{Ejemplo de la entidad \textit{Envio\_Servicio}}
        \label{table:T-EnvioServicio}
    \end{table}

\end{itemize}



\subsection{Entidad Contrato}
\begin{itemize}
    \item \textbf{Descripción:} este tipo de entidad representa los contratos que realiza un usuario para la adquisición temporal de productos o servicios de suscripción.

    \item \textbf{Restricciones:} entidad fuerte.

    \item \textbf{Características:}
    \begin{itemize}
        \item Nombre del tipo de entidad: Contrato.
        \item Tipo: Fuerte.
        \item Atributos heredados: Ninguno.
        \item Atributo identificador primario: id.
        \item Atributo identificador alternativo: ninguno.
        \item Número de atributos: 6 propios.
    \end{itemize}

    \item Atributos propios:
    \begin{itemize}

        \item \textbf{id}
        \begin{itemize}
            \item Definición: identificador único del contrato.
            \item Dominio: entero positivo.
            \item Tipo: clave primaria.
            \item Opcional: no.
            \item Ejemplo: 10.
        \end{itemize}

        \item \textbf{id\_usuario}
        \begin{itemize}
            \item Definición: identificador del usuario que realiza el contrato.
            \item Dominio: entero positivo.
            \item Tipo: clave foránea.
            \item Opcional: no.
            \item Ejemplo: 5.
        \end{itemize}

        \item \textbf{id\_direccion}
        \begin{itemize}
            \item Definición: identificador de la dirección asociada al contrato.
            \item Dominio: entero positivo.
            \item Tipo: clave foránea.
            \item Opcional: no.
            \item Ejemplo: 3.
        \end{itemize}

        \item \textbf{fecha\_inicio}
        \begin{itemize}
            \item Definición: fecha de inicio del contrato.
            \item Dominio: tipo fecha.
            \item Tipo: atributo simple.
            \item Opcional: no.
            \item Ejemplo: 2025-03-01.
        \end{itemize}

        \item \textbf{fecha\_fin}
        \begin{itemize}
            \item Definición: fecha de finalización del contrato.
            \item Dominio: tipo fecha.
            \item Tipo: atributo simple.
            \item Opcional: sí (en caso de contratos abiertos).
            \item Ejemplo: 2025-09-01.
        \end{itemize}

        \item \textbf{descripcion}
        \begin{itemize}
            \item Definición: información adicional sobre el contrato.
            \item Dominio: texto de hasta 500 caracteres.
            \item Tipo: atributo simple.
            \item Opcional: sí.
            \item Ejemplo: Contrato de suscripción Premium para 6 meses.
        \end{itemize}

    \end{itemize}

    \item \textbf{Diagrama (Figura \ref{fig:E-Contrato}):}

    \begin{figure}[H]
        \centering
        \includegraphics[scale=0.3]{img/diagramas/contrato.png}
        \caption{Entidad Contrato}
        \label{fig:E-Contrato}
    \end{figure}

    \item \textbf{Ejemplo de entidad (Tabla \ref{table:T-Contrato}):}

    \begin{table}[H]
    \centering
        \begin{tabular}{ |p{6cm}||p{6cm}|  }
             \hline
                \multicolumn{2}{|c|}{\textbf{Contrato}} \\ \hline
                 \textbf{Atributo} & \textbf{Valor} \\ \hline
                 id & 10 \\ \hline
                 id\_usuario & 5 \\ \hline
                 id\_direccion & 3 \\ \hline
                 fecha\_inicio & 2025-03-01 \\ \hline
                 fecha\_fin & 2025-09-01 \\ \hline
                 descripcion & Contrato de suscripción Premium para 6 meses \\ \hline
        \end{tabular}
        \caption{Ejemplo de la entidad \textit{Contrato}}
        \label{table:T-Contrato}
    \end{table}

\end{itemize}
\subsection{Entidad Subscricion}
\begin{itemize}
    \item \textbf{Descripción:} este tipo de entidad representa las suscripciones activas o históricas asociadas a los contratos. Una suscripción define el tipo de plan contratado y su coste mensual.

    \item \textbf{Restricciones:} entidad fuerte.

    \item \textbf{Características:}
    \begin{itemize}
        \item Nombre del tipo de entidad: Subscricion.
        \item Tipo: Fuerte.
        \item Atributos heredados: Ninguno.
        \item Atributo identificador primario: id.
        \item Atributo identificador alternativo: ninguno.
        \item Número de atributos: 4 propios.
    \end{itemize}

    \item Atributos propios:
    \begin{itemize}

        \item \textbf{id}
        \begin{itemize}
            \item Definición: identificador único de la suscripción.
            \item Dominio: entero positivo.
            \item Tipo: clave primaria.
            \item Opcional: no.
            \item Ejemplo: 7.
        \end{itemize}

        \item \textbf{ID\_tipo}
        \begin{itemize}
            \item Definición: tipo de suscripción contratada.
            \item Dominio: entero positivo.
            \item Tipo: clave foránea.
            \item Opcional: no.
            \item Ejemplo: 2.
        \end{itemize}

        \item \textbf{ID\_contrato}
        \begin{itemize}
            \item Definición: identificador del contrato asociado.
            \item Dominio: entero positivo.
            \item Tipo: clave foránea.
            \item Opcional: no.
            \item Ejemplo: 10.
        \end{itemize}

        \item \textbf{mensualidad}
        \begin{itemize}
            \item Definición: coste mensual de la suscripción.
            \item Dominio: número real positivo.
            \item Tipo: atributo simple.
            \item Opcional: no.
            \item Ejemplo: 29.99.
        \end{itemize}

    \end{itemize}

    \item \textbf{Diagrama (Figura \ref{fig:E-Subscricion}):}

    \begin{figure}[H]
        \centering
        \includegraphics[scale=0.3]{img/diagramas/subscricion.png}
        \caption{Entidad Subscricion}
        \label{fig:E-Subscricion}
    \end{figure}

    \item \textbf{Ejemplo de entidad (Tabla \ref{table:T-Subscricion}):}

    \begin{table}[H]
    \centering
        \begin{tabular}{ |p{6cm}||p{6cm}| }
             \hline
             \multicolumn{2}{|c|}{\textbf{Subscricion}} \\ \hline
             \textbf{Atributo} & \textbf{Valor} \\ \hline
             id & 7 \\ \hline
             ID\_tipo & 2 \\ \hline
             ID\_contrato & 10 \\ \hline
             mensualidad & 29.99 \\ \hline
        \end{tabular}
        \caption{Ejemplo de la entidad \textit{Subscricion}}
        \label{table:T-Subscricion}
    \end{table}

\end{itemize}

\subsection{Entidad Tipo}
\begin{itemize}
    \item \textbf{Descripción:} este tipo de entidad define los tipos de suscripción disponibles (por ejemplo, Básica, Premium o Profesional), cada uno asociado a un precio, descripción y una imagen representativa.

    \item \textbf{Restricciones:} entidad fuerte.

    \item \textbf{Características:}
    \begin{itemize}
        \item Nombre del tipo de entidad: Tipo.
        \item Tipo: Fuerte.
        \item Atributos heredados: Ninguno.
        \item Atributo identificador primario: ID.
        \item Atributo identificador alternativo: ninguno.
        \item Número de atributos: 4 propios.
    \end{itemize}

    \item Atributos propios:
    \begin{itemize}

        \item \textbf{ID}
        \begin{itemize}
            \item Definición: identificador único del tipo de suscripción.
            \item Dominio: entero positivo.
            \item Tipo: clave primaria.
            \item Opcional: no.
            \item Ejemplo: 2.
        \end{itemize}

        \item \textbf{Nombre}
        \begin{itemize}
            \item Definición: nombre del tipo de suscripción.
            \item Dominio: cadena de hasta 150 caracteres.
            \item Tipo: atributo simple.
            \item Opcional: no.
            \item Ejemplo: Premium.
        \end{itemize}

        \item \textbf{Precio}
        \begin{itemize}
            \item Definición: coste base mensual del tipo de suscripción.
            \item Dominio: número real positivo.
            \item Tipo: atributo simple.
            \item Opcional: no.
            \item Ejemplo: 29.99.
        \end{itemize}

        \item \textbf{Descripcion}
        \begin{itemize}
            \item Definición: detalle del plan y sus beneficios.
            \item Dominio: cadena de hasta 500 caracteres.
            \item Tipo: atributo simple.
            \item Opcional: sí.
            \item Ejemplo: Acceso ilimitado a todo el catálogo.
        \end{itemize}

        \item \textbf{imagen}
        \begin{itemize}
            \item Definición: imagen asociada al tipo de suscripción.
            \item Dominio: ruta o nombre de archivo.
            \item Tipo: atributo simple.
            \item Opcional: sí.
            \item Ejemplo: premium.png.
        \end{itemize}

    \end{itemize}

    \item \textbf{Diagrama (Figura \ref{fig:E-Tipo}):}

    \begin{figure}[H]
        \centering
        \includegraphics[scale=0.3]{img/diagramas/tipo.png}
        \caption{Entidad Tipo}
        \label{fig:E-Tipo}
    \end{figure}

    \item \textbf{Ejemplo de entidad (Tabla \ref{table:T-Tipo}):}

    \begin{table}[H]
    \centering
        \begin{tabular}{ |p{6cm}||p{6cm}| }
             \hline
             \multicolumn{2}{|c|}{\textbf{Tipo}} \\ \hline
             \textbf{Atributo} & \textbf{Valor} \\ \hline
             ID & 2 \\ \hline
             Nombre & Premium \\ \hline
             Precio & 29.99 \\ \hline
             Descripcion & Acceso ilimitado a todo el catálogo \\ \hline
             imagen & premium.png \\ \hline
        \end{tabular}
        \caption{Ejemplo de la entidad \textit{Tipo}}
        \label{table:T-Tipo}
    \end{table}

\end{itemize}


\subsection{Entidad Producto}
\begin{itemize}
    \item \textbf{Descripción:} este tipo de entidad representa los productos deportivos disponibles para alquiler, con información como nombre, categoría, cantidad, imagen y características adicionales.

    \item \textbf{Restricciones:} entidad fuerte.

    \item \textbf{Características:}
    \begin{itemize}
        \item Nombre del tipo de entidad: Producto.
        \item Tipo: Fuerte.
        \item Atributos heredados: Ninguno.
        \item Atributo identificador primario: ID.
        \item Atributo identificador alternativo: ninguno.
        \item Número de atributos: 8 propios.
    \end{itemize}

    \item Atributos propios:
    \begin{itemize}

        \item \textbf{ID}
        \begin{itemize}
            \item Definición: identificador único del producto.
            \item Dominio: entero positivo.
            \item Tipo: clave primaria.
            \item Opcional: no.
            \item Ejemplo: 15.
        \end{itemize}

        \item \textbf{ID\_Categoria}
        \begin{itemize}
            \item Definición: categoría a la que pertenece el producto.
            \item Dominio: entero positivo.
            \item Tipo: clave foránea.
            \item Opcional: no.
            \item Ejemplo: 3.
        \end{itemize}

        \item \textbf{Nombre}
        \begin{itemize}
            \item Definición: nombre del producto.
            \item Dominio: cadena de hasta 150 caracteres.
            \item Tipo: atributo simple.
            \item Opcional: no.
            \item Ejemplo: Mancuernas 10 kg.
        \end{itemize}

        \item \textbf{Descripcion}
        \begin{itemize}
            \item Definición: información descriptiva del producto.
            \item Dominio: hasta 500 caracteres.
            \item Tipo: atributo simple.
            \item Opcional: sí.
            \item Ejemplo: Mancuernas recubiertas de goma antideslizante.
        \end{itemize}

        \item \textbf{Cantidad}
        \begin{itemize}
            \item Definición: número de unidades disponibles.
            \item Dominio: entero positivo.
            \item Tipo: atributo simple.
            \item Opcional: no.
            \item Ejemplo: 20.
        \end{itemize}

        \item \textbf{Instalacion}
        \begin{itemize}
            \item Definición: indica si el producto requiere instalación.
            \item Dominio: booleano.
            \item Tipo: atributo simple.
            \item Opcional: no.
            \item Ejemplo: false.
        \end{itemize}

        \item \textbf{espacio}
        \begin{itemize}
            \item Definición: espacio aproximado necesario para el producto.
            \item Dominio: cadena de hasta 50 caracteres.
            \item Tipo: atributo simple.
            \item Opcional: sí.
            \item Ejemplo: 1 m\textsuperscript{2}.
        \end{itemize}

        \item \textbf{Precio}
        \begin{itemize}
            \item Definición: precio del alquiler mensual.
            \item Dominio: número real positivo.
            \item Tipo: atributo simple.
            \item Opcional: no.
            \item Ejemplo: 12.99.
        \end{itemize}

        \item \textbf{Imagen}
        \begin{itemize}
            \item Definición: imagen ilustrativa del producto.
            \item Dominio: cadena con ruta o nombre de archivo.
            \item Tipo: atributo simple.
            \item Opcional: sí.
        \end{itemize}

    \end{itemize}

    \item \textbf{Diagrama (Figura \ref{fig:E-Producto}):}

    \begin{figure}[H]
        \centering
        \includegraphics[scale=0.3]{img/diagramas/producto.png}
        \caption{Entidad Producto}
        \label{fig:E-Producto}
    \end{figure}

    \item \textbf{Ejemplo de entidad (Tabla \ref{table:T-Producto}):}

    \begin{table}[H]
    \centering
        \begin{tabular}{ |p{6cm}||p{6cm}| }
             \hline
             \multicolumn{2}{|c|}{\textbf{Producto}} \\ \hline
             \textbf{Atributo} & \textbf{Valor} \\ \hline
             ID & 15 \\ \hline
             ID\_Categoria & 3 \\ \hline
             Nombre & Mancuernas 10 kg \\ \hline
             Descripcion & Mancuernas recubiertas de goma antideslizante \\ \hline
             Cantidad & 20 \\ \hline
             Instalacion & false \\ \hline
             espacio & 1 m\textsuperscript{2} \\ \hline
             Precio & 12.99 \\ \hline
             Imagen & imagen \\ \hline
        \end{tabular}
        \caption{Ejemplo de la entidad \textit{Producto}}
        \label{table:T-Producto}
    \end{table}

\end{itemize}

\subsection{Entidad Categorias}
\begin{itemize}
    \item \textbf{Descripción:} este tipo de entidad representa las categorías del catálogo de productos. Las categorías pueden estar organizadas jerárquicamente mediante el atributo padre\_id.

    \item \textbf{Restricciones:} entidad fuerte.

    \item \textbf{Características:}
    \begin{itemize}
        \item Nombre del tipo de entidad: Categorias.
        \item Tipo: Fuerte.
        \item Atributos heredados: Ninguno.
        \item Atributo identificador primario: ID.
        \item Atributo identificador alternativo: ninguno.
        \item Número de atributos: 3 propios.
    \end{itemize}

    \item Atributos propios:
    \begin{itemize}

        \item \textbf{ID}
        \begin{itemize}
            \item Definición: identificador único de la categoría.
            \item Dominio: entero positivo.
            \item Tipo: clave primaria.
            \item Opcional: no.
            \item Ejemplo: 3.
        \end{itemize}

        \item \textbf{Nombre}
        \begin{itemize}
            \item Definición: nombre de la categoría.
            \item Dominio: cadena de 150 caracteres.
            \item Tipo: atributo simple.
            \item Opcional: no.
            \item Ejemplo: Pesas libres.
        \end{itemize}

        \item \textbf{padre\_id}
        \begin{itemize}
            \item Definición: referencia a la categoría padre (si existe jerarquía).
            \item Dominio: entero positivo o nulo.
            \item Tipo: clave foránea opcional.
            \item Opcional: sí.
            \item Ejemplo: null.
        \end{itemize}

    \end{itemize}

    \item \textbf{Diagrama (Figura \ref{fig:E-Categorias}):}

    \begin{figure}[H]
        \centering
        \includegraphics[scale=0.3]{img/diagramas/categoria.png}
        \caption{Entidad Categorias}
        \label{fig:E-Categorias}
    \end{figure}

    \item \textbf{Ejemplo de entidad (Tabla \ref{table:T-Categorias}):}

    \begin{table}[H]
    \centering
        \begin{tabular}{ |p{6cm}||p{6cm}| }
             \hline
             \multicolumn{2}{|c|}{\textbf{Categorias}} \\ \hline
             \textbf{Atributo} & \textbf{Valor} \\ \hline
             ID & 3 \\ \hline
             Nombre & Pesas libres \\ \hline
             padre\_id & null \\ \hline
        \end{tabular}
        \caption{Ejemplo de la entidad \textit{Categorias}}
        \label{table:T-Categorias}
    \end{table}

\end{itemize}



\subsection{Entidad Contrato\_productos}
\begin{itemize}
    \item \textbf{Descripción:} esta entidad representa la relación entre un contrato y los productos asociados al mismo. Permite registrar qué productos forman parte de cada contrato y su coste mensual individual.

    \item \textbf{Restricciones:} entidad fuerte (aunque conceptualmente actúa como una entidad intermedia de relación N:N).

    \item \textbf{Características:}
    \begin{itemize}
        \item Nombre del tipo de entidad: Contrato\_productos.
        \item Tipo: Fuerte.
        \item Atributos heredados: Ninguno.
        \item Atributo identificador primario: ID.
        \item Atributo identificador alternativo: ninguno.
        \item Número de atributos: 4 propios.
    \end{itemize}

    \item Atributos propios:
    \begin{itemize}

        \item \textbf{ID}
        \begin{itemize}
            \item Definición: identificador único del registro de producto asociado a contrato.
            \item Dominio: entero positivo.
            \item Tipo: clave primaria.
            \item Opcional: no.
            \item Ejemplo: 40.
        \end{itemize}

        \item \textbf{ID\_producto}
        \begin{itemize}
            \item Definición: identificador del producto asociado al contrato.
            \item Dominio: entero positivo.
            \item Tipo: clave foránea.
            \item Opcional: no.
            \item Ejemplo: 15.
        \end{itemize}

        \item \textbf{ID\_contrato}
        \begin{itemize}
            \item Definición: identificador del contrato.
            \item Dominio: entero positivo.
            \item Tipo: clave foránea.
            \item Opcional: no.
            \item Ejemplo: 10.
        \end{itemize}

        \item \textbf{mensualidad}
        \begin{itemize}
            \item Definición: coste mensual del producto dentro del contrato.
            \item Dominio: número real positivo.
            \item Tipo: atributo simple.
            \item Opcional: no.
            \item Ejemplo: 12.99.
        \end{itemize}

    \end{itemize}

    \item \textbf{Diagrama (Figura \ref{fig:E-ContratoProductos}):}

    \begin{figure}[H]
        \centering
        \includegraphics[scale=0.3]{img/diagramas/contrato_productos.png}
        \caption{Entidad Contrato\_productos}
        \label{fig:E-ContratoProductos}
    \end{figure}

    \item \textbf{Ejemplo de entidad (Tabla \ref{table:T-ContratoProductos}):}

    \begin{table}[H]
    \centering
        \begin{tabular}{ |p{6cm}||p{6cm}| }
             \hline
             \multicolumn{2}{|c|}{\textbf{Contrato\_productos}} \\ \hline
             \textbf{Atributo} & \textbf{Valor} \\ \hline
             ID & 40 \\ \hline
             ID\_producto & 15 \\ \hline
             ID\_contrato & 10 \\ \hline
             mensualidad & 12.99 \\ \hline
        \end{tabular}
        \caption{Ejemplo de la entidad \textit{Contrato\_productos}}
        \label{table:T-ContratoProductos}
    \end{table}

\end{itemize}

\section{Tipos de interrelación} \label{sec:relaciones}
En esta sección se identificarán y describirán las interrelaciones entre los tipos de entidad descritos en la sección 7.2.

Las interrelaciones pueden ser de tipo débil o fuerte.

\begin{itemize}
    \item Un tipo de Interrelación Fuerte es aquella que representa la relación existente entre dos tipos de entidad fuertes.

    \item Un tipo de Interrelación Débil es aquella que representa la relación entre un tipo de entidad débil y otro fuerte o entre dos tipos de entidad débiles.
\end{itemize}

De acuerdo con la notación del modelo E-R, una interrelación se representa mediante un rombo del que parten flechas hacia los tipos de entidad que forman parte de la relación. Cada tipo de entidad interviene en la interrelación con una determinada cardinalidad, que indica el número mínimo y máximo de instancias de cada tipo de entidad que pueden participar en la interrelación. Se representa por dos valores entre paréntesis (mínimo y máximo). Las posibles cardinalidades son: (0,1), (1,1),(0,n),(1,n),(m,n). Estas cardinalidades determinan el tipo de interrelación que se está definiendo, que puede ser:

\begin{itemize}
    \item \textbf{1:1} Uno a uno. Cuando los dos tipos de entidad participan con una cardinalidad máxima de 1.
    \item \textbf{1:N o N:1} Uno a muchos, o muchos a uno. Cuando uno de los tipos de entidad participa con una cardinalidad máxima de 1 y el otro con una cardinalidad máxima de n.
    \item \textbf{N:N} Muchos a muchos. Cuando ambos tipos de entidad participan una cardinalidad máxima de n.
\end{itemize}

Para cada una de las interrelaciones que se han identificado en la definición del problema, se indicará la siguiente información:

\begin{itemize}
    \item \textbf{Nombre}. Nombre del tipo de interrelación.
    \item \textbf{Descripción}. Definición del tipo de interrelación y de los tipos de entidad que participan en ella.
    \item \textbf{Tipo}. Indicación de si se trata de un tipo de interrelación débil o fuerte identificando los tipos de debilidad en su caso.
    \item \textbf{Cardinalidad}. Indicación de la cardinalidad del tipo de interrelación y cardinalidades mínima y máxima de los tipos de entidad intervinientes.
    \item \textbf{Atributos}. Indicación del número y descripción de los atributos del tipo de interrelación, en su caso.
    \item \textbf{Diagrama}. Representación gráfica del tipo de interrelación, de acuerdo con la notación E-R.
    \item \textbf{Ejemplo}. Valores de muestra.
\end{itemize}

Se han identificado las interrelaciones que se indican a continuación y que se describirán en las siguientes subsecciones:

\begin{itemize}
    \item Tipo de interrelación: Usuario - Rol
    \item Tipo de interrelación: Usuario - Dirección
    \item Tipo de interrelación: Dirección - Envio\_Servicio
    \item Tipo de interrelación: Usuario - Contrato
    \item Tipo de interrelación: Dirección - Contrato
    \item Tipo de interrelación: Contrato - Subscricion
    \item Tipo de interrelación: Subscricion - Tipo
    \item Tipo de interrelación: Contrato - Contrato\_productos
    \item Tipo de interrelación: Contrato\_productos - Producto
    \item Tipo de interrelación: Categorias - Producto
\end{itemize}
\subsection{Interrelación Usuario - Rol}

\begin{itemize}
    \item \textbf{Nombre}. Usuario - Rol.

    \item \textbf{Descripción}. Relación que vincula a cada usuario con el rol que tiene asignado en el sistema. Un usuario desempeña exactamente un rol, mientras que un rol puede estar asociado a varios usuarios.

    \item \textbf{Tipo}. Interrelación fuerte entre dos entidades fuertes (Usuario y Rol).

    \item \textbf{Cardinalidad}.  
    Un rol puede estar asociado a entre 0 y N usuarios \((0,N)\).  
    Cada usuario pertenece exactamente a un único rol \((1,1)\).  
    Por tanto, se trata de una interrelación de tipo \(N:1\) (muchos usuarios a un rol).

    \item \textbf{Atributos}. La interrelación no tiene atributos propios; la relación se identifica mediante las claves ajenas en la entidad Usuario (id\_rol).

    \item \textbf{Diagrama}. Se representa mediante un rombo que conecta las entidades Usuario y Rol.

    \item \textbf{Ejemplo}.  
    Un usuario con id = 5, nombre = “Antonio”, pertenece al rol con id = 1, nombre = “Administrador”.
\end{itemize}


\subsection{Interrelación Usuario - Dirección}

\begin{itemize}
    \item \textbf{Nombre}. Usuario - Dirección.

    \item \textbf{Descripción}. Relación que indica qué direcciones pertenecen a cada usuario. Un usuario puede registrar varias direcciones, mientras que cada dirección está asociada a un único usuario.

    \item \textbf{Tipo}. Interrelación fuerte entre las entidades Usuario y Dirección.

    \item \textbf{Cardinalidad}.  
    Un usuario puede tener entre 0 y N direcciones \((0,N)\).  
    Cada dirección pertenece exactamente a un usuario \((1,1)\).  
    Se trata de una relación de tipo \(1:N\) (un usuario, muchas direcciones).

    \item \textbf{Atributos}. La interrelación no tiene atributos propios; se materializa mediante el atributo id\_usuario en la entidad Dirección.

    \item \textbf{Diagrama}. Representada mediante un rombo que conecta Usuario y Dirección.

    \item \textbf{Ejemplo}.  
    El usuario con id = 5 tiene asociadas las direcciones con id = 3 (Calle Mayor 15) y id = 4 (Avenida del Deporte 2).
\end{itemize}

\subsection{Interrelación Contrato - Subscricion}

\begin{itemize}
    \item \textbf{Nombre}. Contrato - Subscricion.

    \item \textbf{Descripción}. Relación que vincula los contratos con las suscripciones que incluyen. Un contrato puede estar compuesto por una o varias suscripciones.

    \item \textbf{Tipo}. Interrelación fuerte entre Contrato y Subscricion.

    \item \textbf{Cardinalidad}.  
    Un contrato puede tener entre 0 y N suscripciones \((0,N)\).  
    Cada suscripción pertenece exactamente a un contrato \((1,1)\).  
    Es una interrelación \(1:N\).

    \item \textbf{Atributos}. Sin atributos propios; se refleja mediante ID\_contrato en Subscricion.

    \item \textbf{Diagrama}. Rombo entre Contrato y Subscricion.

    \item \textbf{Ejemplo}.  
    El contrato con id = 10 incluye la suscripción con id = 7.
\end{itemize}


\subsection{Interrelación Subscricion - Tipo}

\begin{itemize}
    \item \textbf{Nombre}. Subscricion - Tipo.

    \item \textbf{Descripción}. Relación que indica de qué tipo es cada suscripción (por ejemplo, Básica, Premium, etc.).

    \item \textbf{Tipo}. Interrelación fuerte entre Subscricion y Tipo.

    \item \textbf{Cardinalidad}.  
    Un tipo de suscripción puede estar asociado a entre 0 y N subscripciones \((0,N)\).  
    Cada subscripción pertenece exactamente a un tipo \((1,1)\).  
    Es una relación \(N:1\).

    \item \textbf{Atributos}. Sin atributos propios; se implementa mediante ID\_tipo en Subscricion.

    \item \textbf{Diagrama}. Rombo entre Subscricion y Tipo.

    \item \textbf{Ejemplo}.  
    La subscripción con id = 7 pertenece al tipo con ID = 2 (Premium).
\end{itemize}


\subsection{Interrelación Contrato - Contrato\_productos}

\begin{itemize}
    \item \textbf{Nombre}. Contrato - Contrato\_productos.

    \item \textbf{Descripción}. Relación que asocia un contrato con los distintos productos incluidos en él. Cada contrato puede tener varios registros en Contrato\_productos.

    \item \textbf{Tipo}. Interrelación fuerte entre Contrato y Contrato\_productos.

    \item \textbf{Cardinalidad}.  
    Un contrato puede tener entre 0 y N filas en Contrato\_productos \((0,N)\).  
    Cada registro de Contrato\_productos pertenece a un único contrato \((1,1)\).  
    Es una interrelación \(1:N\).

    \item \textbf{Atributos}. Los atributos propios de la relación (como mensualidad) se almacenan en la entidad Contrato\_productos.

    \item \textbf{Diagrama}. Rombo entre Contrato y Contrato\_productos.

    \item \textbf{Ejemplo}.  
    El contrato con id = 10 tiene los registros Contrato\_productos con ID = 40 y 41.
\end{itemize}


\subsection{Interrelación Contrato\_productos - Producto}

\begin{itemize}
    \item \textbf{Nombre}. Contrato\_productos - Producto.

    \item \textbf{Descripción}. Relación que indica qué producto concreto está asociado a cada registro de Contrato\_productos.

    \item \textbf{Tipo}. Interrelación fuerte entre Contrato\_productos y Producto.

    \item \textbf{Cardinalidad}.  
    Un producto puede formar parte de entre 0 y N registros en Contrato\_productos \((0,N)\).  
    Cada registro de Contrato\_productos hace referencia a un único producto \((1,1)\).  
    Es una interrelación \(N:1\).

    \item \textbf{Atributos}. Sin atributos adicionales; la mensualidad está en Contrato\_productos.

    \item \textbf{Diagrama}. Rombo entre Contrato\_productos y Producto.

    \item \textbf{Ejemplo}.  
    El registro Contrato\_productos con ID = 40 hace referencia al producto con ID = 15.
\end{itemize}


\subsection{Interrelación Categorias - Producto}

\begin{itemize}
    \item \textbf{Nombre}. Categorias - Producto.

    \item \textbf{Descripción}. Relación que establece la categoría a la que pertenece cada producto del catálogo.

    \item \textbf{Tipo}. Interrelación fuerte entre Categorias y Producto.

    \item \textbf{Cardinalidad}.  
    Una categoría puede tener entre 0 y N productos \((0,N)\).  
    Cada producto pertenece exactamente a una categoría \((1,1)\).  
    Es una interrelación \(1:N\).

    \item \textbf{Atributos}. No tiene atributos propios; se modela mediante ID\_Categoria en Producto.

    \item \textbf{Diagrama}. Rombo entre Categorias y Producto.

    \item \textbf{Ejemplo}.  
    La categoría con ID = 3 (Pesas libres) contiene el producto con ID = 15 (Mancuernas 10 kg).
\end{itemize}

\begin{landscape}
\section{Diagrama del Modelo Entidad-Interrelación}\label{sec:diagrama-E-R}
El diagrama completo se muestra en la figura \ref{fig:EER_v5}
\begin{figure}[H]
    \centering
    \includegraphics[scale=0.35]{img/diagramas/diagrama_Proyecto.jpg}
    \caption{Diagrama del modelo Entidad - Interrelación}
    \label{fig:EER_v5}
\end{figure}
\end{landscape}
