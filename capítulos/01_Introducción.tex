\chapter{Introducción}

\section{Contexto del problema a resolver}

    La gestión de turnos y guardias en instituciones sanitarias es una tarea crítica que requiere precisión,
    organización y una correcta asignación de recursos humanos.
    En hospitales de tamaño medio, como el hospital de Pozoblanco, 
    esta gestión influye directamente tanto en el correcto funcionamiento del servicio como en la satisfacción laboral 
    del personal sanitario.

    Actualmente, muchas de estas tareas siguen realizándose mediante herramientas genéricas como hojas de cálculo,
    lo que dificulta la automatización de procesos, la trazabilidad de la información y la reducción de errores humanos. 
    
    La ausencia de un sistema específico adaptado a las necesidades reales del hospital provoca ineficiencias que pueden 
    derivar en conflictos organizativos y errores en la planificación.

\section{Definición del problema real}
    
    Existe un problema con respecto a la gestión de las guardias del hospital de Pozoblanco.

    En resumen, el problema real que se desea resolver con el desarrollo de este proyecto es el siguiente:

    Actualmente, el sistema de guardias del hospital de Pozoblanco se gestiona mediante hojas de cálculo en formato Excel, 
    las cuales son administradas manualmente por los empleados responsables. Estos empleados deciden, 
    utilizando distintos términos y patrones no estandarizados, qué trabajadores se encargan de las guardias de cada día, 
    plasmando finalmente la información en un documento PDF.

    Este proceso resulta tedioso y propenso a errores, ya que gran parte de la gestión de guardias, 
    horas trabajadas y días libres (entre otros aspectos) depende del factor humano. 
    Esto puede provocar inconsistencias en los datos, errores de cálculo y dificultades a la hora de realizar modificaciones 
    o auditorías posteriores.
\section{Definición del problema técnico}

    El problema técnico consiste en la ausencia de una herramienta software específica que permita
    gestionar de forma automatizada, segura y centralizada las guardias del hospital de Pozoblanco.

    Desde el punto de vista técnico, se requiere el desarrollo de una aplicación web capaz de
    gestionar usuarios con distintos niveles de acceso, almacenar información estructurada sobre
    guardias y profesionales, y aplicar reglas de negocio para la planificación y validación de los
    turnos.

    Para abordar este problema, se utilizará una arquitectura basada en una aplicación web con un
    backend encargado de la lógica de negocio y un frontend responsable de la interacción con el
    usuario, comunicados mediante una API REST.

\subsection{Funcionamiento}

    Se desea implementar una aplicación web que permita realizar una gestión automatizada de las
    guardias del hospital, reduciendo la dependencia del factor humano y minimizando los errores
    derivados de la gestión manual.

    Los usuarios que interactuarán con la aplicación son los siguientes:

    \begin{itemize}
    \item \textbf{Administrador}
        \begin{itemize}
        \item Usuario con acceso completo a las funcionalidades del sistema.
        \item Responsable de la gestión de usuarios, guardias, jefaturas y validaciones.
        \end{itemize}

    \item \textbf{Usuario de consulta}
        \begin{itemize}
            \item Usuario con acceso a funcionalidades de consulta de guardias.
            \item Sin permisos para modificar información crítica del sistema.
        \end{itemize}
    \end{itemize}
\subsection{Entorno}
Existirán dos entornos de trabajo para esta aplicación:

\begin{itemize}
\item \textbf{Entorno de desarrollo y testeo}: entorno utilizado para desarrollar, probar y depurar la aplicación web antes de su despliegue.
\item \textbf{Entorno de producción}: entorno en el que se encontrará la versión final de la aplicación web accesible para los usuarios finales.
\end{itemize}

Asimismo, existirán dos entornos de ejecución:

\begin{itemize}
\item \textbf{Entorno de ejecución del administrador}: acceso completo a las funcionalidades del sistema.
\item \textbf{Entorno de ejecución del resto de usuarios}: acceso limitado a funcionalidades de consulta.
\end{itemize}

La interfaz será el medio de comunicación entre los usuarios y la aplicación web. Es fundamental que sea intuitiva y amigable para garantizar una buena experiencia de usuario. Los requisitos de la interfaz se describirán en la Sección \ref{sec:requisitos-interfaz}.  
\subsection{Vida esperada}

    El ciclo de vida esperado de la aplicación web será elevado, ya que se ha diseñado siguiendo una
    arquitectura modular y escalable que permite su adaptación a futuras necesidades del hospital,
    como la incorporación de nuevas funcionalidades o la ampliación de los tipos de usuario.

\subsection{Ciclo de mantenimiento} \label{subsec:ciclo_mantenimiento}

Dado que el desarrollo de la aplicación web se enmarca dentro de un proyecto académico, el
mantenimiento posterior de la aplicación no correrá a cargo de su autor.

No obstante, se ha diseñado una estructura modular que facilite posibles tareas de mantenimiento,
corrección de errores o ampliaciones futuras por parte de terceros.

\subsection{Competencia}

    En la actualidad existen soluciones comerciales para la gestión de turnos y guardias, así como
    herramientas genéricas utilizadas para este fin. No obstante, muchas de estas soluciones no se
    adaptan completamente a las necesidades específicas del hospital de Pozoblanco.

    Este proyecto se plantea como una solución personalizada, orientada a cubrir de forma concreta
    los requisitos del contexto hospitalario descrito.

\subsection{Aspecto externo}

En relación con el aspecto externo, se tendrán en cuenta los siguientes aspectos:

\begin{itemize}
\item \textbf{Interfaz de usuario}
\begin{itemize}
\item Se desarrollará una interfaz totalmente responsiva, intuitiva y amigable. El Capítulo \ref{cap:diseño_interfaz} describirá el diseño de la interfaz de la aplicación web.
\end{itemize}
\item \textbf{Distribución de la aplicación}
\begin{itemize}
\item La aplicación se podrá distribuir mediante un servidor web accesible a través de un navegador estándar.
\end{itemize}
\end{itemize}


\subsection{Estandarización}

    Para el desarrollo de la aplicación web, se revisarán las recomendaciones propuestas por \textit{World Wide Web Consortium} (W3C) \cite{w3c}, que ``promueve el uso de estándares para reducir el coste y la complejidad del desarrollo, así como para incrementar la accesibilidad y usabilidad de cualquier documento publicado en la web''.

    También se debe tener en cuenta que los recursos  que se van a utilizar son herramientas informáticas que están validadas por prestigiosas organizaciones que indican que cumplen con los estándares. Véase el capítulo \ref{cap:recursos} de Recursos. 

\subsection{Calidad y fiabilidad}

La calidad y fiabilidad de la aplicación web estarán garantizadas por:

\begin{itemize}
    \item Uso de un framework consolidado como Laravel y React.
    \item Aplicación de buenas prácticas de programación.
    \item Realización de pruebas funcionales y de seguridad.
\end{itemize}

\subsection{Programa de tareas}

 El desarrollo del presente proyecto va estar compuesto por la siguientes fases:
 \begin{itemize}
     \item Introducción: descripción del problema, establecimiento de los objetivos, revisión de antecedentes, identificación de restricciones iniciales y estratégicas y selección de recursos.
     \item Análisis: especificación de requisitos (funcionales, no funcionales, de información y de la interfaz), descripción del modelo de datos y análisis funcional (casos de uso y diagramas de secuencia).
    
     \item Diseño: descripción del diseño de datos, clases, paquetes y de la interfaz.
    
     \item Implementación: codificación de la aplicación web teniendo en cuenta el diseño desarrollado.
    
     \item Pruebas: comprobación de que la aplicación web funciona correctamente, es robusta y amigable.
 \end{itemize}


\subsection{Pruebas}

La fase de pruebas es esencial para garantizar que la aplicación web funciona correctamente. En particular, se pretende comprobar que la aplicación web:
\begin{itemize}
    \item Hace lo que debe hacer.
    \item No provoca efectos secundarios que pueden desencadenar situaciones catastróficas.
    \item Contiene módulos que se ejecutan correctamente.
    \item Garantiza los privilegios de cada tipo de usuario.
\end{itemize}

Cada prueba tendrá la siguiente estructura para detectar los errores y corregirlos:
\begin{itemize}
    \item Objetivo de la prueba. Se debe indicar en qué consiste la prueba y el resultado esperado.
    \item Problema detectado, en su caso. Si ocurre un error entonces se debe describir la causa que lo ha provocado.
    \item Solución adoptada, en su caso. Si se ha producido un error, se deben indicar las medidas tomadas para solucionarlo.
\end{itemize}

El Capítulo \ref{cap:pruebas} de Pruebas describirá las pruebas realizadas.

\subsection{Seguridad}

    Durante el desarrollo de la aplicación web se tendrán en cuenta aspectos fundamentales de
    seguridad, entre los que se incluyen:

    \begin{itemize}
    \item Autenticación y autorización de usuarios.
    \item Protección contra ataques de inyección SQL.
    \item Gestión segura de sesiones.
    \item Protección de datos sensibles.
    \end{itemize}

Estos aspectos se describirán con mayor detalle en los capítulos correspondientes.
