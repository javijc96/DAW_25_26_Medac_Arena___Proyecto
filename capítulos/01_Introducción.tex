\chapter{Introducción}

\section{Contexto del problema a resolver}

    
\section{Definición del problema real}
    
    Existe un problema generalizado con respecto a

    En resumen, el problema real que se desea resolver con el desarrollo de este proyecto es el siguiente: 
    
\section{Definición del problema técnico}

    Se va a utilizar la metodología denominada 

\subsection{Funcionamiento}
    Se desea implementar una aplicación web que permita realizar una gestión de... 

    Los usuarios que interactuarán con la aplicación son los siguientes:
    \begin{itemize}
        \item Administrador
          \begin{itemize}
              \item Este tipo de usuario estará registrado en el sistema ...
              \item En particular, tendrá las competencias exclusivas de...
          \end{itemize}
         
     \end{itemize}
     
\subsection{Entorno}
 Existirán dos entornos de trabajo para esta aplicación. 

  \begin{itemize}
     \item Entorno de desarrollo y testeo: entorno que se utilizará para desarrollar, probar y depurar la aplicación web antes de su lanzamiento al entorno de producción.
     \item Entorno de producción: entorno en el que se encontrará la versión final de la aplicación web...
 \end{itemize}

Existirán dos entornos de ejecución para esta aplicación. 

 \begin{itemize}
     \item Entorno de ejecución del administrador: el administrador ejecutará la aplicación...
     \item Entorno de ejecución del resto de usuarios: ...
 \end{itemize}

 La interfaz es el medio de comunicación entre los usuarios y la aplicación web. Es importante que la interfaz sea lo más intuitiva y amigable posible para poder generar una mejor experiencia en el usuario. Los requisitos de la interfaz se describirán en la Sección \ref{sec:requisitos-interfaz}.

        
\subsection{Vida esperada}
    
    El ciclo de vida esperado para dicha aplicación web será alto...
    
\subsection{Ciclo de mantenimiento} \label{subsec:ciclo_mantenimiento}

 Teniendo en cuenta que el propósito de la creación de esta aplicación web es la realización de un proyecto, el mantenimiento de dicha aplicación no correrá a cargo de su autor.

 No obstante, se realizará un diseño modular de cada una de las partes de la aplicación web para facilitar, en el futuro, posibles tareas de mantenimiento y mejoras.


\subsection{Competencia}




\subsection{Aspecto externo}

En relación con el aspecto externo, se tendrán en cuenta los siguientes aspectos:

\begin{itemize}
    \item \textbf{Interfaz de usuario}
    \begin{itemize}
        \item Se realizará una interfaz totalmente responsiva, intuitiva y amigable para que el usuario pueda navegar de la manera más cómoda posible. El Capítulo \ref{cap:diseño_interfaz} describirá el Diseño de la Interfaz de la aplicación web que se va  a desarrollar.
    \end{itemize}
    \item \textbf{Distribución de la aplicación: formato de almacenamiento}
     \begin{itemize}
         \item La aplicación se podrá distribuir a través...
     \end{itemize}
\end{itemize}


\subsection{Estandarización}

 Para el desarrollo de la aplicación web, se revisarán las recomendaciones propuestas por \textit{World Wide Web Consortium} (W3C) \cite{w3c}, que ``promueve el uso de estándares para reducir el coste y la complejidad del desarrollo, así como para incrementar la accesibilidad y usabilidad de cualquier documento publicado en la web''.

 También se debe tener en cuenta que los recursos  que se van a utilizar son herramientas informáticas que están validadas por prestigiosas organizaciones que indican que cumplen con los estándares. Véase el capítulo \ref{cap:recursos} de Recursos. 

\subsection{Calidad y fiabilidad}

 La calidad y la fiabilidad de la aplicación web estará garantizadas por las siguientes razones.
 \begin{itemize}
     \item 
 \end{itemize}

\subsection{Programa de tareas}

 El desarrollo del presente proyecto va estar compuesto por la siguientes fases:
 \begin{itemize}
     \item Introducción: descripción del problema, establecimiento de los objetivos, revisión de antecedentes, identificación de restricciones iniciales y estratégicas y selección de recursos.
     \item Análisis: especificación de requisitos (funcionales, no funcionales, de información y de la interfaz), descripción del modelo de datos y análisis funcional (casos de uso y diagramas de secuencia).
    
     \item Diseño: descripción del diseño de datos, clases, paquetes y de la interfaz.
    
     \item Implementación: codificación de la aplicación web teniendo en cuenta el diseño desarrollado.
    
     \item Pruebas: comprobación de que la aplicación web funciona correctamente, es robusta y amigable.
 \end{itemize}


\subsection{Pruebas}

La fase de pruebas es esencial para garantizar que la aplicación web funciona correctamente. En particular, se pretende comprobar que la aplicación web:
\begin{itemize}
    \item Hace lo que debe hacer.
    \item No provoca efectos secundarios que pueden desencadenar situaciones catastróficas.
    \item Contiene módulos que se ejecutan correctamente.
    \item Garantiza los privilegios de cada tipo de usuario.
\end{itemize}

Cada prueba tendrá la siguiente estructura para detectar los errores y corregirlos:
\begin{itemize}
    \item Objetivo de la prueba. Se debe indicar en qué consiste la prueba y el resultado esperado.
    \item Problema detectado, en su caso. Si ocurre un error entonces se debe describir la causa que lo ha provocado.
    \item Solución adoptada, en su caso. Si se ha producido un error, se deben indicar las medidas tomadas para solucionarlo.
\end{itemize}

El Capítulo \ref{cap:pruebas} de Pruebas describirá las pruebas realizadas.

\subsection{Seguridad}

Hay varias consideraciones de seguridad que se deben tener en cuenta al desarrollar una aplicación web moderna. Algunas de las más importantes son:

\begin{itemize}
    \item Autenticación y autorización: 

    \item Protección contra ataques de inyección...
\end{itemize}


