\chapter{README (BORRAR EN EL main.tex)}

\section{Ejemplos uso Overleaf comunes}

% ---------------------------------------------------------------------------------

Ejemplo cita de la bibliografía \textbf{MEDAC}\cite{medac} es un centro de Formación Profesional...

\cite{arquitecturaMariadb}
\cite{administracionMysql}
\cite{mongodbEnterprise}

% ---------------------------------------------------------------------------------

\begin{itemize}
    \item Áreas de sistemas y departamentos de informática en cualquier sector de actividad.
    \item Sector de servicios tecnológicos y comunicaciones.
    \item Área comercial con gestión de transacciones por Internet.
\end{itemize}

% ---------------------------------------------------------------------------------

\begin{figure}[htp]
    \centering
    \includegraphics[scale=0.9]{img/diagramas/Modelo-Físico-ER.png}
    \caption{Ejemplo de adjuntar diagrama ER}
    \label{fig:ModeloFísicoER}
\end{figure}

% ---------------------------------------------------------------------------------

\textit{Texto escrito en cursiva}
\textup{Texto escrito en letras rectas}
\textsl{Texto roman de estilo inclinado}
\textsc{Texto escrito en mayúsculas pequeñas}


\section{Justificación}

El presente documento describe la guía de trabajo a seguir, para la correcta elaboración del documento sobre el que se soporta el desarrollo del módulo Proyecto, del segundo curso del C.F.G.S de Desarrollo de Aplicaciones Web.

Este módulo profesional complementa la formación de otros módulos profesionales en las funciones de análisis del contexto, diseño y organización de la intervención y planificación de la evaluación de la misma.


Así mismo, teniendo en cuenta que las actividades profesionales asociadas a estas funciones se aplican en:

\begin{itemize}
    \item Áreas de sistemas y departamentos de informática en cualquier sector de actividad.
    \item Sector de servicios tecnológicos y comunicaciones.
    \item Área comercial con gestión de transacciones por Internet.
\end{itemize}
    
Y, por otra parte, las líneas de actuación en el proceso de enseñanza aprendizaje que permiten alcanzar los objetivos del módulo versarán sobre:
\begin{itemize}
    \item La ejecución de trabajos en equipo.
    \item La autoevaluación del trabajo realizado.
    \item La autonomía y la iniciativa.
    \item El uso de las TIC.
\end{itemize}

Por estos motivos se propone un Proyecto Integrado orientado al desarrollo de forma autónoma de nuevos productos y servicios innovadores, así como al autoempleo, con el propósito de que el alumno aplique de forma eficaz y realista, los conocimientos y destrezas adquiridos durante el ciclo. 
Este proyecto deberá seguir el índice de contenidos que se expone en el punto 3.

\section{Formato}
Los alumnos deberán hacer la entrega del proyecto siguiendo la siguiente normativa de manera obligatoria:

\begin{itemize}
    \item Incluir Índice y Portada (disponible en la Plataforma Virtual).  
    \item La Portada debe contener el nombre del proyecto (de forma clara, precisa y representativa del trabajo), nombre del alumno, curso académico, nombre del módulo y nombre del Centro y su logotipo.
\end{itemize}

El índice ha de ser automático, es decir, que ayude a localizar los distintos apartados del trabajo. 

\begin{itemize}
    \item Tras el índice deberá aparecer una página de cortesía.
    \item Enumerar páginas a partir del primer apartado, es decir excluir índice y página de cortesía.
    \item Justificar párrafos.
    \item Enumerar los distintos apartados.
    \item Corregir ortografía y gramática (no cometer faltas de ortografía, cuidar la expresión, etc.).
    \item Referenciar según normas APA.
    \item Utilizar la plantilla Word de Medac (disponible en la Plataforma).
    \item Tipo de letra: Times New Roman 12
    \item Interlineado 1,5.
    \item La extensión del proyecto debe ser de mínimo 18 páginas y máximo 30 páginas (Sin incluir portada, índice y anexos).
\end{itemize}

\section{Orientaciones sobre evaluación}

A parte de los criterios de evaluación dispuestos en la Real Decreto de la titulación, el tutor del proyecto evaluará el trabajo en base a los siguientes criterios de evaluación referentes a la realización del mismo, así como a la actitud del alumno ante su elaboración y trabajo en equipo:

\begin{itemize}
    \item Originalidad de la idea.
    \item Inclusión de los contenidos reflejados en el índice.
    \item Cumplimiento de las normas de formato de entrega, descritas en esta programación.
    \item Coherencia y argumentación del contenido.
    \item Que el proyecto evidencie que el alumno ha adquirido conocimientos propios de su titulación.
    \item Puntualidad y asistencia tanto a las clases teóricas de Proyecto como a las tutorías propuestas por el tutor.
    \item Nivel de responsabilidad observado.
    \item Grado de iniciativa mostrado por el alumno.
    \item Predisposición para el trabajo en equipo.
    \item Compañerismo y respeto con todos los agentes implicados en la elaboración del proyecto: compañeros y tutores.

\end{itemize}

Por último, se detallan a continuación los criterios de evaluación sobre los que se apoyará el tribunal ante la defensa del proyecto:

\begin{itemize}
    \item Originalidad en la presentación.
    \item Correcta expresión oral y corporal.
    \item Dominio de todo el contenido del proyecto.
    \item Indumentaria formal y acorde para la ocasión.
    \item Ajuste al tiempo indicado para la defensa.
    \item Trato respetuoso hacia todos los presentes en el aula durante la defensa.

\end{itemize}

Recordaros que aquellos trabajos en los que se detecte plagio, no serán evaluados, debiendo el alumnado entregar ese trabajo de nuevo en el periodo de recuperaciones correspondiente. La diferencia entre algo que está literalmente copiado y algo que ha sido citado o en lo que el alumnado se ha basado para elaborar su trabajo es clara. MEDAC persigue una evaluación justa y limpia persiguiendo la correcta competencia entre el alumnado y como tal evaluará.
