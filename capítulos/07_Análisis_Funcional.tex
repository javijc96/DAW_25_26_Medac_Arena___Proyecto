\chapter{Análisis funcional} \label{cap:analisis}

\section{Introducción}

Este capítulo identificará a los tipos de actores que podrán interactuar con el sistema informático que se desea desarrollar. A continuación, se utilizarán los casos de uso y los diagramas de secuencia para describir las acciones que podrán realizar los diferentes actores.

\section{Actores}

Un actor es una representación de una persona, proceso o entidad externa que interactúa con el sistema. Se van a considerar los siguientes tipos de actores:
   \begin{itemize}
        \item \textbf{Administrador}
        \begin{itemize}
            \item Este tipo de usuario estará registrado en el sistema y tendrá un control completo de la aplicación.
            \item Tendrá competencias exclusivas para la gestión del catálogo de productos, categorías, planes, contratos, usuarios y órdenes de logística.
        \end{itemize}

        \item \textbf{Cliente}
        \begin{itemize}
            \item Usuario registrado que utiliza la plataforma para alquilar material de gimnasio.
            \item Podrá gestionar su cuenta, consultar el catálogo de productos, añadir productos al carrito, contratar planes, gestionar sus direcciones, contratos, suscripciones y pagos.
        \end{itemize}

        \item \textbf{Usuario público}
        \begin{itemize}
            \item Usuario no autenticado que accede a la página web sin necesidad de registro.
            \item Podrá consultar la página inicial y una vista general del catálogo antes de registrarse o iniciar sesión.
        \end{itemize}

        \item \textbf{Pasarela de pago}
        \begin{itemize}
            \item Sistema externo encargado de autorizar y capturar los pagos realizados por el cliente.
            \item Permite confirmar o rechazar las operaciones de pago asociadas a los contratos y suscripciones.
        \end{itemize}

     \end{itemize}


\section{Casos de uso}

  Los casos de uso describen las acciones que pueden desarrollar los actores del sistema. Se ha identificado los siguientes casos de uso principales, que son descritos en las secciones que se indican:
\begin{itemize}
    \item \textbf{CU-0. Diagrama de contexto} (Sección \ref{sec:CU-0}).  
    \item \textbf{CU-1. Gestionar cuenta de usuario} (Sección \ref{sec:CU-1}). 
    \item \textbf{CU-2. Consultar catálogo de productos} (Sección \ref{sec:CU-2}).
    \item \textbf{CU-3. Gestionar carrito y contratos} (Sección \ref{sec:CU-3}).
    \item \textbf{CU-4. Gestionar suscripción} (Sección \ref{sec:CU-4}).
    \item \textbf{CU-5. Gestionar direcciones y envíos} (Sección \ref{sec:CU-5}).
    \item \textbf{CU-6. Gestionar pagos y facturas} (Sección \ref{sec:CU-6}).
    \item \textbf{CU-7. Consultar panel del cliente} (Sección \ref{sec:CU-7}).
    \item \textbf{CU-8. Administrar la plataforma} (Sección \ref{sec:CU-8}).
\end{itemize}

\subsection{Caso de Uso 0. Diagrama de Contexto}\label{sec:CU-0}

  El diagrama de contexto engloba los casos de uso principales que componen el sistema. Véanse la Figura \ref{fig:Diagrama-Contexto} y la Tabla \ref{tab:CU-0}.

%Diagrama de contexto
\begin{figure}[H]
        \centering
        %\includegraphics[scale=0.6]{img/diagramas/Funcional/CU-0.png}
        \caption{Diagrama de Contexto}\label{fig:Diagrama-Contexto}
\end{figure}


\begin{table}[H]
\caption{CU-0. Diagrama de contexto}\label{tab:CU-0}
\begin{center}
    \begin{tabular}{|l|p{12cm}|}
    \hline
    \multicolumn{2}{|c|}{Caso de uso 0 - Diagrama de contexto} \\ 
    \hline \hline
    Actores                 &   Usuario público, Cliente, Administrador, Pasarela de pago \\ \hline
    Descripción             &   Visión general de las acciones principales del sistema de alquiler de material de gimnasio. \\  \hline
    Precondiciones          &   El usuario debe acceder a la página inicial de la aplicación web mediante un navegador compatible.   \\  \hline
    Casos de uso            &   CU-1. Gestionar cuenta de usuario. \\  
    &   CU-2. Consultar catálogo de productos. \\
    &   CU-3. Gestionar carrito y contratos. \\
    &   CU-4. Gestionar suscripción. \\
    &   CU-5. Gestionar direcciones y envíos. \\
    &   CU-6. Gestionar pagos y facturas. \\
    &   CU-7. Consultar panel del cliente. \\
    &   CU-8. Administrar la plataforma. \\ \hline
    Flujo principal     &  1a. El usuario público accede a la aplicación sin necesidad de iniciar sesión y puede consultar la página inicial y el catálogo básico de productos. \\
        & 1b. El cliente o el administrador se identifican en el sistema para acceder a las funcionalidades asociadas a su rol (gestión de cuenta, alquiler de productos, administración del catálogo, etc.).
    \\ \hline
    Flujo alternativo    & 1. La aplicación no se carga correctamente o se produce un error en el servidor. \\
      & 2. Se informa del error al usuario mediante un mensaje adecuado. \\
      & 3. Se indica al usuario que puede intentar cargar de nuevo la aplicación o volver a intentarlo más tarde.
    \\  \hline
    \end{tabular}
    \end{center}
    \end{table}

\newpage

\subsection{CU-1. Gestionar cuenta de usuario}\label{sec:CU-1}

El caso de uso \textit{CU-1.  Gestionar cuenta de usuario} describe las acciones que puede realizar  (véanse la Tabla \ref{tab:CU-1} y la Figura \ref{fig:Diagrama-Caso de uso 1.}.). Este caso de uso está compuesto por los siguientes sub-casos de uso que se describen en las tablas que se indican:
\begin{itemize}
    \item CU-1.1. Registrarse en la plataforma.
    \item CU-1.2. Iniciar sesión.
    \item CU-1.3. Cerrar sesión.
    \item CU-1.4. Recuperar contraseña.
    \item CU-1.5. Editar perfil de usuario.
\end{itemize}
\begin{figure}[H]
\centering
%\includegraphics[scale=0.75]{img/diagramas/Funcional/CU-1.png}
\caption{CU-1.}\label{fig:Diagrama-Caso de uso 1.}
\end{figure}


\begin{table}[H]
\caption{CU-1. Gestionar cuenta de usuario}\label{tab:CU-1}
\begin{center}
    \begin{tabular}{|l|p{12cm}|}
    \hline
    \multicolumn{2}{|c|}{Caso de uso 1 - Gestionar cuenta de usuario} \\ 
    \hline \hline
    Actores                 &   Usuario público, Cliente, Administrador \\ \hline
    Descripción             &   Permite registrar una nueva cuenta, iniciar y cerrar sesión, recuperar la contraseña y editar los datos básicos del perfil de usuario. \\  \hline
    Precondiciones          &   El usuario debe haber accedido a la página principal de la aplicación. Para editar el perfil o cerrar sesión, el usuario debe estar autenticado.   \\  \hline
    Casos de uso            &   CU-1.1. Registrarse en la plataforma.  \\  
                            &   CU-1.2. Iniciar sesión.  \\
                            &   CU-1.3. Cerrar sesión.  \\
                            &   CU-1.4. Recuperar contraseña.  \\
                            &   CU-1.5. Editar perfil de usuario.  \\ \hline

    Flujo principal     &   1. El usuario selecciona una opción relacionada con la gestión de su cuenta (registro, inicio de sesión, recuperación de contraseña o edición de perfil). \\ 
                        &   2. El sistema muestra el formulario correspondiente. \\
                        &   3. El usuario introduce la información requerida (datos personales, credenciales, correo de recuperación, etc.). \\
                        &   4. El sistema valida la información introducida. \\
                        &   5. Si la información es correcta, el sistema realiza la acción solicitada (creación de cuenta, inicio/cierre de sesión, actualización de datos o envío de correo de recuperación). \\ \hline

    Flujo alternativo    &  1. Se introduce información incompleta o no válida en el formulario. \\ 
                            &  2. El sistema muestra mensajes de error indicando los campos que deben corregirse.  \\  
                            &  3. El usuario modifica los datos y vuelve a enviar el formulario.  \\  
                            &  4. Si el problema persiste (por ejemplo, correo ya registrado o credenciales incorrectas), se informa al usuario y se le permite volver a intentarlo o cancelar la operación. \\  \hline    
            \end{tabular}
        \end{center}
    \end{table}

\newpage

\subsection{CU-2. Consultar catálogo de productos}\label{sec:CU-2}

El caso de uso \textit{CU-2. Consultar catálogo de productos} describe las acciones que puede realizar un usuario (público, cliente o administrador) para visualizar el catálogo, aplicar filtros y consultar la ficha detallada de un producto. Véanse la Tabla \ref{tab:CU-2} y la Figura \ref{fig:Diagrama-CU2}. Este caso de uso está compuesto por los siguientes sub-casos de uso:
\begin{itemize}
    \item CU-2.1. Listar productos.
    \item CU-2.2. Buscar productos mediante filtros.
    \item CU-2.3. Consultar ficha de producto.
\end{itemize}

\begin{figure}[H]
\centering
%\includegraphics[scale=0.75]{img/diagramas/Funcional/CU-2.png}
\caption{CU-2. Consultar catálogo de productos}\label{fig:Diagrama-CU2}   
\end{figure}

\begin{table}[H]
\caption{CU-2. Consultar catálogo de productos}\label{tab:CU-2}
\begin{center}
    \begin{tabular}{|l|p{12cm}|}
    \hline
    \multicolumn{2}{|c|}{Caso de uso 2 - Consultar catálogo de productos} \\ 
    \hline \hline
    Actores                 &   Usuario público, Cliente, Administrador \\ \hline

    Descripción             &   Permite visualizar el catálogo completo de productos disponibles, aplicar filtros de búsqueda (categoría, precio, espacio en m², instalación), y acceder a la ficha de un producto. \\  \hline

    Precondiciones          &   El usuario debe haber accedido a la página principal de la aplicación o a la sección del catálogo. \\  \hline

    Casos de uso            &   CU-2.1. Listar productos.  \\  
    &   CU-2.2. Buscar productos mediante filtros.  \\ 
    &   CU-2.3. Consultar ficha de producto. \\ \hline

    Flujo principal     &   1. El usuario accede al catálogo de productos. \\ 
    &   2. El sistema muestra la lista de productos disponibles. \\ 
    &   3. El usuario puede aplicar filtros (categoría, precio, m², instalación). \\ 
    &   4. El sistema actualiza la lista en función de los filtros. \\ 
    &   5. El usuario selecciona un producto para ver su ficha detallada. \\ 
    &   6. El sistema muestra la información completa del producto (fotos, descripción, m², tarifas, disponibilidad). \\ \hline
    Flujo alternativo    &  1. No existen productos que cumplan los filtros seleccionados.  \\ 
    &  2. El sistema muestra un mensaje indicando que no hay resultados. \\ 
    &  3. Permite modificar los filtros o limpiar la búsqueda. \\ \hline    
            \end{tabular}
        \end{center}
    \end{table}

\subsection{CU-3. Gestionar carrito y contratos}\label{sec:CU-3}

El caso de uso \textit{CU-3. Gestionar carrito y contratos} describe las acciones que puede realizar un cliente para añadir productos al carrito, configurar la duración del alquiler, calcular el precio final y generar un contrato. Véanse la Tabla \ref{tab:CU-3} y la Figura \ref{fig:Diagrama-CU3}. Este caso de uso está compuesto por los siguientes sub-casos de uso:
\begin{itemize}
    \item CU-3.1. Añadir producto al carrito.
    \item CU-3.2. Modificar o eliminar productos del carrito.
    \item CU-3.3. Calcular precio del contrato.
    \item CU-3.4. Generar contrato.
\end{itemize}

\begin{figure}[H]
\centering
%\includegraphics[scale=0.75]{img/diagramas/Funcional/CU-3.png}
\caption{CU-3. Gestionar carrito y contratos}\label{fig:Diagrama-CU3}   
\end{figure}

\begin{table}[H]
\caption{CU-3. Gestionar carrito y contratos}\label{tab:CU-3}
\begin{center}
    \begin{tabular}{|l|p{12cm}|}
    \hline
    \multicolumn{2}{|c|}{Caso de uso 3 - Gestionar carrito y contratos} \\ 
    \hline \hline

    Actores                 &   Cliente \\ \hline

    Descripción             &   Permite añadir productos al carrito, configurar la duración del alquiler, calcular el precio final y generar un contrato con dirección y fechas seleccionadas. \\  \hline

    Precondiciones          &   El usuario debe estar autenticado para generar un contrato; para usar el carrito puede estar o no registrado. \\  \hline

    Casos de uso            &   CU-3.1. Añadir producto al carrito.  \\  
                            &   CU-3.2. Modificar o eliminar productos del carrito.  \\ 
                            &   CU-3.3. Calcular precio del contrato. \\ 
                            &   CU-3.4. Generar contrato. \\ \hline

    Flujo principal     &   1. El usuario selecciona un producto del catálogo. \\ 
                        &   2. Añade el producto al carrito indicando su duración (días/semanas/meses). \\ 
                        &   3. El sistema actualiza el carrito y calcula el precio del alquiler. \\ 
                        &   4. El usuario selecciona la dirección de entrega y recogida. \\ 
                        &   5. El sistema genera un resumen del contrato. \\ 
                        &   6. El usuario confirma la operación y se genera el contrato pendiente de pago. \\ \hline

    Flujo alternativo    &  1. Un producto ya no está disponible.  \\ 
                            &  2. El sistema avisa al usuario y lo elimina del carrito. \\ 
                            &  3. El usuario puede modificar la duración o seleccionar otro producto.  \\ \hline    
            \end{tabular}
        \end{center}
    \end{table}
\newpage

\subsection{CU-4. Gestionar suscripción}\label{sec:CU-4}

El caso de uso \textit{CU-4. Gestionar suscripción} describe las acciones que realiza el cliente para contratar, gestionar y cancelar una suscripción mensual a un plan de servicio. Véanse la Tabla \ref{tab:CU-4} y la Figura \ref{fig:Diagrama-CU4}. Este caso de uso está compuesto por los siguientes sub-casos de uso:
\begin{itemize}
    \item CU-4.1. Contratar suscripción.
    \item CU-4.2. Consultar estado de la suscripción.
    \item CU-4.3. Pausar o cancelar la suscripción.
\end{itemize}

\begin{figure}[H]
\centering
%\includegraphics[scale=0.75]{img/diagramas/Funcional/CU-4.png}
\caption{CU-4. Gestionar suscripción}\label{fig:Diagrama-CU4}   
\end{figure}

\begin{table}[H]
\caption{CU-4. Gestionar suscripción}\label{tab:CU-4}
\begin{center}
    \begin{tabular}{|l|p{12cm}|}
    \hline
    \multicolumn{2}{|c|}{Caso de uso 4 - Gestionar suscripción} \\ 
    \hline \hline

    Actores                 &   Cliente \\ \hline

    Descripción             &   Permite contratar un plan de suscripción mensual, consultar su estado actual (activa, pausada, cancelada) y gestionarla, incluyendo pausarla temporalmente o cancelarla de forma permanente. \\  \hline

    Precondiciones          &   El usuario debe estar autenticado. Debe existir al menos un plan de suscripción disponible. \\  \hline

    Casos de uso            &   CU-4.1. Contratar suscripción.  \\  
                            &   CU-4.2. Consultar estado de la suscripción.  \\ 
                            &   CU-4.3. Pausar o cancelar la suscripción. \\ \hline

    Flujo principal     &   1. El usuario accede a la sección de planes de suscripción. \\ 
                        &   2. Selecciona uno de los planes disponibles. \\ 
                        &   3. El sistema muestra la información completa del plan (precio, características, período de facturación). \\ 
                        &   4. El usuario confirma la contratación del plan. \\ 
                        &   5. El sistema activa la suscripción y registra la fecha de inicio y el método de pago. \\ \hline

    Flujo alternativo    &  1. El usuario intenta contratar un plan pero su método de pago falla o no está disponible.  \\ 
                            &  2. El sistema muestra un mensaje de error indicando la incidencia. \\ 
                            &  3. El usuario puede actualizar su método de pago o intentar de nuevo la contratación. \\ \hline    
            \end{tabular}
        \end{center}
    \end{table}

    \subsection{CU-5. Gestionar direcciones y envíos}\label{sec:CU-5}

El caso de uso \textit{CU-5. Gestionar direcciones y envíos} describe las acciones que puede realizar el cliente para gestionar sus direcciones de entrega/recogida y las órdenes de logística asociadas a los contratos. Véanse la Tabla \ref{tab:CU-5} y la Figura \ref{fig:Diagrama-CU5}. Este caso de uso está compuesto por los siguientes sub-casos de uso:
\begin{itemize}
    \item CU-5.1. Gestionar direcciones del cliente (crear, modificar, eliminar).
    \item CU-5.2. Seleccionar dirección de entrega y recogida en un contrato.
    \item CU-5.3. Gestionar órdenes de logística (consulta de estado).
\end{itemize}

\begin{figure}[H]
\centering
%\includegraphics[scale=0.75]{img/diagramas/Funcional/CU-5.png}
\caption{CU-5. Gestionar direcciones y envíos}\label{fig:Diagrama-CU5}   
\end{figure}

\begin{table}[H]
\caption{CU-5. Gestionar direcciones y envíos}\label{tab:CU-5}
\begin{center}
    \begin{tabular}{|l|p{12cm}|}
    \hline
    \multicolumn{2}{|c|}{Caso de uso 5 - Gestionar direcciones y envíos} \\ 
    \hline \hline

    Actores                 &   Cliente, Administrador \\ \hline

    Descripción             &   Permite al cliente gestionar sus direcciones de entrega y recogida, seleccionar una dirección durante la creación de un contrato y consultar las órdenes de logística generadas para los envíos y recogidas. \\  \hline

    Precondiciones          &   El usuario debe estar autenticado para gestionar sus direcciones. Debe existir al menos una dirección para poder asociarla a un contrato. \\  \hline

    Casos de uso            &   CU-5.1. Gestionar direcciones del cliente. \\  
                            &   CU-5.2. Seleccionar dirección de entrega y recogida. \\ 
                            &   CU-5.3. Gestionar órdenes de logística. \\ \hline

    Flujo principal         &   1. El usuario accede a la sección de direcciones en su cuenta. \\ 
                            &   2. El sistema muestra el listado de direcciones registradas. \\ 
                            &   3. El usuario puede crear una nueva dirección o modificar/eliminar una existente. \\ 
                            &   4. Durante la creación de un contrato, el usuario selecciona la dirección de entrega y, en su caso, de recogida. \\ 
                            &   5. El sistema registra la dirección en el contrato y genera la orden de logística correspondiente. \\ \hline

    Flujo alternativo       &   1. El usuario introduce datos de dirección incompletos o no válidos. \\ 
                            &   2. El sistema muestra mensajes de error indicando los campos a corregir. \\ 
                            &   3. Si no existen direcciones válidas, el sistema impide continuar con la creación del contrato hasta que se registre al menos una dirección correcta. \\ \hline    
            \end{tabular}
        \end{center}
    \end{table}

    \subsection{CU-6. Gestionar pagos y facturas}\label{sec:CU-6}

El caso de uso \textit{CU-6. Gestionar pagos y facturas} describe las acciones que realiza el cliente para efectuar el pago de contratos y suscripciones, así como la generación y consulta de las facturas asociadas. Véanse la Tabla \ref{tab:CU-6} y la Figura \ref{fig:Diagrama-CU6}. Este caso de uso está compuesto por los siguientes sub-casos de uso:
\begin{itemize}
    \item CU-6.1. Realizar pago de contrato o suscripción.
    \item CU-6.2. Validar el pago con la pasarela.
    \item CU-6.3. Generar y consultar factura.
\end{itemize}

\begin{figure}[H]
\centering
%\includegraphics[scale=0.75]{img/diagramas/Funcional/CU-6.png}
\caption{CU-6. Gestionar pagos y facturas}\label{fig:Diagrama-CU6}   
\end{figure}

\begin{table}[H]
\caption{CU-6. Gestionar pagos y facturas}\label{tab:CU-6}
\begin{center}
    \begin{tabular}{|l|p{12cm}|}
    \hline
    \multicolumn{2}{|c|}{Caso de uso 6 - Gestionar pagos y facturas} \\ 
    \hline \hline

    Actores                 &   Cliente, Pasarela de pago, Administrador \\ \hline

    Descripción             &   Permite realizar el pago de contratos y suscripciones a través de una pasarela de pago, validar la transacción y generar la factura asociada, que quedará disponible para su consulta y descarga. \\  \hline

    Precondiciones          &   Debe existir un contrato o una suscripción pendiente de pago. La pasarela de pago debe estar disponible. El usuario debe estar autenticado. \\  \hline

    Casos de uso            &   CU-6.1. Realizar pago de contrato o suscripción. \\  
                            &   CU-6.2. Validar el pago con la pasarela. \\ 
                            &   CU-6.3. Generar y consultar factura. \\ \hline

    Flujo principal         &   1. El usuario accede al resumen del contrato o suscripción pendiente de pago. \\ 
                            &   2. Selecciona el método de pago y confirma la operación. \\ 
                            &   3. El sistema envía la información a la pasarela de pago. \\ 
                            &   4. La pasarela de pago procesa la transacción y devuelve el resultado (éxito o fallo). \\ 
                            &   5. Si el pago es aceptado, el sistema marca el contrato o la suscripción como activa y genera la factura en formato PDF. \\ 
                            &   6. El usuario puede consultar y descargar la factura desde su panel. \\ \hline

    Flujo alternativo       &   1. La pasarela de pago rechaza la operación o se produce un error en la transacción. \\ 
                            &   2. El sistema informa al usuario del fallo en el pago. \\ 
                            &   3. El contrato o la suscripción permanecen en estado pendiente y no se activan. \\ 
                            &   4. El usuario puede intentar de nuevo el pago o cancelar la operación. \\ \hline    
            \end{tabular}
        \end{center}
    \end{table}

    \subsection{CU-7. Consultar panel del cliente}\label{sec:CU-7}

El caso de uso \textit{CU-7. Consultar panel del cliente} describe las acciones que realiza el cliente para visualizar un resumen de sus contratos, suscripciones, direcciones y facturas dentro de la plataforma. Véanse la Tabla \ref{tab:CU-7} y la Figura \ref{fig:Diagrama-CU7}. Este caso de uso está compuesto por los siguientes sub-casos de uso:
\begin{itemize}
    \item CU-7.1. Consultar contratos.
    \item CU-7.2. Consultar suscripciones.
    \item CU-7.3. Consultar y descargar facturas.
\end{itemize}

\begin{figure}[H]
\centering
%\includegraphics[scale=0.75]{img/diagramas/Funcional/CU-7.png}
\caption{CU-7. Consultar panel del cliente}\label{fig:Diagrama-CU7}   
\end{figure}

\begin{table}[H]
\caption{CU-7. Consultar panel del cliente}\label{tab:CU-7}
\begin{center}
    \begin{tabular}{|l|p{12cm}|}
    \hline
    \multicolumn{2}{|c|}{Caso de uso 7 - Consultar panel del cliente} \\ 
    \hline \hline

    Actores                 &   Cliente \\ \hline

    Descripción             &   Permite al cliente consultar de forma centralizada la información relacionada con sus contratos, suscripciones, direcciones y facturas, así como acceder al detalle de cada elemento. \\  \hline

    Precondiciones          &   El usuario debe estar autenticado en la plataforma. \\  \hline

    Casos de uso            &   CU-7.1. Consultar contratos. \\  
                            &   CU-7.2. Consultar suscripciones. \\ 
                            &   CU-7.3. Consultar y descargar facturas. \\ \hline

    Flujo principal         &   1. El usuario accede al panel de cliente desde el menú principal. \\ 
                            &   2. El sistema muestra un resumen de sus contratos, suscripciones y facturas. \\ 
                            &   3. El usuario selecciona el apartado que desea consultar (contratos, suscripciones o facturas). \\ 
                            &   4. El sistema muestra el listado detallado del apartado seleccionado. \\ 
                            &   5. El usuario puede acceder al detalle de un elemento concreto o descargar la factura correspondiente, en su caso. \\ \hline

    Flujo alternativo       &   1. El usuario no dispone de contratos, suscripciones o facturas en alguno de los apartados. \\ 
                            &   2. El sistema muestra un mensaje indicando que no existen registros disponibles en esa sección. \\ 
                            &   3. Se ofrecen enlaces o acciones para guiar al usuario (por ejemplo, acceder al catálogo o contratar un plan). \\ \hline    
            \end{tabular}
        \end{center}
    \end{table}

    \subsection{CU-8. Administrar la plataforma}\label{sec:CU-8}

El caso de uso \textit{CU-8. Administrar la plataforma} describe las acciones que realiza el administrador para gestionar los elementos principales del sistema: productos, categorías, planes, usuarios, contratos y órdenes de logística. Véanse la Tabla \ref{tab:CU-8} y la Figura \ref{fig:Diagrama-CU8}. Este caso de uso está compuesto por los siguientes sub-casos de uso:
\begin{itemize}
    \item CU-8.1. Gestionar productos.
    \item CU-8.2. Gestionar categorías y planes.
    \item CU-8.3. Gestionar usuarios y roles.
    \item CU-8.4. Gestionar contratos y órdenes de logística.
\end{itemize}

\begin{figure}[H]
\centering
%\includegraphics[scale=0.75]{img/diagramas/Funcional/CU-8.png}
\caption{CU-8. Administrar la plataforma}\label{fig:Diagrama-CU8}   
\end{figure}

\begin{table}[H]
\caption{CU-8. Administrar la plataforma}\label{tab:CU-8}
\begin{center}
    \begin{tabular}{|l|p{12cm}|}
    \hline
    \multicolumn{2}{|c|}{Caso de uso 8 - Administrar la plataforma} \\ 
    \hline \hline

    Actores                 &   Administrador \\ \hline

    Descripción             &   Permite al administrador gestionar el catálogo de productos y categorías, los planes de suscripción, los usuarios y sus roles, así como revisar y mantener los contratos y las órdenes de logística. \\  \hline

    Precondiciones          &   El usuario debe estar autenticado y disponer de permisos de administrador. \\  \hline

    Casos de uso            &   CU-8.1. Gestionar productos. \\  
                            &   CU-8.2. Gestionar categorías y planes. \\ 
                            &   CU-8.3. Gestionar usuarios y roles. \\
                            &   CU-8.4. Gestionar contratos y órdenes de logística. \\ \hline

    Flujo principal         &   1. El administrador accede al panel de administración de la plataforma. \\ 
                            &   2. El sistema muestra las opciones de gestión disponibles (productos, categorías, planes, usuarios, contratos, logística). \\ 
                            &   3. El administrador selecciona el módulo que desea gestionar. \\ 
                            &   4. El sistema muestra el listado de registros asociados (por ejemplo, productos o usuarios). \\ 
                            &   5. El administrador puede crear, modificar o eliminar registros, según las necesidades de la plataforma. \\ 
                            &   6. El sistema valida los datos introducidos y actualiza la información en la base de datos. \\ \hline

    Flujo alternativo       &   1. El administrador introduce datos incompletos o no válidos al crear o modificar un registro. \\ 
                            &   2. El sistema muestra mensajes de error indicando los campos que deben corregirse. \\ 
                            &   3. Si la eliminación de un registro afecta a otros elementos (por ejemplo, productos asociados a contratos), el sistema puede impedir la operación o solicitar confirmación adicional. \\ \hline    
            \end{tabular}
        \end{center}
    \end{table}


\section{Validación de casos de uso}
La tabla \ref{tab:ValidacionCU} permite comprobar que los casos de uso cubren todos los requisitos funcionales de la aplicación web propuestos en la Sección \ref{sec:requisitos-funcionales}.

\begin{table}[H]
    \centering
    \caption{Tabla de validación casos de uso} \label{tab:ValidacionCU}
    \begin{tabular}{|l|l|}
        \hline
            \textbf{Requisito funcional} & \textbf{Caso de uso} \\ 
        \hline 
        \hline         
            \textbf{RF-01}  & CU-1 \\ % Gestionar cuenta de usuario
        \hline
            \textbf{RF-02}  & CU-1 \\ % Gestionar cuenta de usuario
        \hline
            \textbf{RF-03}  & CU-1, CU-8 \\ % Roles y administración
        \hline
            \textbf{RF-04}  & CU-2 \\ % Consultar catálogo
        \hline
            \textbf{RF-05}  & CU-2 \\ % Ficha de producto
        \hline
            \textbf{RF-06}  & CU-3 \\ % Carrito
        \hline
            \textbf{RF-07}  & CU-3 \\ % Cálculo de precio
        \hline
            \textbf{RF-08}  & CU-3 \\ % Checkout / contrato
        \hline
            \textbf{RF-09}  & CU-4 \\ % Gestionar suscripción
        \hline
            \textbf{RF-10}  & CU-4 \\ % Validación de suscripción
        \hline
            \textbf{RF-11}  & CU-4 \\ % Alta/pausa/cancelación
        \hline
            \textbf{RF-12}  & CU-5 \\ % CRUD direcciones
        \hline
            \textbf{RF-13}  & CU-3, CU-5 \\ % Seleccionar dirección en contrato
        \hline
            \textbf{RF-14}  & CU-5 \\ % Orden logística
        \hline
            \textbf{RF-15}  & CU-6 \\ % Pasarela de pago
        \hline
            \textbf{RF-16}  & CU-3, CU-6 \\ % Contrato solo si pago OK
        \hline
            \textbf{RF-17}  & CU-6 \\ % Factura PDF
        \hline
            \textbf{RF-18}  & CU-7 \\ % Panel cliente
        \hline
            \textbf{RF-19}  & CU-8 \\ % Backoffice admin
        \hline
    \end{tabular}
\end{table}


\newpage 

\section{Diagrama de secuencia}

El diagrama de secuencia es una representación gráfica que pretende dar una visión de las acciones que se realizarán durante la ejecución de alguna operación en el sistema. A continuación, se muestran los diagramas de secuencias para la creación (Figura  \ref{fig:Diagrama de secuencia crear}), búsqueda (Figura \ref{fig:Diagrama de secuencia buscar}), modificación (Figura \ref{fig:Diagrama de secuencia modificar}) y borrado (Figura \ref{fig:Diagrama de secuencia borrar}) de instancias genéricas () que se corresponderán con...

\begin{figure}[H]
    \centering
%        \includegraphics[scale=0.55]{img/diagramas/Secuencia/SEC-1.png}
    \caption{Diagrama de secuencia de crear}
    \label{fig:Diagrama de secuencia crear}
\end{figure}

\begin{figure}[H]
    \centering
%        \includegraphics[scale=0.55]{img/diagramas/Secuencia/SEC-1.png}
    \caption{Diagrama de secuencia de borrar}
    \label{fig:Diagrama de secuencia borrar}
\end{figure}

\begin{figure}[H]
    \centering
%        \includegraphics[scale=0.55]{img/diagramas/Secuencia/SEC-1.png}
    \caption{Diagrama de secuencia de buscar}
    \label{fig:Diagrama de secuencia buscar}
\end{figure}

\begin{figure}[H]
    \centering
%        \includegraphics[scale=0.55]{img/diagramas/Secuencia/SEC-1.png}
    \caption{Diagrama de secuencia de modificar}
    \label{fig:Diagrama de secuencia modificar}
\end{figure}