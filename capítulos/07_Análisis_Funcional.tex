\chapter{Análisis funcional} \label{cap:analisis}

\section{Introducción}

Este capítulo identificará a los tipos de actores que podrán interactuar con el sistema informático que se desea desarrollar. A continuación, se utilizarán los casos de uso y los diagramas de secuencia para describir las acciones que podrán realizar los diferentes actores.

\section{Actores}

Un actor es una representación de una persona, proceso o entidad externa que interactúa con el sistema. Se van a considerar los siguientes tipos de actores:
    \begin{itemize}
        \item Administrador
          \begin{itemize}
              \item Este tipo de usuario estará registrado en el sistema y tendrá un control completo de la aplicación. 
              \item En particular, tendrá las competencias exclusivas de la gestión de...
          \end{itemize}
          
     \end{itemize}

\section{Casos de uso}

  Los casos de uso describen las acciones que pueden desarrollar los actores del sistema. Se ha identificado los siguientes casos de uso principales, que son descritos en las secciones que se indican:
    \begin{itemize}
    \item \textbf{CU-0. Diagrama de contexto} (Sección \ref{sec:CU-0}).  
    \item \textbf{CU-1. Administrar...} (Sección \ref{sec:CU-1}). 
\end{itemize}

\subsection{Caso de Uso 0. Diagrama de Contexto}\label{sec:CU-0}         

  El diagrama de contexto engloba los casos de uso principales que componen el sistemas. Véanse la Figura \ref{fig:Diagrama-Contexto} y la Tabla \ref{tab:CU-0}.

%Diagrama de contexto
\begin{figure}[H]
        \centering
        %\includegraphics[scale=0.6]{img/diagramas/Funcional/CU-0.png}
        \caption{Diagrama de Contexto}\label{fig:Diagrama-Contexto}
\end{figure}

\begin{table}[H]
\caption{CU-0. Diagrama de contexto}\label{tab:CU-0}
\begin{center}
    \begin{tabular}{|l|p{12cm}|}
    \hline
    \multicolumn{2}{|c|}{Caso de uso 0 - Diagrama de contexto} \\ 
    \hline \hline
    Actores                 &   Administrador \\ \hline
    Descripción             &   Acciones principales del sistema\\  \hline
    Precondiciones          &   El usuario debe acceder a la página inicial de la aplicación   \\  \hline
    Casos de uso            &   CU-1. \\  
    & CU-2. Administrar \\ 
    \hline
    Flujo principal     &  1a. El usuario público accede a la aplicación sin necesidad de iniciar sesión y podrá. \\
        & 1b. El resto de usuarios se deberá identificar para acceder a los módulos correspondientes dentro de la aplicación.
    \\ \hline
    Flujo alternativo    & 1. La aplicación no se carga correctamente. \\
      & 2. Se informa del error al usuario. \\
      & 3. Se informa al usuario que puede intentar cargar de nuevo la aplicación.
    \\  \hline
    \end{tabular}
    \end{center}
    \end{table}

\newpage

\subsection{CU-1. }\label{sec:CU-1}

El caso de uso \textit{CU-1. } describe las acciones que puede realizar  (véanse la Tabla \ref{tab:CU-1} y la Figura \ref{fig:Diagrama-Caso de uso 1.}.). Este caso de uso está compuesto por los siguientes sub-casos de uso que se describen en las tablas que se indican:
\begin{itemize}
    \item CU-1.1 
\end{itemize}

\begin{figure}[H]
\centering
%\includegraphics[scale=0.75]{img/diagramas/Funcional/CU-1.png}
\caption{CU-1.}\label{fig:Diagrama-Caso de uso 1.}   
\end{figure}

\begin{table}[H]
\caption{CU-1.}\label{tab:CU-1}
\begin{center}
    \begin{tabular}{|l|p{12cm}|}
    \hline
    \multicolumn{2}{|c|}{Caso de uso 1 - } \\ 
    \hline \hline
    Actores                 &   Usua          \\ \hline
    Descripción             &   Acciones que puede realizar el usuario  \\  \hline
    Precondiciones          &   El usuario debe haber accedido a la página principal de la aplicación y elegido el caso de uso CU-1..   \\  \hline
    Casos de uso            &   CU-1.1.  \\  
    & CU-1.2. Administrar  \\ 

    \hline
    Flujo principal     &   1. El usuario elige un \\ 
    & 2. El sistema muestra  \\ \hline
    Flujo alternativo    &  1. Se produce un error al intentar ejecutar la información elegida. \\ 
                            &  2. Se informa al usuario del error.  \\  
                            &  3. El sistema vuelve a mostrar las opciones disponibles.  \\  \hline    
            \end{tabular}
        \end{center}
    \end{table}

\newpage

\section{Validación de casos de uso}
La tabla \ref{tab:Validación CU} permite comprobar que los casos de uso cubren todos los requisitos funcionales de la aplicación web propuestos en la Sección \ref{sec:requisitos-funcionales}.
\begin{table}[H]
    \centering
    \caption{Tabla validación casos de uso} \label{tab:Validación CU}
    \begin{tabular}{|l|l|}
        \hline
            \textbf{Requisito funcional} & \textbf{Caso de uso} \\ 
        \hline 
        \hline         
            \textbf{RF-1}  & CU 1.1 \\
        \hline
            \textbf{RF-8}  & CU 2.1 \\
        \hline
    \end{tabular}%
    \end{table}
    
\newpage 

\section{Diagrama de secuencia}

El diagrama de secuencia es una representación gráfica que pretende dar una visión de las acciones que se realizarán durante la ejecución de alguna operación en el sistema. A continuación, se muestran los diagramas de secuencias para la creación (Figura  \ref{fig:Diagrama de secuencia crear}), búsqueda (Figura \ref{fig:Diagrama de secuencia buscar}), modificación (Figura \ref{fig:Diagrama de secuencia modificar}) y borrado (Figura \ref{fig:Diagrama de secuencia borrar}) de instancias genéricas () que se corresponderán con...

\begin{figure}[H]
    \centering
%        \includegraphics[scale=0.55]{img/diagramas/Secuencia/SEC-1.png}
    \caption{Diagrama de secuencia de crear}
    \label{fig:Diagrama de secuencia crear}
\end{figure}

\begin{figure}[H]
    \centering
%        \includegraphics[scale=0.55]{img/diagramas/Secuencia/SEC-1.png}
    \caption{Diagrama de secuencia de borrar}
    \label{fig:Diagrama de secuencia borrar}
\end{figure}

\begin{figure}[H]
    \centering
%        \includegraphics[scale=0.55]{img/diagramas/Secuencia/SEC-1.png}
    \caption{Diagrama de secuencia de buscar}
    \label{fig:Diagrama de secuencia buscar}
\end{figure}

\begin{figure}[H]
    \centering
%        \includegraphics[scale=0.55]{img/diagramas/Secuencia/SEC-1.png}
    \caption{Diagrama de secuencia de modificar}
    \label{fig:Diagrama de secuencia modificar}
\end{figure}