\chapter{Especificación de requisitos}\label{cap:especificacion_requisitos}

\section{Introducción}


Se desea desarrollar una aplicación web que permita la gestión automatizada y centralizada de las guardias del hospital 
de Pozoblanco, sustituyendo el proceso actual basado en hojas de cálculo y generación manual de documentos. 

Las siguientes secciones describen los actores del sistema (Sección \ref{sec:actores}), la descripción 
modular (Sección \ref{sec:modulos}) y los requisitos del sistema (Sección  \ref{sec:requisitos-del-sistema}), 
incluyendo requisitos funcionales, no funcionales, de interfaz y de información.

\section{Actores del sistema}\label{sec:actores}

Se van a considerar los siguientes tipos de usuario en la aplicación web:
\begin{itemize}
    \item \textbf{Administrador}: usuario registrado en el sistema con permisos completos para la gestión de datos,
    profesionales, especialidades y guardias. Es responsable de la generación automática de los cuadrantes mensuales,
    la asignación y validación de jefes de guardia, la aprobación de cambios de guardia, el cierre de meses y
    la administración de usuarios y auditoría.

    \item \textbf{Usuario de consulta}: usuario registrado con permisos limitados,
    orientado a la consulta de información, que puede visualizar guardias, vistas diaria y mensual por sección,
    y realizar solicitudes de cambio de guardia que deberán ser validadas.
\end{itemize}


\section{Módulos de la aplicación}\label{sec:modulos}

La aplicación web estará compuesta por módulos que corresponden a cada tipo de usuario.

\subsection{Módulo del usuario de consulta}

El usuario de consulta podrá:
\begin{itemize}
    \item Consultar guardias por mes, sección y profesional.
    \item Visualizar la vista diaria con guardias agrupadas por sección,
    incluyendo profesional, tipo de guardia y jefe de guardia global.
    \item Consultar sus guardias asignadas en un mes determinado.
    \item Solicitar cambios de guardia con otros profesionales.
    \item Consultar el estado de las solicitudes de cambio realizadas.
\end{itemize}


\subsection{Módulo del administrador}

El usuario Administrador tendrá control total de la aplicación y se encargará de:
\begin{itemize}
    \item Generación automática de los cuadrantes mensuales de guardias por sección.
    \item Gestión manual de guardias y jefes de guardia cuando sea necesario.
    \item Validación y aprobación de solicitudes de cambio de guardia.
    \item Gestión de profesionales, secciones y asociaciones entre ellos.
    \item Cierre y reapertura de meses de guardias con control de permisos.
    \item Gestión de usuarios, roles y auditoría de acciones.
\end{itemize}

\section{Requisitos del sistema}\label{sec:requisitos-del-sistema}

Los requisitos del sistema hacen referencia a todas las características relacionadas con la aplicación web. Se describirán los siguientes tipos de requisitos:
\begin{itemize}
    \item Requisitos funcionales: describen las tareas que la aplicación debe realizar para satisfacer las necesidades del problema (Sección \ref{sec:requisitos-funcionales}).
    \item Requisitos no funcionales: describen cómo se tiene que satisfacer las necesidades del problema  (Sección \ref{sec:requisitos-no-funcionales}).
    \item Requisitos de la interfaz: describen cómo debe ser la comunicación entre el usuario y la aplicación  (Sección \ref{sec:requisitos-interfaz}).
    \item Requisitos de la información: describen las características de los datos que se van a gestionar (Sección \ref{sec:requisitos-información}).
\end{itemize}


\subsection{Requisitos funcionales}\label{sec:requisitos-funcionales}
Los requisitos funcionales indican lo que el sistema debe hacer. Se denotarán como RF-<nº requisito> y se agrupan por funcionalidades del sistema.

\begin{itemize}
        \item \textbf{Generación automática del cuadrante de guardias}
    \begin{itemize}
        \item RF-00. Generación automática del cuadrante mensual de guardias por sección del hospital.
        \item RF-00.1. Garantizar que todas las guardias necesarias estén cubiertas para cada día del mes.
        \item RF-00.2. Asociar cada guardia generada a una sección y a un profesional.
        \item RF-00.3. Permitir la revisión y modificación manual del cuadrante generado por un administrador.
    \end{itemize}

    \item \textbf{Importación del cuadrante de guardias}
    \begin{itemize}
        \item RF-01. Subida de un fichero Excel correspondiente al cuadrante mensual.
        \item RF-02. Validación de la estructura del Excel (columnas, formatos, especialidad, profesional, tipo CA/PF/LOC).
        \item RF-03. Importación de guardias a la base de datos asociando mes, especialidad, profesional y tipo.
        \item RF-04. Detección y evitación de guardias duplicadas.
        \item RF-05. Informe de importación con registros correctos/erróneos y motivo.
        \item RF-06. Repetición de importación de un mes solo con confirmación de un usuario autorizado.
    \end{itemize}

    \item \textbf{Gestión de datos maestros}
    \begin{itemize}
        \item RF-07. Gestión de especialidades (alta, baja y modificación).
        \item RF-08. Importación de Excel con listado oficial de facultativos y fecha de inicio en el centro.
        \item RF-09. Asociación de profesionales a una especialidad, si procede.
    \end{itemize}

    \item \textbf{Gestión de tipos de guardia}
    \begin{itemize}
        \item RF-10. Tipos predefinidos: CA (15:00--20:00), PF (24h), LOC (24h).
        \item RF-11. Toda guardia debe asociarse obligatoriamente a uno de estos tipos.
        \item RF-12. Horas asociadas por tipo: CA=5, PF=24, LOC=24.
    \end{itemize}

    \item \textbf{Gestión del mes de guardias}
    \begin{itemize}
        \item RF-13. Consulta de meses de guardias importados.
        \item RF-14. Cierre de mes para evitar modificaciones tras validación.
        \item RF-15. Reapertura de mes cerrado solo por usuarios autorizados.
    \end{itemize}

    \item \textbf{Asignación de Jefe de Guardia Global}
    \begin{itemize}
        \item RF-16. Un único jefe de guardia por cada día del mes.
        \item RF-17. Candidatos: profesionales con guardia ese día (cualquier tipo/especialidad).
        \item RF-18. Asignación automática al profesional con mayor antigüedad.
        \item RF-19. Ningún profesional debe tener más de 3 jefaturas en el mismo mes.
        \item RF-20. Si el de mayor antigüedad supera el límite, asignar al siguiente que cumpla.
        \item RF-21. Si nadie cumple, marcar el día como “pendiente de revisión”.
        \item RF-22. Asignación manual en días pendientes, registrando motivo y usuario.
        \item RF-23. Mostrar el jefe de guardia en las vistas diaria y mensual.
    \end{itemize}

    \item \textbf{Consultas y visualización de guardias}
    \begin{itemize}
        \item RF-24. Consulta por mes y especialidad.
        \item RF-25. Consulta por profesional en un mes determinado.
        \item RF-26. Vista diaria con guardias agrupadas por especialidad, profesional, tipo y jefe global.
        \item RF-27. Búsqueda de profesionales por nombre o identificador.
    \end{itemize}

    \item \textbf{Cálculo de horas de guardia}
    \begin{itemize}
        \item RF-28. Cálculo automático de horas totales por profesional en un mes según tipo.
        \item RF-29. Mostrar desglose de horas por día y por tipo (CA, PF, LOC).
        \item RF-30. Exportación del resumen de horas por profesional y por especialidad.
    \end{itemize}

    \item \textbf{Cálculo de importe estimado por guardias}
    \begin{itemize}
        \item RF-31. Configuración de tarifas por tipo de guardia (CA, PF, LOC).
        \item RF-32. Cálculo del importe estimado mensual por profesional (horas, tipo y tarifas).
        \item RF-33. Contemplar plus de jefatura configurable sumado al total cuando corresponda.
        \item RF-34. Mostrar desglose del importe (horas, tarifas y pluses).
        \item RF-35. Generar informe mensual de importes por profesional y por especialidad.
    \end{itemize}

    \item \textbf{Generación de documentos y envío por correo}
    \begin{itemize}
        \item RF-36. Generar PDF diario con fecha, jefe global y guardias agrupadas por especialidad (profesional y tipo).
        \item RF-37. Permitir descarga del PDF diario.
        \item RF-38. Envío automático por email del PDF diario a destinatarios configurados.
        \item RF-39. Generar Excel mensual con total de horas por profesional.
        \item RF-40. El Excel incluirá profesional, especialidad, total de horas y desglose por tipo.
        \item RF-41. Permitir descarga del Excel mensual.
    \end{itemize}

    \item \textbf{Usuarios, roles y auditoría}
    \begin{itemize}
        \item RF-42. Autenticación de usuarios.
        \item RF-43. Gestión de roles diferenciando al menos Administrador y Usuario de consulta.
        \item RF-44. Registro de historial de acciones (importaciones, cambios manuales, cierres de mes y envíos de correo).
    \end{itemize}
    \item \textbf{Gestión de cambios de guardia}
    \begin{itemize}
        \item RF-45. Solicitud de cambio de guardia por parte de un usuario de consulta.
        \item RF-46. Asociación de la solicitud a una guardia concreta y a un profesional receptor.
        \item RF-47. Aceptación o rechazo inicial de la solicitud por el profesional receptor.
        \item RF-48. Validación del cambio por el jefe de guardia del día afectado.
        \item RF-49. Aprobación o rechazo final del cambio por parte de un administrador.
        \item RF-50. Actualización automática del cuadrante tras la aprobación del cambio.
        \item RF-51. Registro del cambio de guardia en el sistema de auditoría.
    \end{itemize}

\end{itemize}

\subsection{Requisitos no funcionales}\label{sec:requisitos-no-funcionales}

Los requisitos no funcionales representan cómo tiene que trabajar la aplicación. Se denotarán como RNF-<nº requisito>.

\begin{itemize}
    \item RNF-1. La aplicación solamente tendrá un usuario con el rol de administrador.
    \item RNF-2. El borrado de registros en la base de datos se realizará mediante borrado lógico (soft delete), cuando aplique, mediante la actualización de un campo.

    \item RNF-3. La aplicación deberá implementar autenticación segura mediante Laravel Sanctum para el acceso de usuarios registrados.
    \item RNF-4. La autorización deberá estar basada en roles, de forma que cada usuario solo pueda acceder a las funcionalidades correspondientes.
    \item RNF-5. Las contraseñas deberán almacenarse utilizando algoritmos de hash seguros proporcionados por el framework.
    \item RNF-6. La aplicación deberá protegerse frente a ataques comunes en aplicaciones web, tales como inyección SQL, XSS y CSRF.

    \item RNF-7. La comunicación entre el frontend (React) y el backend (Laravel) se realizará exclusivamente mediante una API REST.
    \item RNF-8. La API deberá devolver respuestas consistentes con códigos HTTP apropiados y mensajes de error tratables por el frontend.
    \item RNF-9. La API deberá permitir el acceso únicamente desde orígenes autorizados mediante una configuración de CORS.
    \item RNF-10. La gestión de sesión/autenticación en la SPA deberá mantenerse de forma segura utilizando los mecanismos proporcionados por Sanctum.

    \item RNF-11. Las operaciones críticas (importación de Excel, cierre/reapertura de mes y asignaciones manuales) deberán ejecutarse de forma transaccional para garantizar la integridad de los datos.
    \item RNF-12. El sistema deberá garantizar la integridad referencial entre entidades (guardias, profesionales, especialidades, meses, etc.).
    \item RNF-13. El sistema deberá registrar un historial/auditoría de acciones relevantes, indicando usuario, fecha y detalle de la operación.

    \item RNF-14. La interfaz deberá ser responsiva y compatible con los principales navegadores web actuales.
    \item RNF-15. La aplicación deberá ser usable por personal no técnico, priorizando claridad en textos, navegación y mensajes de validación.

    \item RNF-16. El código deberá seguir buenas prácticas de desarrollo y una estructura modular que facilite el mantenimiento y la escalabilidad.
    \item RNF-17. El proyecto deberá incluir control de versiones (por ejemplo, Git) y un flujo de trabajo en equipo que facilite la integración de cambios.
    \item RNF-18. Las solicitudes de cambio de guardia deberán seguir un flujo de validación que incluya al jefe de guardia y a un administrador.

\end{itemize}

\subsection{Requisitos de la interfaz}\label{sec:requisitos-interfaz}

La interfaz es el medio que permite la comunicación entre el usuario y el sistema. En esta sección se enumeran los requisitos que debe cumplir la interfaz para que pueda ser utilizada por todos los tipos de usuario.

Los requisitos de la interfaz especifican cómo debe ser la comunicación entre el usuario y la parte visible de la aplicación. Se denotarán como RINT-<nº de requisito>.

\begin{itemize}
    \item RINT-1. La interfaz deberá ser responsiva y adaptarse a distintos tamaños de pantalla.
    \item RINT-2. La navegación deberá ser clara y consistente en todas las vistas de la aplicación.
    \item RINT-3. La interfaz deberá mostrar menús y opciones en función del rol del usuario autenticado.
    \item RINT-4. La aplicación deberá informar al usuario mediante mensajes claros de éxito o error tras cada operación relevante.
    \item RINT-5. La interfaz deberá mostrar indicadores de carga durante operaciones que requieran tiempo de procesamiento.
    \item RINT-6. Las vistas diaria y mensual deberán presentar la información de guardias de forma estructurada y fácilmente legible.
\end{itemize}

\subsection{Requisitos de información}\label{sec:requisitos-información}

Los requisitos de información hacen referencia a los datos que debe gestionar la aplicación web. Se denotarán como RI-<nº de requisito>.

Se deberá almacenar la siguiente información:
\begin{itemize}
    \item RI-1. Información de usuarios del sistema, incluyendo credenciales y rol asociado.
    \item RI-2. Información de profesionales (facultativos), incluyendo datos identificativos y fecha de inicio en el centro.
    \item RI-3. Información de especialidades del hospital.
    \item RI-4. Información de guardias, incluyendo fecha, profesional, especialidad y tipo de guardia (CA, PF, LOC).
    \item RI-5. Información de los meses de guardias y su estado (abierto o cerrado).
    \item RI-6. Información de la jefatura global diaria, incluyendo profesional asignado y motivo en caso de asignación manual.
    \item RI-7. Información de tarifas por tipo de guardia y plus de jefatura.
    \item RI-8. Información de documentos generados (PDF diario y Excel mensual).
    \item RI-9. Registro de auditoría de acciones relevantes realizadas en el sistema.
    \item RI-10. Información de las solicitudes de cambio de guardia, incluyendo guardia afectada, usuario solicitante, profesional receptor, estado de la solicitud y validaciones realizadas.

\end{itemize}

Una descripción más detallada de la información que se va a gestionar se puede consultar en el capítulo \ref{cap:modelo_de_datos} de Modelo de Datos.