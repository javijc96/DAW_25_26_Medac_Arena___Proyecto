\chapter{Especificación de requisitos}\label{cap:especificacion_requisitos}

\section{Introducción}


Se desea desarrollar una aplicación web que permita la gestión de servicios de suscripción para el alquiler de material de gimnasia. Esta plataforma estará orientada tanto a gimnasios como a empresas con recursos limitados y a usuarios particulares que deseen acceder a material deportivo sin realizar una inversión elevada. 

Las siguientes secciones describen los actores del sistema (Sección \ref{sec:actores}), la descripción modular (Sección \ref{sec:modulos}) y los requisitos del sistema (Sección \ref{sec:requisitos-del-sistema}).


\section{Actores del sistema}\label{sec:actores}
Se consideran los siguientes tipos de usuario en la aplicación web:

\begin{itemize}
    \item \textbf{Usuario invitado}: usuario no registrado que puede consultar información general sobre la plataforma, el catálogo básico y los niveles de suscripción.
    \item \textbf{Usuario registrado}: usuario que dispone de una cuenta activa y puede gestionar su suscripción, consultar material disponible según su plan, realizar solicitudes de alquiler o renovación y modificar su información personal.
    \item \textbf{Administrador}: usuario registrado con permisos avanzados con un control total sobre la aplicación. Será responsable de la gestión de usuarios, suscripciones, material, inventario y operaciones internas.
\end{itemize}

\section{Módulos de la aplicación}\label{sec:modulos}

La aplicación web estará compuesta por módulos correspondientes a cada tipo de usuario. En las siguientes subsecciones se detalla la funcionalidad de cada uno de ellos.

\subsection{Módulo del usuario}

El usuario registrado podrá realizar las siguientes acciones:
\begin{itemize}
    \item Consultar el catálogo de material disponible según su nivel de suscripción.
    \item Solicitar el alquiler de material deportivo.
    \item Gestionar sus renovaciones o devoluciones.
    \item Modificar sus datos personales.
    \item Consultar su historial de uso y facturación.
    \item Acceder a soporte y documentación de ayuda.
\end{itemize}

El usuario invitado podrá:
\begin{itemize}
    \item Consultar la información general de la plataforma.
    \item Visualizar los distintos niveles de suscripción.
    \item Consultar un catálogo básico del material disponible.
    \item Registrarse en la plataforma para acceder a las funcionalidades completas.
\end{itemize}

\subsection{Módulo del administrador}

El usuario administrador tendrá control total de la aplicación y será responsable de los siguientes módulos:

\begin{itemize}
    \item Gestión del catálogo de material deportivo disponible.
    \item Gestión de usuarios registrados.
    \item Gestión de los niveles o planes de suscripción.
    \item Supervisión de pedidos, renovaciones y devoluciones.
    \item Control del inventario y del estado del material.
    \item Gestión de incidencias y solicitudes especiales.
    \item Acceso al registro de actividad de la aplicación.
\end{itemize}

En principio, la aplicación solo tendrá un único usuario con rol de administrador.

\section{Requisitos del sistema}\label{sec:requisitos-del-sistema}

Los requisitos del sistema hacen referencia a todas las características relacionadas con la aplicación web. Se describirán los siguientes tipos de requisitos:
\begin{itemize}
    \item Requisitos funcionales: describen las tareas que la aplicación debe realizar para satisfacer las necesidades del problema (Sección \ref{sec:requisitos-funcionales}).
    \item Requisitos no funcionales: describen cómo se tiene que satisfacer las necesidades del problema  (Sección \ref{sec:requisitos-no-funcionales}).
    \item Requisitos de la interfaz: describen cómo debe ser la comunicación entre el usuario y la aplicación  (Sección \ref{sec:requisitos-interfaz}).
    \item Requisitos de la información: describen las características de los datos que se van a gestionar (Sección \ref{sec:requisitos-información}).
\end{itemize}


\subsection{Requisitos funcionales}\label{sec:requisitos-funcionales}

Los requisitos funcionales indican lo que el sistema debe hacer. Cada uno de estos requisitos debe tener dos propiedades: 
\begin{itemize}
    \item Ser completo: el requisito debe mencionar exactamente lo que el sistema debe hacer
    \item Ser cerrado: el requisito debe ser claro y no estar abierto a múltiples interpretaciones, sino solamente a una.
\end{itemize}

Los requisitos funcionales que se van a considerar se agruparán según los módulos de los tipos de usuario y se denotarán como RF-\textit{n}, donde \textit{n} es el número de requisito.

 \begin{itemize}
 \item \textbf{Módulo del usuario}
 \item[] La aplicación debe permitir que el usuario público pueda realizar las siguientes acciones:
    \begin{itemize}
        \item RF-1. Consultar el catálogo de material disponible.
        \item RF-2. Registrarse en la plataforma mediante un formulario.
        \item RF-3. Iniciar sesión con sus credenciales.
        \item RF-4. Seleccionar o modificar su plan de suscripción.
        \item RF-5. Solicitar el alquiler de material según su suscripción.
        \item RF-6. Consultar su historial de pedidos.
        \item RF-7. Gestionar renovaciones y devoluciones.
        \item RF-8. Modificar sus datos personales.
        \item RF-9. Consultar documentación de ayuda.
    \end{itemize}

 \item \textbf{Módulo del administrador}
 \item[] La aplicación debe permitir que el administrador pueda realizar las siguientes acciones:
    \begin{itemize}
        \item RF-10. Gestionar usuarios registrados.
            \begin{itemize}
                \item RF-10.1. Crear un usuario.
                \item RF-10.2. Buscar un usuario.
                \item RF-10.3. Consultar un usuario.
                \item RF-10.4. Modificar un usuario.
                \item RF-10.5. Eliminar un usuario.
            \end{itemize}

        \item RF-11. Gestionar el catálogo de material deportivo.
            \begin{itemize}
                \item RF-11.1. Registrar nuevo material.
                \item RF-11.2. Consultar material.
                \item RF-11.3. Modificar información del material.
                \item RF-11.4. Dar de baja material (borrado lógico).
            \end{itemize}

        \item RF-12. Gestionar los planes de suscripción.
            \begin{itemize}
                \item RF-12.1. Crear un plan de suscripción.
                \item RF-12.2. Modificar un plan existente.
                \item RF-12.3. Eliminar un plan (soft delete).
            \end{itemize}

        \item RF-13. Supervisar pedidos de usuarios.
        \item RF-14. Gestionar incidencias reportadas por los usuarios.
        \item RF-15. Consultar reportes y estadísticas del sistema.
    \end{itemize}

\end{itemize}

\subsection{Requisitos no funcionales}\label{sec:requisitos-no-funcionales}

Los requisitos no funcionales representan cómo tiene que trabajar la aplicación. Los requisitos no funcionales se denotarán como RNF-\textit{n}, donde \textit{n} es el número de requisito.


\begin{itemize}
    \item RNF-1. La aplicación solo tendrá un usuario con rol de administrador.
    \item RNF-2. El borrado de registros se realizará mediante soft delete.
    \item RNF-3. La aplicación debe ser accesible desde cualquier navegador moderno.
    \item RNF-4. La interfaz debe ser completamente responsiva.
    \item RNF-5. Las operaciones deben ejecutarse con un tiempo de respuesta inferior a 2 segundos.
    \item RNF-6. La aplicación debe seguir estándares de seguridad como HTTPS, hash de contraseñas y validación del lado del servidor.
    \item RNF-7. El sistema debe permitir una escalabilidad mínima para soportar al menos 500 usuarios simultáneos.
    \item RNF-8. La base de datos debe permitir integridad referencial y transacciones seguras.
\end{itemize}


\subsection{Requisitos de la interfaz}\label{sec:requisitos-interfaz}

  La interfaz es el dispositivo que permite la comunicación entre el usuario y el sistema. En esta sección, se enumeran los requisitos que debe tener la interfaz para que pueda ser utilizada por todos los tipos de usuario.
  
Los requisitos de la interfaz especifican cómo debe ser la comunicación entre el usuario y la parte visible de la aplicación. Se denotarán como RINT-\textit{n}, donde \textit{n} es el número de requisito.


\begin{itemize}
    \item RINT-1. La interfaz debe ser clara, intuitiva y coherente en todas las pantallas.
    \item RINT-2. Debe cumplir con los criterios básicos de accesibilidad WCAG.
    \item RINT-3. Debe permitir una navegación sencilla entre módulos.
    \item RINT-4. Los formularios deben incluir validaciones tanto visuales como textuales.
    \item RINT-5. Todos los mensajes de error deben mostrarse de forma visible y comprensible.
    \item RINT-6. El usuario debe recibir retroalimentación en cada acción (confirmaciones, avisos, alertas).
    
\end{itemize}

\subsection{Requisitos de información}\label{sec:requisitos-información}

Los requisitos de información hacen referencia a los datos que debe gestionar la aplicación web. Se denotarán como RI-\textit{n}, donde \textit{n} es el número de requisito.


Se deberá almacenar la siguiente información:
\begin{itemize}
    \item RI-1. Se debe almacenar la información personal de los usuarios (nombre, email, dirección, contraseña).
    \item RI-2. Se debe almacenar información de cada suscripción (nivel, fecha de inicio, renovación, estado).
    \item RI-3. Se debe registrar el catálogo completo de material deportivo disponible.
    \item RI-4. Se debe almacenar información sobre pedidos, devoluciones y renovaciones.
    \item RI-5. Se debe guardar el historial de actividad del administrador.
    \item RI-6. Se deben almacenar incidencias reportadas por los usuarios.
\end{itemize}

        
Una descripción más detallada de la información que se va a gestionar se puede consultar en el capítulo \ref{cap:modelo_de_datos} de Modelo de Datos.

