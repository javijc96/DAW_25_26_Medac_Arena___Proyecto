\chapter{Especificación de requisitos}\label{cap:especificacion_requisitos}

\section{Introducción}

Se desea desarrollar una aplicación web que permita la la gestión de.... Las siguientes secciones describen los actores del sistema (Sección \ref{sec:actores}), la descripción modular (Sección \ref{sec:modulos}), los requisitos del sistema (Sección  \ref{sec:requisitos-del-sistema}).

\section{Actores del sistema}\label{sec:actores}

Se van a considerar los siguientes tipos de usuario en la aplicación web: 
    \begin{itemize}
    \item Usuario 
    \item Administrador: usuario registrado en el sistema que...   
    \end{itemize}

\section{Módulos de la aplicación}\label{sec:modulos}

La aplicación web va a estar compuesta por módulos que corresponderán a cada uno de los tipos de usuario considerados y que se describen en las siguientes secciones.

\subsection{Módulo del usuario...}

El usuario... podrá :
\begin{itemize}
    \item 
\end{itemize}

\subsection{Módulo del administrador}

El usuario de tipo Administrador tendrá un control total de la aplicación. Además, se encargará de forma exclusiva de los siguientes módulos:
\begin{itemize}
    \item Gestión...
\end{itemize}

En principio, la aplicación solamente tendrá un único administrador.

\section{Requisitos del sistema}\label{sec:requisitos-del-sistema}

Los requisitos del sistema hacen referencia a todas las características relacionadas con la aplicación web. Se describirán los siguientes tipos de requisitos:
\begin{itemize}
    \item Requisitos funcionales: describen las tareas que la aplicación debe realizar para satisfacer las necesidades del problema (Sección \ref{sec:requisitos-funcionales}).
    \item Requisitos no funcionales: describen cómo se tiene que satisfacer las necesidades del problema  (Sección \ref{sec:requisitos-no-funcionales}).
    \item Requisitos de la interfaz: describen cómo debe ser la comunicación entre el usuario y la aplicación  (Sección \ref{sec:requisitos-interfaz}).
    \item Requisitos de la información: describen las características de los datos que se van a gestionar (Sección \ref{sec:requisitos-información}).
\end{itemize}


\subsection{Requisitos funcionales}\label{sec:requisitos-funcionales}

Los requisitos funcionales indican lo que el sistema debe hacer. Cada uno de estos requisitos debe tener dos propiedades: 
\begin{itemize}
    \item Ser completo: el requisito debe mencionar exactamente lo que el sistema debe hacer
    \item Ser cerrado: el requisito debe ser claro y no estar abierto a múltiples interpretaciones, sino solamente a una.
\end{itemize}

Los requisitos funcionales que se van a considerar se agruparán según los módulos de los tipos de usuario y se denotarán como RF-<nº requisito>.

 \begin{itemize}
 \item \textbf{Módulo del usuario...}
 \item[] La aplicación debe permitir que el usuario público pueda realizar las siguientes acciones:
     \begin{itemize}
         \item RF-1. Consultar 
     \end{itemize}

 \item \textbf{Módulo del administrador}
 \item[] La aplicación debe permitir que el administrador pueda realizar las siguientes acciones:
     \begin{itemize}
        \item RF-2. Gestionar... .
             \begin{itemize}
                  \item RF-2.1. Crear un .
                  \item RF-2.2. Buscar un .
                  \item RF-2.3. Consultar un .
                  \item RF-2.4. Modificar un .
                  \item RF-2.5. Eliminar un .
             \end{itemize} 
        \item RF-3. Gestionar los ...
             \begin{itemize}
                  \item RF-3.1. Crear...
             \end{itemize} 
     \end{itemize}
 \end{itemize}

\subsection{Requisitos no funcionales}\label{sec:requisitos-no-funcionales}

Los requisitos no funcionales representan cómo tiene que trabajar la aplicación. Los requisitos no funcionales se denotarán como RNF-<nº requisito>.

\begin{itemize}
    \item RNF-1. La aplicación solamente tendrá un usuario con el rol de administrador.
    \item RNF-2. El borrado de todos registros en la base de datos serán borrados lógicos (soft delete), mediante actualización de un campo.
    \item ...
\end{itemize}


\subsection{Requisitos de la interfaz}\label{sec:requisitos-interfaz}
  
  La interfaz es el dispositivo que permite la comunicación entre el usuario y el sistema. En esta sección, se enumeran los requisitos que debe tener la interfaz para que pueda ser utilizada por todos los tipos de usuario.
  
  Los requisitos la interfaz  especifican cómo deber la comunicación entre el usuario y la parte visible de la aplicación. Se denotarán como RINT-<nº de requisito> 


\begin{itemize}
    \item RINT-1. 
\end{itemize}

\subsection{Requisitos de información}\label{sec:requisitos-información}

Los requisitos de información hacen referencia a los datos que debe gestionar la aplicación web. Se denotarán como RI-<no de requisito>.

Se deberá almacenar la siguiente información:
\begin{itemize}
    \item RI-1. 
\end{itemize}

        
Una descripción más detallada de la información que se va a gestionar se puede consultar en el capítulo \ref{cap:modelo_de_datos} de Modelo de Datos.

