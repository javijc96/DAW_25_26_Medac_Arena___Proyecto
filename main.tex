\documentclass[a4paper,12pt,twoside,final]{book}

\usepackage[utf8]{inputenc}

\usepackage[spanish,es-nodecimaldot]{babel}
\usepackage[backend=biber, maxbibnames=99, giveninits=true]{biblatex}

\addbibresource{references.bib}
% COLOR
\usepackage[usenames,dvipsnames,svgnames,table]{xcolor}

% Colores para estilo Proyecto
\definecolor{lightback}{HTML}{F4E0BF}
     \definecolor{back}{HTML}{F3C591}
\definecolor{lightline}{HTML}{FCAF5F}
     \definecolor{line}{HTML}{ED7900}
% Colores para portada
\definecolor{medac:oscuro}{HTML}{002C4E}
 \definecolor{medac:medio}{HTML}{4C5CC5}
 \definecolor{medac:claro}{HTML}{3FCFD5}
 \definecolor{medac:verde}{HTML}{00B299}

\usepackage{lscape}

\usepackage{tikz} % used in cover to place images

\usepackage{datetime} % allow formal date format
\newcommand\Monthname[1][EMPTY]{
  \ifnum1=#1Enero\else
  \ifnum2=#1Febrero\else
  \ifnum3=#1Marzo\else
  \ifnum4=#1Abril\else
  \ifnum5=#1Mayo\else
  \ifnum6=#1Junio\else
  \ifnum7=#1Julio\else
  \ifnum8=#1Agosto\else
  \ifnum9=#1Septiembre\else
  \ifnum10=#1Octubre\else
  \ifnum11=#1Noviembre\else
  \ifnum12=#1Diciembre\else
  \fi\fi\fi\fi\fi\fi\fi\fi\fi\fi\fi\fi
}
\newdateformat{monthyeardate}{%
  \Monthname[\THEMONTH], \THEYEAR}

% FONTs
\usepackage[T1]{fontenc}
\usepackage{textcomp}% Needed for new symbols like € ?

\usepackage[scaled]{berasans} % Font for the cover similar to Vera 33
%\renewcommand*\familydefault{\sfdefault}  %% To use as the base font of the document is to be sans serif


% PAGE STYLE
\usepackage[twoside,bindingoffset=0cm,headheight=30pt,margin=25mm]{geometry} %,verbose,showframe
\usepackage{fancyhdr} % Encabezados
\pagestyle{fancy}
\fancyhf{}
\fancyhead[LE]{\leftmark}
\fancyhead[RO]{\rightmark}
\fancyfoot[RO,LE]{\thepage}

% Sangría para párrafo, tabulaciones de 1 cm
\parindent=1.0cm
% Espacio o separación entre párrafos
\parskip 1.5ex

% TABLES AND FIGURES
\usepackage{graphicx}
\graphicspath{ {img/} }

\usepackage{todonotes}
\usepackage{float}
\usepackage{tabularx}
\usepackage{url}
\usepackage{emptypage}
\usepackage[toc,page]{appendix}
\usepackage{hyperref}
\hypersetup{ colorlinks=true,  %habilitar colorear enlaces
            linkcolor=black,
            filecolor=black,
            urlcolor=cyan,
            citecolor=blue,
            }
\usepackage{array}
\usepackage{wrapfig}
\usepackage{multirow}
\usepackage{multicol} %Para poner columnas de un solo elemento
\usepackage{tabu}

\usepackage{pifont} %Para usar el vmark (tick) y xmark (cross)
\newcommand{\vmark}{\textcolor{green}{\ding{51}}} %Definición del vmark en verde
\newcommand{\xmark}{\textcolor{red}{\ding{55}}} %Definición del xmark en rojo

\usepackage{chngcntr}
\usepackage{verbatim}
\usepackage{graphicx}
\usepackage[export]{adjustbox}
\usepackage{listings}
\usepackage{minted}

\usepackage{booktabs,caption}
\usepackage[flushleft]{threeparttable}
\usepackage{fancyvrb}
\usepackage{verbatimbox}
\usepackage{afterpage}

\usepackage{csquotes}
\usepackage{rotating} % para rotar tablas

\newsavebox{\FVerbBox}
\newenvironment{FVerbatim}
 {\VerbatimEnvironment
  \begin{center}
  \begin{lrbox}{\FVerbBox}
  \begin{BVerbatim}}
 {\end{BVerbatim}
  \end{lrbox}
  \fbox{\usebox{\FVerbBox}}
  \end{center}}
  
  \newcommand\blankpage{%
    \null
    \thispagestyle{empty}%
    \addtocounter{page}{-1}%
    \newpage}
    
\counterwithout{footnote}{chapter}
  
\begin{document}
\frontmatter

%------------- Cover --------------
\thispagestyle{empty}

% Backgroud images
\begin{tikzpicture}[remember picture, overlay]
  % Top
  \node [anchor=north west, inner sep=50pt]  at (current page.north west)
     {\includegraphics[height=4cm]{img/topMEDAC.pdf}};
     
  % Bottom
  \node [anchor=south east, inner sep=0pt]  at (current page.south east)
     {\includegraphics[height=18cm]{img/bottomMEDAC.pdf}};

\end{tikzpicture}

\renewcommand*\listtablename{Índice de tablas}
\renewcommand{\tablename}{Tabla}

\fontfamily{\sfdefault}\selectfont
\vspace*{2cm}

\vfill\vfill\vfill\vfill\vfill
\vfill\vfill\vfill\vfill\vfill
\vfill\vfill\vfill\vfill\vfill

\LARGE\textbf{\color{medac:oscuro}
  [RentFit]
}
  
\vfill

\Large\textbf{\color{medac:verde}
  Desarrollo de Aplicaciones Web
}
\vfill

\Large\textbf{\color{medac:oscuro} Tutor: }
{\color{medac:medio}{ Javier Ruiz Jurado }}
\vfill

\Large\textbf{\color{medac:oscuro} Curso académico: }
{\color{medac:medio}{ 2025-2026 }}

\vfill\vfill\vfill

\Large\textbf{\color{medac:oscuro}Autores}
\begin{itemize}

    \item {\color{medac:medio}{ Raul Henares }}
    \item {\color{medac:medio}{ Antonio Moyano }}
    \item {\color{medac:medio}{ Jose Ramón Rejano }}
    \item {\color{medac:medio}{ Javier Juárez }}
\end{itemize}

\vfill\vfill

\textbf{\color{medac:verde} \monthyeardate\today}
\vfill
\vspace{2.7cm}

%----------------------------------------------------------------------------------------------
%\clearpage
%\thispagestyle{empty}
%\pagecolor{white}
%\cleardoublepage
%---------------------------------------------------------------------------------------------
\clearpage
\thispagestyle{empty}
\pagecolor{white}
\normalsize{}
“Hueco destinado a una frase de alguna persona célebre que tenga relación con vuestro proyecto”
[Autor de la frase] (Año de nacimiento-año de muerte)
\cleardoublepage

%----------------------------------------------------------------------------------------------
\cleardoublepage
\setcounter{page}{1}
\setcounter{tocdepth}{3} %Numeración anidada de profundidad 3 en el índice
\setcounter{secnumdepth}{3} %Numeración anidada de profundidad 3 en los capítulos
\tableofcontents\cleardoublepage
\mainmatter

%Insertar capítulos
\let\cleardoublepage\clearpage

\part{Presentación}
\chapter{Introducción}

\section{Contexto del problema a resolver}

    La gestión de turnos y guardias en instituciones sanitarias es una tarea crítica que requiere precisión,
    organización y una correcta asignación de recursos humanos.
    En hospitales de tamaño medio, como el hospital de Pozoblanco, 
    esta gestión influye directamente tanto en el correcto funcionamiento del servicio como en la satisfacción laboral 
    del personal sanitario.

    Actualmente, muchas de estas tareas siguen realizándose mediante herramientas genéricas como hojas de cálculo,
    lo que dificulta la automatización de procesos, la trazabilidad de la información y la reducción de errores humanos. 
    
    La ausencia de un sistema específico adaptado a las necesidades reales del hospital provoca ineficiencias que pueden 
    derivar en conflictos organizativos y errores en la planificación.

\section{Definición del problema real}
    
    Existe un problema con respecto a la gestión de las guardias del hospital de Pozoblanco.

    En resumen, el problema real que se desea resolver con el desarrollo de este proyecto es el siguiente:

    Actualmente, el sistema de guardias del hospital de Pozoblanco se gestiona mediante hojas de cálculo en formato Excel, 
    las cuales son administradas manualmente por los empleados responsables. Estos empleados deciden, 
    utilizando distintos términos y patrones no estandarizados, qué trabajadores se encargan de las guardias de cada día, 
    plasmando finalmente la información en un documento PDF.

    Este proceso resulta tedioso y propenso a errores, ya que gran parte de la gestión de guardias, 
    horas trabajadas y días libres (entre otros aspectos) depende del factor humano. 
    Esto puede provocar inconsistencias en los datos, errores de cálculo y dificultades a la hora de realizar modificaciones 
    o auditorías posteriores.
\section{Definición del problema técnico}

    El problema técnico consiste en la ausencia de una herramienta software específica que permita
    gestionar de forma automatizada, segura y centralizada las guardias del hospital de Pozoblanco.

    Desde el punto de vista técnico, se requiere el desarrollo de una aplicación web capaz de
    gestionar usuarios con distintos niveles de acceso, almacenar información estructurada sobre
    guardias y profesionales, y aplicar reglas de negocio para la planificación y validación de los
    turnos.

    Para abordar este problema, se utilizará una arquitectura basada en una aplicación web con un
    backend encargado de la lógica de negocio y un frontend responsable de la interacción con el
    usuario, comunicados mediante una API REST.

\subsection{Funcionamiento}

    Se desea implementar una aplicación web que permita realizar una gestión automatizada de las
    guardias del hospital, reduciendo la dependencia del factor humano y minimizando los errores
    derivados de la gestión manual.

    Los usuarios que interactuarán con la aplicación son los siguientes:

    \begin{itemize}
    \item \textbf{Administrador}
        \begin{itemize}
        \item Usuario con acceso completo a las funcionalidades del sistema.
        \item Responsable de la gestión de usuarios, guardias, jefaturas y validaciones.
        \end{itemize}

    \item \textbf{Usuario de consulta}
        \begin{itemize}
            \item Usuario con acceso a funcionalidades de consulta de guardias.
            \item Sin permisos para modificar información crítica del sistema.
        \end{itemize}
    \end{itemize}
\subsection{Entorno}
Existirán dos entornos de trabajo para esta aplicación:

\begin{itemize}
\item \textbf{Entorno de desarrollo y testeo}: entorno utilizado para desarrollar, probar y depurar la aplicación web antes de su despliegue.
\item \textbf{Entorno de producción}: entorno en el que se encontrará la versión final de la aplicación web accesible para los usuarios finales.
\end{itemize}

Asimismo, existirán dos entornos de ejecución:

\begin{itemize}
\item \textbf{Entorno de ejecución del administrador}: acceso completo a las funcionalidades del sistema.
\item \textbf{Entorno de ejecución del resto de usuarios}: acceso limitado a funcionalidades de consulta.
\end{itemize}

La interfaz será el medio de comunicación entre los usuarios y la aplicación web. Es fundamental que sea intuitiva y amigable para garantizar una buena experiencia de usuario. Los requisitos de la interfaz se describirán en la Sección \ref{sec:requisitos-interfaz}.  
\subsection{Vida esperada}

    El ciclo de vida esperado de la aplicación web será elevado, ya que se ha diseñado siguiendo una
    arquitectura modular y escalable que permite su adaptación a futuras necesidades del hospital,
    como la incorporación de nuevas funcionalidades o la ampliación de los tipos de usuario.

\subsection{Ciclo de mantenimiento} \label{subsec:ciclo_mantenimiento}

Dado que el desarrollo de la aplicación web se enmarca dentro de un proyecto académico, el
mantenimiento posterior de la aplicación no correrá a cargo de su autor.

No obstante, se ha diseñado una estructura modular que facilite posibles tareas de mantenimiento,
corrección de errores o ampliaciones futuras por parte de terceros.

\subsection{Competencia}

    En la actualidad existen soluciones comerciales para la gestión de turnos y guardias, así como
    herramientas genéricas utilizadas para este fin. No obstante, muchas de estas soluciones no se
    adaptan completamente a las necesidades específicas del hospital de Pozoblanco.

    Este proyecto se plantea como una solución personalizada, orientada a cubrir de forma concreta
    los requisitos del contexto hospitalario descrito.

\subsection{Aspecto externo}

En relación con el aspecto externo, se tendrán en cuenta los siguientes aspectos:

\begin{itemize}
\item \textbf{Interfaz de usuario}
\begin{itemize}
\item Se desarrollará una interfaz totalmente responsiva, intuitiva y amigable. El Capítulo \ref{cap:diseño_interfaz} describirá el diseño de la interfaz de la aplicación web.
\end{itemize}
\item \textbf{Distribución de la aplicación}
\begin{itemize}
\item La aplicación se podrá distribuir mediante un servidor web accesible a través de un navegador estándar.
\end{itemize}
\end{itemize}


\subsection{Estandarización}

    Para el desarrollo de la aplicación web, se revisarán las recomendaciones propuestas por \textit{World Wide Web Consortium} (W3C) \cite{w3c}, que ``promueve el uso de estándares para reducir el coste y la complejidad del desarrollo, así como para incrementar la accesibilidad y usabilidad de cualquier documento publicado en la web''.

    También se debe tener en cuenta que los recursos  que se van a utilizar son herramientas informáticas que están validadas por prestigiosas organizaciones que indican que cumplen con los estándares. Véase el capítulo \ref{cap:recursos} de Recursos. 

\subsection{Calidad y fiabilidad}

La calidad y fiabilidad de la aplicación web estarán garantizadas por:

\begin{itemize}
    \item Uso de un framework consolidado como Laravel y React.
    \item Aplicación de buenas prácticas de programación.
    \item Realización de pruebas funcionales y de seguridad.
\end{itemize}

\subsection{Programa de tareas}

 El desarrollo del presente proyecto va estar compuesto por la siguientes fases:
 \begin{itemize}
     \item Introducción: descripción del problema, establecimiento de los objetivos, revisión de antecedentes, identificación de restricciones iniciales y estratégicas y selección de recursos.
     \item Análisis: especificación de requisitos (funcionales, no funcionales, de información y de la interfaz), descripción del modelo de datos y análisis funcional (casos de uso y diagramas de secuencia).
    
     \item Diseño: descripción del diseño de datos, clases, paquetes y de la interfaz.
    
     \item Implementación: codificación de la aplicación web teniendo en cuenta el diseño desarrollado.
    
     \item Pruebas: comprobación de que la aplicación web funciona correctamente, es robusta y amigable.
 \end{itemize}


\subsection{Pruebas}

La fase de pruebas es esencial para garantizar que la aplicación web funciona correctamente. En particular, se pretende comprobar que la aplicación web:
\begin{itemize}
    \item Hace lo que debe hacer.
    \item No provoca efectos secundarios que pueden desencadenar situaciones catastróficas.
    \item Contiene módulos que se ejecutan correctamente.
    \item Garantiza los privilegios de cada tipo de usuario.
\end{itemize}

Cada prueba tendrá la siguiente estructura para detectar los errores y corregirlos:
\begin{itemize}
    \item Objetivo de la prueba. Se debe indicar en qué consiste la prueba y el resultado esperado.
    \item Problema detectado, en su caso. Si ocurre un error entonces se debe describir la causa que lo ha provocado.
    \item Solución adoptada, en su caso. Si se ha producido un error, se deben indicar las medidas tomadas para solucionarlo.
\end{itemize}

El Capítulo \ref{cap:pruebas} de Pruebas describirá las pruebas realizadas.

\subsection{Seguridad}

    Durante el desarrollo de la aplicación web se tendrán en cuenta aspectos fundamentales de
    seguridad, entre los que se incluyen:

    \begin{itemize}
    \item Autenticación y autorización de usuarios.
    \item Protección contra ataques de inyección SQL.
    \item Gestión segura de sesiones.
    \item Protección de datos sensibles.
    \end{itemize}

Estos aspectos se describirán con mayor detalle en los capítulos correspondientes.

\chapter{Objetivos}
\section{Objetivo principal}

El objetivo principal de este proyecto es el desarrollo de una aplicación web que permita la
gestión automatizada y centralizada de las guardias del hospital de Pozoblanco,
sustituyendo el actual sistema manual basado en hojas de cálculo por una solución informática
fiable, segura y fácil de usar.

La aplicación estará orientada a mejorar la eficiencia en la planificación de las guardias,
reducir errores humanos y facilitar el acceso a la información tanto a los administradores
como al personal del hospital.

\section{Objetivos específicos}

Los objetivos específicos de este proyecto son los siguientes:


\begin{itemize}
    \item Diseñar e implementar un sistema de gestión de usuarios con distintos niveles de acceso,
    diferenciando entre administradores y usuarios de consulta.

    \item Desarrollar un sistema de generación automática de cuadrantes mensuales de guardias,
    aplicando reglas de negocio definidas por el hospital.

    \item Permitir la consulta de guardias por parte del personal del hospital mediante vistas
    diarias y mensuales, organizadas por sección.

    \item Implementar un mecanismo de solicitud y validación de cambios de guardia entre profesionales,
    sujeto a un flujo de aprobación controlado.

    \item Diseñar y desarrollar una base de datos relacional que permita almacenar de forma segura
    toda la información relacionada con usuarios, guardias, secciones y jefaturas.

    \item Proporcionar herramientas de gestión para el administrador que faciliten la supervisión,
    modificación y validación de las guardias.

    \item Diseñar una interfaz de usuario intuitiva, accesible y adaptable a distintos dispositivos
    y navegadores web.

    \item Garantizar la seguridad del sistema mediante mecanismos de autenticación, autorización
    y control de acceso adecuados.
\end{itemize}

\begin{itemize}
\item \textbf{Tipos de usuarios}
\item[] Se deberán permitir los siguientes tipos de usuarios:
\begin{itemize}
    \item \textbf{Administrador}
    \begin{itemize}
        \item Este tipo de usuario estará registrado en el sistema y dispondrá de control completo sobre la aplicación.
        \item Tendrá competencias exclusivas para la gestión de usuarios, creación y modificación de guardias, asignación de turnos, consulta de estadísticas y generación de informes.
    \end{itemize}

    \item \textbf{Usuario estándar}
    \begin{itemize}
        \item Podrá consultar las guardias que tiene asignadas, así como su histórico de horas trabajadas y días libres.
        \item No dispondrá de permisos para modificar información crítica del sistema.
    \end{itemize}
\end{itemize}

\item \textbf{Base de datos relacional}
\item[] Se deberá diseñar una base de datos relacional que permita gestionar toda la información relacionada con el sistema de guardias, incluyendo:
\begin{itemize}
    \item Usuarios registrados y sus roles.
    \item Guardias y turnos asignados.
    \item Fechas, horarios y tipos de guardia.
    \item Históricos de asignaciones y modificaciones.
\end{itemize}

\item \textbf{Módulos del sistema}
\item[] Se deberán diseñar e implementar los siguientes módulos principales:
\begin{itemize}
    \item \textbf{Módulo del administrador}
    \begin{itemize}
        \item Gestión de usuarios registrados.
        \item Creación, modificación y eliminación de guardias.
        \item Asignación y reorganización de turnos.
        \item Generación de informes y consultas.
    \end{itemize}

    \item \textbf{Módulo del usuario estándar}
    \begin{itemize}
        \item Consulta de guardias asignadas.
        \item Visualización de calendario de turnos.
        \item Acceso a información personal relacionada con horas trabajadas.
    \end{itemize}
\end{itemize}

\item \textbf{Diseño de la interfaz}
\item[] Se deberá diseñar una interfaz de usuario intuitiva, robusta, amigable y adaptable a distintos dispositivos y navegadores web, garantizando una experiencia de uso satisfactoria.

\item \textbf{Seguridad y control de acceso}
\item[] Se deberán implementar mecanismos de autenticación y autorización que garanticen que cada usuario solo pueda acceder a las funcionalidades correspondientes a su rol.

\end{itemize}

\section{Objetivos personales}

Se proponen los siguientes objetivos personales:

\begin{itemize}
\item Aprender y afianzar conocimientos en el desarrollo de aplicaciones web utilizando
    el framework Laravel y la librería React.

    \item Profundizar en el uso del patrón Modelo--Vista--Controlador (MVC).

    \item Mejorar habilidades en el diseño y gestión de bases de datos relacionales.

    \item Aplicar buenas prácticas de desarrollo de software y documentación técnica.

    \item Adquirir experiencia en el análisis, diseño, implementación y pruebas de un proyecto
    software completo.

    \item Conseguir que todos los miembros del grupo mejoren y se adapten al trabajo en equipo y tener a Raul contento.
\end{itemize}
\chapter{Recursos}\label{cap:recursos}

\section{Recursos humanos}
\begin{itemize}
    \item \textbf{Autor}
    \item[] 
    \item \textbf{Tutor}
    \item[] 
\end{itemize}

\section{Recursos de hardware}

Para el desarrollo de la aplicación en el entorno local...:
\begin{itemize}
    \item Ordenador: 
    \item Sistema operativo: 
    \item RAM instalada:
    \item Procesador: 
\end{itemize}

Para el despliegue de la aplicación web, se hará uso de un servidor VPS con las características siguientes:
\begin{itemize}
    \item Platform: 
    \item OS Package: 
    \item Memory: 
    \item SSD: 
\end{itemize}


\section{Recursos de software}

Se van a utilizar los siguientes recursos de software para el desarrollo de la aplicación web: 
\begin{itemize}
    \item Editor de código fuente para el desarrollo de la aplicación web: Visual Studio Code \cite{visualStudioCode}
\end{itemize}




\part{Contextualización empresarial}
\chapter{CONTEXTUALIZACIÓN EMPRESARIAL}

\section{Denominación}    
    RentFit tiene la intención de cubrir las necesidades de particulares y empresas de disponer de material de gimnasia que puede llegar
    a resultar muy costoso a traves de servicios de alquiler llevados a traves de un sistema de subscripciones de distintos rangos 

\section{Justificación de la idea del proyecto}
    El origen de este proyecto surge de la evidente necesidad que presentan numerosos gimnasios, centros deportivos y entrenadores de acceder a material actualizado, de buena calidad y en óptimas condiciones. A esto se suma el elevado coste que suelen tener estos productos y la situación económica de muchos establecimientos, que dificulta su adquisición de manera permanente. Como resultado, estos negocios no siempre pueden asumir el impacto financiero que supone la compra constante de nuevo equipamiento.
    RentFit propone una alternativa innovadora frente a la compra definitiva de equipamiento deportivo, ofreciendo un servicio de alquiler flexible y accesible. Este modelo permite que gimnasios, centros deportivos y entrenadores puedan disponer de material actualizado y en buen estado sin tener que asumir el elevado coste de adquisición. De esta manera, RentFit facilita el acceso a recursos de calidad, reduce la carga económica de los establecimientos y contribuye a que puedan renovar o ampliar su equipamiento de forma más rápida, eficiente y sostenible. Además, el alquiler permite adaptar el material a las necesidades cambiantes de cada negocio, evitando inversiones innecesarias y mejorando su capacidad de respuesta ante variaciones en la demanda.
\section{Estudio de la competencia}

Procederéis a identificar quiénes son las empresas con proyectos en marcha que puedan considerarse competidores (hay que tener en cuenta el concepto de competencia en su sentido más amplio, es decir, no sólo existen competidores directos, sino que también podemos tener competidores indirectos). 
    La competencia real a RentFit resulta componerse mas por competencia indirecta que por competencia directa ya que existen pocas empresas que se dediquen
    al alquiler de este tipo de material que operen en Córdoba. 

Competencia directa

Aunque limitada, sí existen algunas empresas que ofrecen servicios similares al modelo planteado por RentFit. Una de las más destacadas es FitnessRent, empresa dedicada al alquiler de equipamiento deportivo, principalmente para hoteles, anuncios publicitarios y eventos temporales.

Si bien su público objetivo difiere del de RentFit —centrado más en entornos corporativos, hoteleros o promocionales—, el hecho de ofrecer un servicio de alquiler de material deportivo la convierte en un competidor directo relevante. Representa un ejemplo claro de que existe demanda para un modelo basado en el acceso temporal al equipamiento sin necesidad de realizar una compra definitiva.

Competencia indirecta

La mayoría de competidores de RentFit pertenecen a la categoría de competencia indirecta. Se trata de empresas que no alquilan material, pero que ofrecen una alternativa válida mediante la venta directa de equipamiento deportivo, obligando al cliente a realizar una inversión inicial mayor.

Entre estas empresas destacan:

Decathlon

Sprinter

ProFit

Otros distribuidores especializados en material deportivo y fitness

Estas compañías disponen de una amplia variedad de productos y cuentan con presencia consolidada en el mercado. Sin embargo, su modelo de negocio exige a los gimnasios y entrenadores asumir el coste completo del equipamiento, lo que RentFit busca precisamente evitar ofreciendo un sistema más flexible y económicamente accesible.

\section{Oportunidad de negocio}

 En este apartado se deberían valorar las oportunidades de negocio previsibles en el sector. ¿Es un producto o servicio que está demandando el mercado? ¿Económicamente es viable? Demuéstralo brevemente.

\section{Obligaciones fiscales}

Se debe especificar si el desarrollo del proyecto conlleva la necesidad de cumplimiento de obligaciones fiscales, laborales y/o de prevención de riesgos y sus condiciones de aplicación.
    
\section{Financiación, ayudas y subvenciones}

Habría que identificar las posibles necesidades de financiación para llevar a cabo el desarrollo del proyecto, así como la búsqueda de ayudas o subvenciones para la incorporación de las nuevas tecnologías de producción o de servicio propuestas.


    





\part{Análisis}
\chapter{Especificación de requisitos}\label{cap:especificacion_requisitos}

\section{Introducción}

Se desea desarrollar una aplicación web que permita la la gestión de.... Las siguientes secciones describen los actores del sistema (Sección \ref{sec:actores}), la descripción modular (Sección \ref{sec:modulos}), los requisitos del sistema (Sección  \ref{sec:requisitos-del-sistema}).

\section{Actores del sistema}\label{sec:actores}

Se van a considerar los siguientes tipos de usuario en la aplicación web: 
    \begin{itemize}
    \item Usuario 
    \item Administrador: usuario registrado en el sistema que...   
    \end{itemize}

\section{Módulos de la aplicación}\label{sec:modulos}

La aplicación web va a estar compuesta por módulos que corresponderán a cada uno de los tipos de usuario considerados y que se describen en las siguientes secciones.

\subsection{Módulo del usuario...}

El usuario... podrá :
\begin{itemize}
    \item 
\end{itemize}

\subsection{Módulo del administrador}

El usuario de tipo Administrador tendrá un control total de la aplicación. Además, se encargará de forma exclusiva de los siguientes módulos:
\begin{itemize}
    \item Gestión...
\end{itemize}

En principio, la aplicación solamente tendrá un único administrador.

\section{Requisitos del sistema}\label{sec:requisitos-del-sistema}

Los requisitos del sistema hacen referencia a todas las características relacionadas con la aplicación web. Se describirán los siguientes tipos de requisitos:
\begin{itemize}
    \item Requisitos funcionales: describen las tareas que la aplicación debe realizar para satisfacer las necesidades del problema (Sección \ref{sec:requisitos-funcionales}).
    \item Requisitos no funcionales: describen cómo se tiene que satisfacer las necesidades del problema  (Sección \ref{sec:requisitos-no-funcionales}).
    \item Requisitos de la interfaz: describen cómo debe ser la comunicación entre el usuario y la aplicación  (Sección \ref{sec:requisitos-interfaz}).
    \item Requisitos de la información: describen las características de los datos que se van a gestionar (Sección \ref{sec:requisitos-información}).
\end{itemize}


\subsection{Requisitos funcionales}\label{sec:requisitos-funcionales}

Los requisitos funcionales indican lo que el sistema debe hacer. Cada uno de estos requisitos debe tener dos propiedades: 
\begin{itemize}
    \item Ser completo: el requisito debe mencionar exactamente lo que el sistema debe hacer
    \item Ser cerrado: el requisito debe ser claro y no estar abierto a múltiples interpretaciones, sino solamente a una.
\end{itemize}

Los requisitos funcionales que se van a considerar se agruparán según los módulos de los tipos de usuario y se denotarán como RF-<nº requisito>.

 \begin{itemize}
 \item \textbf{Módulo del usuario...}
 \item[] La aplicación debe permitir que el usuario público pueda realizar las siguientes acciones:
     \begin{itemize}
         \item RF-1. Consultar 
     \end{itemize}

 \item \textbf{Módulo del administrador}
 \item[] La aplicación debe permitir que el administrador pueda realizar las siguientes acciones:
     \begin{itemize}
        \item RF-2. Gestionar... .
             \begin{itemize}
                  \item RF-2.1. Crear un .
                  \item RF-2.2. Buscar un .
                  \item RF-2.3. Consultar un .
                  \item RF-2.4. Modificar un .
                  \item RF-2.5. Eliminar un .
             \end{itemize} 
        \item RF-3. Gestionar los ...
             \begin{itemize}
                  \item RF-3.1. Crear...
             \end{itemize} 
     \end{itemize}
 \end{itemize}

\subsection{Requisitos no funcionales}\label{sec:requisitos-no-funcionales}

Los requisitos no funcionales representan cómo tiene que trabajar la aplicación. Los requisitos no funcionales se denotarán como RNF-<nº requisito>.

\begin{itemize}
    \item RNF-1. La aplicación solamente tendrá un usuario con el rol de administrador.
    \item RNF-2. El borrado de todos registros en la base de datos serán borrados lógicos (soft delete), mediante actualización de un campo.
    \item ...
\end{itemize}


\subsection{Requisitos de la interfaz}\label{sec:requisitos-interfaz}
  
  La interfaz es el dispositivo que permite la comunicación entre el usuario y el sistema. En esta sección, se enumeran los requisitos que debe tener la interfaz para que pueda ser utilizada por todos los tipos de usuario.
  
  Los requisitos la interfaz  especifican cómo deber la comunicación entre el usuario y la parte visible de la aplicación. Se denotarán como RINT-<nº de requisito> 


\begin{itemize}
    \item RINT-1. 
\end{itemize}

\subsection{Requisitos de información}\label{sec:requisitos-información}

Los requisitos de información hacen referencia a los datos que debe gestionar la aplicación web. Se denotarán como RI-<no de requisito>.

Se deberá almacenar la siguiente información:
\begin{itemize}
    \item RI-1. 
\end{itemize}

        
Una descripción más detallada de la información que se va a gestionar se puede consultar en el capítulo \ref{cap:modelo_de_datos} de Modelo de Datos.


\chapter{Modelo de datos} \label{cap:modelo_de_datos}

\section{Introducción}

En este capítulo se describirá conceptualmente el modelo de datos que empleará el sistema. Se identificarán las entidades que intervienen en el problema, sus atributos y las interrelaciones entre dichas entidades.

Para la confección del modelo, se empleará la notación del Modelo Entidad - Interrelación (modelo E-R) propuesto por Peter Chen%\cite{peter}. 
El esquema E-R describe la información representando los distintos elementos que la componen mediante un conjunto limitado de símbolos y reglas de relación entre ellos. Básicamente, los elementos principales del modelo son “tipo de entidad” y “tipo de interrelación”.

En las siguientes secciones se especifican con detalle los tipos de entidad y los tipos de interrelación que intervienen en el modelo, y se mostrará el diagrama E-R completo, que ofrecerá una visión global del problema.

\section{Tipos de entidad} \label{sec:entidades}

De acuerdo con el modelo E-R, un tipo de entidad representa a una serie de entes, objetos o personas reales o abstractos que forman parte del universo del problema a describir. Los tipos de entidad pueden ser fuertes o débiles.

\begin{itemize}

    \item Tipo de entidad fuerte: su existencia no depende de la de otro tipo de entidad.

    \item Tipo de entidad débil: su existencia depende de la de otro tipo de entidad. La debilidad puede ser por existencia o por identificación.

    \begin{itemize}
        \item Tipo de entidad débil por existencia: puede ser identificado por sí mismo a partir de sus atributos propios, pero requiere de la existencia de otro tipo de entidad del que depende.

        \item Tipo de entidad débil por identificación: es un tipo de entidad débil por existencia que, además, requiere de algún atributo identificativo del tipo de entidad del que depende para poder ser identificado y diferenciado.
    \end{itemize} 
\end{itemize}    

Las entidades de un determinado tipo de entidad se describen mediante un conjunto de atributos que representan cada una de las características o propiedades que lo describen.

Cada atributo tiene asociado un dominio de valores permitidos. Cada entidad se identifica y diferencia de forma inequívoca mediante un atributo identificador, que toma un valor único para cada entidad.

En esta sección se describirán todos los tipos de entidad que se han identificado que forman parte del problema descrito, indicando para cada uno de ellos la siguiente información:

\begin{itemize}
    \item Descripción: definición de la entidad y función dentro del universo del problema.
    
    \item Restricción: indicación de si tiene alguna debilidad por identificación o existencia respecto de otra entidad.
    
    \item Características: se indicarán las siguientes: 
    \begin{itemize}
        \item Nombre del tipo de entidad.
        \item Tipo: Fuerte o débil.
        \item Atributos heredados.
        \item Atributo identificador primario.
        \item Atributo identificador alternativo.
        \item Número de atributos, incluyendo los heredados.
    \end{itemize}
    
    \item Atributos propios: por cada atributo, además del nombre del mismo, se indicará:
    \begin{itemize}
        \item Definición: descripción del atributo.
        \item Dominio: tipo de dato o valores que puede tomar el atributo.
        \item Tipo: indicación de si es clave primaria o alterna, en su caso, atributo simple, etc.
        \item Opcional: indicación de si el atributo puede contener un valor nulo o no.
        \item Ejemplo: valor de muestra.
    \end{itemize}
    
    \item Diagrama: representación gráfica del tipo de entidad, de acuerdo con la notación E-R.
    
    \item Ejemplo de entidad.
\end{itemize}    

Los tipos de entidad que se han identificado y que se describirán a continuación son los siguientes:
\begin{itemize}
    \item Tipo de entidad: Usuario
    \item ...
\end{itemize}

\subsection{Entidad Usuario}
\begin{itemize}
    \item Descripción: este tipo de entidad representa a los usuarios registrados en el sistema que contratan los servicios de subscripciones para acceder a los productos
    \item Restricciones: es una entidad fuerte por lo que no depende de otro tipo de entidad.
    \item Características:
    \begin{itemize}
        \item Nombre de la entidad: Usuario.
        \item Tipo: fuerte.
        \item Atributos heredados: ninguno.
        \item Atributo identificador primario: id.
        \item Atributo identificador alternativo: ninguno.
        \item Número de atributos: 7 propios.
    \end{itemize}

    \item Atributos propios:
    \begin{itemize}
        \item id
        \begin{itemize}
            \item Definición: código numérico secuencial e incremental que identifica el usuario.
            \item Dominio: números enteros mayores que 0.
            \item Tipo: clave primaria.
            \item Opcional: no
            \item Ejemplo: 13
        \end{itemize}

        \item name
        \begin{itemize}
            \item Definición: nombre que identifica al usuario.
            \item Dominio: conjunto de 150 caracteres.
            \item Tipo: atributo simple.
            \item Opcional: no
            \item Ejemplo: Juan 
        \end{itemize}

        \item apellidos
        \begin{itemize}
            \item Definicion
            \item Definición: apellidos del usuario.
            \item Dominio: conjunto de 150 caracteres.
            \item Tipo: atributo simple.
            \item Opcional: no
            \item Ejemplo: Bautista
        \end{itemize}

        \item telefono
        \begin{itemize}
            \item Definicion
            \item Definición: número de teléfono del usuario.
            \item Dominio: conjunto de 150 caracteres.
            \item Tipo: atributo simple.
            \item Opcional: no
            \item Ejemplo: 649543987
        \end{itemize} 
            
        \item email
        \begin{itemize}
            \item Definición: dirección de correo del usuario.
            \item Dominio: conjunto de 250 caracteres.
            \item Tipo: atributo simple.
            \item Opcional: no
            \item Ejemplo: moyano@gmail.com
        \end{itemize}

        \item password
        \begin{itemize}
            \item Definición: contraseña del usuario.
            \item Dominio: conjunto de 250 caracteres encriptados.
            \item Tipo: atributo simple.
            \item Opcional: no
            \item Ejemplo: 
        \end{itemize}

        \item fecha_alta
        \begin{itemize}
            \item Definición: Fecha en que se crea la cuenta del usuario.
            \item Dominio: conjunto de 250 caracteres encriptados.
            \item Tipo: atributo simple.
            \item Opcional: no
            \item Ejemplo: 24/11/2025
        \end{itemize}

    \end{itemize}   

    \item Diagrama (Figura \ref{fig:E-Usuario}):

    \begin{figure}[H]
        \centering
        %\includegraphics[scale=0.8]{img/diagramas/EER/E-Usuario.png}
        \caption{Entidad Usuario}
        \label{fig:E-Usuario}
    \end{figure}

    \item Ejemplo de entidad (Tabla \ref{table:T-Usuario}):

    \begin{table}[H]
    \centering
        \begin{tabular}{ |p{6cm}||p{6cm}|  }
             \hline
                \multicolumn{2}{|c|}{\textbf{Usuario}} \\
             \hline
                 \textbf{Atributo} & \textbf{Valor} \\
             \hline
                 id & 1 \\
             \hline
                 nombre & Ciencias \\
            \hline
                apellidos & Galisteo \\
            \hline
                teléfono & 695586666 \\
             \hline
                 email &  \\
            \hline
                 password &  \\
             \hline
        \end{tabular}
        \caption{Ejemplo de la entidad \textit{Usuario}}
        \label{table:T-Usuario}
    \end{table}
\end{itemize}



\section{Tipos de interrelación} \label{sec:relaciones}
En esta sección se identificarán y describirán las interrelaciones entre los tipos de entidad descritos en la sección 7.2.

Las interrelaciones pueden ser de tipo débil o fuerte.

\begin{itemize}
    \item Un tipo de Interrelación Fuerte es aquella que representa la relación existente entre dos tipos de entidad fuertes.

    \item Un tipo de Interrelación Débil es aquella que representa la relación entre un tipo de entidad débil y otro fuerte o entre dos tipos de entidad débiles.
\end{itemize}

De acuerdo con la notación del modelo E-R, una interrelación se representa mediante un rombo del que parten flechas hacia los tipos de entidad que forman parte de la relación. Cada tipo de entidad interviene en la interrelación con una determinada cardinalidad, que indica el número mínimo y máximo de instancias de cada tipo de entidad que pueden participar en la interrelación. Se representa por dos valores entre paréntesis (mínimo y máximo). Las posibles cardinalidades son: (0,1), (1,1),(0,n),(1,n),(m,n). Estas cardinalidades determinan el tipo de interrelación que se está definiendo, que puede ser:

\begin{itemize}
    \item \textbf{1:1} Uno a uno. Cuando los dos tipos de entidad participan con una cardinalidad máxima de 1.
    \item \textbf{1:N o N:1} Uno a muchos, o muchos a uno. Cuando uno de los tipos de entidad participa con una cardinalidad máxima de 1 y el otro con una cardinalidad máxima de n.
    \item \textbf{N:N} Muchos a muchos. Cuando ambos tipos de entidad participan una cardinalidad máxima de n.
\end{itemize}

Para cada una de las interrelaciones que se han identificado en la definición del problema, se indicará la siguiente información:

\begin{itemize}
    \item \textbf{Nombre}. Nombre del tipo de interrelación.
    \item \textbf{Descripción}. Definición del tipo de interrelación y de los tipos de entidad que participan en ella.
    \item \textbf{Tipo}. Indicación de si se trata de un tipo de interrelación débil o fuerte identificando los tipos de debilidad en su caso.
    \item \textbf{Cardinalidad}. Indicación de la cardinalidad del tipo de interrelación y cardinalidades mínima y máxima de los tipos de entidad intervinientes.
    \item \textbf{Atributos}. Indicación del número y descripción de los atributos del tipo de interrelación, en su caso.
    \item \textbf{Diagrama}. Representación gráfica del tipo de interrelación, de acuerdo con la notación E-R.
    \item \textbf{Ejemplo}. Valores de muestra.
\end{itemize}

Se han identificado las interrelaciones que se indican a continuación y que se describirán en las siguientes subsecciones:

\begin{itemize}
    \item Tipo de Interrelación: Usuario - ...
\end{itemize}

\begin{landscape}
\section{Diagrama del Modelo Entidad-Interrelación}\label{sec:diagrama-E-R}
El diagrama completo se muestra en la figura \ref{fig:EER_v5}
\begin{figure}[H]
    \centering
    %\includegraphics[scale=0.35]{img/diagramas/EER/EER_v5.png}
    \caption{Diagrama del modelo Entidad - Interrelación}
    \label{fig:EER_v5}
\end{figure}
\end{landscape}

\chapter{Análisis funcional} \label{cap:analisis}

\section{Introducción}

Este capítulo identificará a los tipos de actores que podrán interactuar con el sistema informático que se desea desarrollar. A continuación, se utilizarán los casos de uso y los diagramas de secuencia para describir las acciones que podrán realizar los diferentes actores.

\section{Actores}

Un actor es una representación de una persona, proceso o entidad externa que interactúa con el sistema. Se van a considerar los siguientes tipos de actores:
   \begin{itemize}
        \item \textbf{Administrador}
        \begin{itemize}
            \item Este tipo de usuario estará registrado en el sistema y tendrá un control completo de la aplicación.
            \item Tendrá competencias exclusivas para la gestión del catálogo de productos, categorías, planes, contratos, usuarios y órdenes de logística.
        \end{itemize}

        \item \textbf{Cliente}
        \begin{itemize}
            \item Usuario registrado que utiliza la plataforma para alquilar material de gimnasio.
            \item Podrá gestionar su cuenta, consultar el catálogo de productos, añadir productos al carrito, contratar planes, gestionar sus direcciones, contratos, suscripciones y pagos.
        \end{itemize}

        \item \textbf{Usuario público}
        \begin{itemize}
            \item Usuario no autenticado que accede a la página web sin necesidad de registro.
            \item Podrá consultar la página inicial y una vista general del catálogo antes de registrarse o iniciar sesión.
        \end{itemize}

        \item \textbf{Pasarela de pago}
        \begin{itemize}
            \item Sistema externo encargado de autorizar y capturar los pagos realizados por el cliente.
            \item Permite confirmar o rechazar las operaciones de pago asociadas a los contratos y suscripciones.
        \end{itemize}

     \end{itemize}


\section{Casos de uso}

  Los casos de uso describen las acciones que pueden desarrollar los actores del sistema. Se ha identificado los siguientes casos de uso principales, que son descritos en las secciones que se indican:
\begin{itemize}
    \item \textbf{CU-0. Diagrama de contexto} (Sección \ref{sec:CU-0}).  
    \item \textbf{CU-1. Gestionar cuenta de usuario} (Sección \ref{sec:CU-1}). 
    \item \textbf{CU-2. Consultar catálogo de productos} (Sección \ref{sec:CU-2}).
    \item \textbf{CU-3. Gestionar carrito y contratos} (Sección \ref{sec:CU-3}).
    \item \textbf{CU-4. Gestionar suscripción} (Sección \ref{sec:CU-4}).
    \item \textbf{CU-5. Gestionar direcciones y envíos} (Sección \ref{sec:CU-5}).
    \item \textbf{CU-6. Gestionar pagos y facturas} (Sección \ref{sec:CU-6}).
    \item \textbf{CU-7. Consultar panel del cliente} (Sección \ref{sec:CU-7}).
    \item \textbf{CU-8. Administrar la plataforma} (Sección \ref{sec:CU-8}).
\end{itemize}

\subsection{Caso de Uso 0. Diagrama de Contexto}\label{sec:CU-0}

  El diagrama de contexto engloba los casos de uso principales que componen el sistema. Véanse la Figura \ref{fig:Diagrama-Contexto} y la Tabla \ref{tab:CU-0}.

%Diagrama de contexto
\begin{figure}[H]
        \centering
        %\includegraphics[scale=0.6]{img/diagramas/Funcional/CU-0.png}
        \caption{Diagrama de Contexto}\label{fig:Diagrama-Contexto}
\end{figure}


\begin{table}[H]
\caption{CU-0. Diagrama de contexto}\label{tab:CU-0}
\begin{center}
    \begin{tabular}{|l|p{12cm}|}
    \hline
    \multicolumn{2}{|c|}{Caso de uso 0 - Diagrama de contexto} \\ 
    \hline \hline
    Actores                 &   Usuario público, Cliente, Administrador, Pasarela de pago \\ \hline
    Descripción             &   Visión general de las acciones principales del sistema de alquiler de material de gimnasio. \\  \hline
    Precondiciones          &   El usuario debe acceder a la página inicial de la aplicación web mediante un navegador compatible.   \\  \hline
    Casos de uso            &   CU-1. Gestionar cuenta de usuario. \\  
    &   CU-2. Consultar catálogo de productos. \\
    &   CU-3. Gestionar carrito y contratos. \\
    &   CU-4. Gestionar suscripción. \\
    &   CU-5. Gestionar direcciones y envíos. \\
    &   CU-6. Gestionar pagos y facturas. \\
    &   CU-7. Consultar panel del cliente. \\
    &   CU-8. Administrar la plataforma. \\ \hline
    Flujo principal     &  1a. El usuario público accede a la aplicación sin necesidad de iniciar sesión y puede consultar la página inicial y el catálogo básico de productos. \\
        & 1b. El cliente o el administrador se identifican en el sistema para acceder a las funcionalidades asociadas a su rol (gestión de cuenta, alquiler de productos, administración del catálogo, etc.).
    \\ \hline
    Flujo alternativo    & 1. La aplicación no se carga correctamente o se produce un error en el servidor. \\
      & 2. Se informa del error al usuario mediante un mensaje adecuado. \\
      & 3. Se indica al usuario que puede intentar cargar de nuevo la aplicación o volver a intentarlo más tarde.
    \\  \hline
    \end{tabular}
    \end{center}
    \end{table}

\newpage

\subsection{CU-1. Gestionar cuenta de usuario}\label{sec:CU-1}

El caso de uso \textit{CU-1.  Gestionar cuenta de usuario} describe las acciones que puede realizar  (véanse la Tabla \ref{tab:CU-1} y la Figura \ref{fig:Diagrama-Caso de uso 1.}.). Este caso de uso está compuesto por los siguientes sub-casos de uso que se describen en las tablas que se indican:
\begin{itemize}
    \item CU-1.1. Registrarse en la plataforma.
    \item CU-1.2. Iniciar sesión.
    \item CU-1.3. Cerrar sesión.
    \item CU-1.4. Recuperar contraseña.
    \item CU-1.5. Editar perfil de usuario.
\end{itemize}
\begin{figure}[H]
\centering
%\includegraphics[scale=0.75]{img/diagramas/Funcional/CU-1.png}
\caption{CU-1.}\label{fig:Diagrama-Caso de uso 1.}
\end{figure}


\begin{table}[H]
\caption{CU-1. Gestionar cuenta de usuario}\label{tab:CU-1}
\begin{center}
    \begin{tabular}{|l|p{12cm}|}
    \hline
    \multicolumn{2}{|c|}{Caso de uso 1 - Gestionar cuenta de usuario} \\ 
    \hline \hline
    Actores                 &   Usuario público, Cliente, Administrador \\ \hline
    Descripción             &   Permite registrar una nueva cuenta, iniciar y cerrar sesión, recuperar la contraseña y editar los datos básicos del perfil de usuario. \\  \hline
    Precondiciones          &   El usuario debe haber accedido a la página principal de la aplicación. Para editar el perfil o cerrar sesión, el usuario debe estar autenticado.   \\  \hline
    Casos de uso            &   CU-1.1. Registrarse en la plataforma.  \\  
                            &   CU-1.2. Iniciar sesión.  \\
                            &   CU-1.3. Cerrar sesión.  \\
                            &   CU-1.4. Recuperar contraseña.  \\
                            &   CU-1.5. Editar perfil de usuario.  \\ \hline

    Flujo principal     &   1. El usuario selecciona una opción relacionada con la gestión de su cuenta (registro, inicio de sesión, recuperación de contraseña o edición de perfil). \\ 
                        &   2. El sistema muestra el formulario correspondiente. \\
                        &   3. El usuario introduce la información requerida (datos personales, credenciales, correo de recuperación, etc.). \\
                        &   4. El sistema valida la información introducida. \\
                        &   5. Si la información es correcta, el sistema realiza la acción solicitada (creación de cuenta, inicio/cierre de sesión, actualización de datos o envío de correo de recuperación). \\ \hline

    Flujo alternativo    &  1. Se introduce información incompleta o no válida en el formulario. \\ 
                            &  2. El sistema muestra mensajes de error indicando los campos que deben corregirse.  \\  
                            &  3. El usuario modifica los datos y vuelve a enviar el formulario.  \\  
                            &  4. Si el problema persiste (por ejemplo, correo ya registrado o credenciales incorrectas), se informa al usuario y se le permite volver a intentarlo o cancelar la operación. \\  \hline    
            \end{tabular}
        \end{center}
    \end{table}

\newpage

\subsection{CU-2. Consultar catálogo de productos}\label{sec:CU-2}

El caso de uso \textit{CU-2. Consultar catálogo de productos} describe las acciones que puede realizar un usuario (público, cliente o administrador) para visualizar el catálogo, aplicar filtros y consultar la ficha detallada de un producto. Véanse la Tabla \ref{tab:CU-2} y la Figura \ref{fig:Diagrama-CU2}. Este caso de uso está compuesto por los siguientes sub-casos de uso:
\begin{itemize}
    \item CU-2.1. Listar productos.
    \item CU-2.2. Buscar productos mediante filtros.
    \item CU-2.3. Consultar ficha de producto.
\end{itemize}

\begin{figure}[H]
\centering
%\includegraphics[scale=0.75]{img/diagramas/Funcional/CU-2.png}
\caption{CU-2. Consultar catálogo de productos}\label{fig:Diagrama-CU2}   
\end{figure}

\begin{table}[H]
\caption{CU-2. Consultar catálogo de productos}\label{tab:CU-2}
\begin{center}
    \begin{tabular}{|l|p{12cm}|}
    \hline
    \multicolumn{2}{|c|}{Caso de uso 2 - Consultar catálogo de productos} \\ 
    \hline \hline
    Actores                 &   Usuario público, Cliente, Administrador \\ \hline

    Descripción             &   Permite visualizar el catálogo completo de productos disponibles, aplicar filtros de búsqueda (categoría, precio, espacio en m², instalación), y acceder a la ficha de un producto. \\  \hline

    Precondiciones          &   El usuario debe haber accedido a la página principal de la aplicación o a la sección del catálogo. \\  \hline

    Casos de uso            &   CU-2.1. Listar productos.  \\  
    &   CU-2.2. Buscar productos mediante filtros.  \\ 
    &   CU-2.3. Consultar ficha de producto. \\ \hline

    Flujo principal     &   1. El usuario accede al catálogo de productos. \\ 
    &   2. El sistema muestra la lista de productos disponibles. \\ 
    &   3. El usuario puede aplicar filtros (categoría, precio, m², instalación). \\ 
    &   4. El sistema actualiza la lista en función de los filtros. \\ 
    &   5. El usuario selecciona un producto para ver su ficha detallada. \\ 
    &   6. El sistema muestra la información completa del producto (fotos, descripción, m², tarifas, disponibilidad). \\ \hline
    Flujo alternativo    &  1. No existen productos que cumplan los filtros seleccionados.  \\ 
    &  2. El sistema muestra un mensaje indicando que no hay resultados. \\ 
    &  3. Permite modificar los filtros o limpiar la búsqueda. \\ \hline    
            \end{tabular}
        \end{center}
    \end{table}

\subsection{CU-3. Gestionar carrito y contratos}\label{sec:CU-3}

El caso de uso \textit{CU-3. Gestionar carrito y contratos} describe las acciones que puede realizar un cliente para añadir productos al carrito, configurar la duración del alquiler, calcular el precio final y generar un contrato. Véanse la Tabla \ref{tab:CU-3} y la Figura \ref{fig:Diagrama-CU3}. Este caso de uso está compuesto por los siguientes sub-casos de uso:
\begin{itemize}
    \item CU-3.1. Añadir producto al carrito.
    \item CU-3.2. Modificar o eliminar productos del carrito.
    \item CU-3.3. Calcular precio del contrato.
    \item CU-3.4. Generar contrato.
\end{itemize}

\begin{figure}[H]
\centering
%\includegraphics[scale=0.75]{img/diagramas/Funcional/CU-3.png}
\caption{CU-3. Gestionar carrito y contratos}\label{fig:Diagrama-CU3}   
\end{figure}

\begin{table}[H]
\caption{CU-3. Gestionar carrito y contratos}\label{tab:CU-3}
\begin{center}
    \begin{tabular}{|l|p{12cm}|}
    \hline
    \multicolumn{2}{|c|}{Caso de uso 3 - Gestionar carrito y contratos} \\ 
    \hline \hline

    Actores                 &   Cliente \\ \hline

    Descripción             &   Permite añadir productos al carrito, configurar la duración del alquiler, calcular el precio final y generar un contrato con dirección y fechas seleccionadas. \\  \hline

    Precondiciones          &   El usuario debe estar autenticado para generar un contrato; para usar el carrito puede estar o no registrado. \\  \hline

    Casos de uso            &   CU-3.1. Añadir producto al carrito.  \\  
                            &   CU-3.2. Modificar o eliminar productos del carrito.  \\ 
                            &   CU-3.3. Calcular precio del contrato. \\ 
                            &   CU-3.4. Generar contrato. \\ \hline

    Flujo principal     &   1. El usuario selecciona un producto del catálogo. \\ 
                        &   2. Añade el producto al carrito indicando su duración (días/semanas/meses). \\ 
                        &   3. El sistema actualiza el carrito y calcula el precio del alquiler. \\ 
                        &   4. El usuario selecciona la dirección de entrega y recogida. \\ 
                        &   5. El sistema genera un resumen del contrato. \\ 
                        &   6. El usuario confirma la operación y se genera el contrato pendiente de pago. \\ \hline

    Flujo alternativo    &  1. Un producto ya no está disponible.  \\ 
                            &  2. El sistema avisa al usuario y lo elimina del carrito. \\ 
                            &  3. El usuario puede modificar la duración o seleccionar otro producto.  \\ \hline    
            \end{tabular}
        \end{center}
    \end{table}
\newpage

\subsection{CU-4. Gestionar suscripción}\label{sec:CU-4}

El caso de uso \textit{CU-4. Gestionar suscripción} describe las acciones que realiza el cliente para contratar, gestionar y cancelar una suscripción mensual a un plan de servicio. Véanse la Tabla \ref{tab:CU-4} y la Figura \ref{fig:Diagrama-CU4}. Este caso de uso está compuesto por los siguientes sub-casos de uso:
\begin{itemize}
    \item CU-4.1. Contratar suscripción.
    \item CU-4.2. Consultar estado de la suscripción.
    \item CU-4.3. Pausar o cancelar la suscripción.
\end{itemize}

\begin{figure}[H]
\centering
%\includegraphics[scale=0.75]{img/diagramas/Funcional/CU-4.png}
\caption{CU-4. Gestionar suscripción}\label{fig:Diagrama-CU4}   
\end{figure}

\begin{table}[H]
\caption{CU-4. Gestionar suscripción}\label{tab:CU-4}
\begin{center}
    \begin{tabular}{|l|p{12cm}|}
    \hline
    \multicolumn{2}{|c|}{Caso de uso 4 - Gestionar suscripción} \\ 
    \hline \hline

    Actores                 &   Cliente \\ \hline

    Descripción             &   Permite contratar un plan de suscripción mensual, consultar su estado actual (activa, pausada, cancelada) y gestionarla, incluyendo pausarla temporalmente o cancelarla de forma permanente. \\  \hline

    Precondiciones          &   El usuario debe estar autenticado. Debe existir al menos un plan de suscripción disponible. \\  \hline

    Casos de uso            &   CU-4.1. Contratar suscripción.  \\  
                            &   CU-4.2. Consultar estado de la suscripción.  \\ 
                            &   CU-4.3. Pausar o cancelar la suscripción. \\ \hline

    Flujo principal     &   1. El usuario accede a la sección de planes de suscripción. \\ 
                        &   2. Selecciona uno de los planes disponibles. \\ 
                        &   3. El sistema muestra la información completa del plan (precio, características, período de facturación). \\ 
                        &   4. El usuario confirma la contratación del plan. \\ 
                        &   5. El sistema activa la suscripción y registra la fecha de inicio y el método de pago. \\ \hline

    Flujo alternativo    &  1. El usuario intenta contratar un plan pero su método de pago falla o no está disponible.  \\ 
                            &  2. El sistema muestra un mensaje de error indicando la incidencia. \\ 
                            &  3. El usuario puede actualizar su método de pago o intentar de nuevo la contratación. \\ \hline    
            \end{tabular}
        \end{center}
    \end{table}

    \subsection{CU-5. Gestionar direcciones y envíos}\label{sec:CU-5}

El caso de uso \textit{CU-5. Gestionar direcciones y envíos} describe las acciones que puede realizar el cliente para gestionar sus direcciones de entrega/recogida y las órdenes de logística asociadas a los contratos. Véanse la Tabla \ref{tab:CU-5} y la Figura \ref{fig:Diagrama-CU5}. Este caso de uso está compuesto por los siguientes sub-casos de uso:
\begin{itemize}
    \item CU-5.1. Gestionar direcciones del cliente (crear, modificar, eliminar).
    \item CU-5.2. Seleccionar dirección de entrega y recogida en un contrato.
    \item CU-5.3. Gestionar órdenes de logística (consulta de estado).
\end{itemize}

\begin{figure}[H]
\centering
%\includegraphics[scale=0.75]{img/diagramas/Funcional/CU-5.png}
\caption{CU-5. Gestionar direcciones y envíos}\label{fig:Diagrama-CU5}   
\end{figure}

\begin{table}[H]
\caption{CU-5. Gestionar direcciones y envíos}\label{tab:CU-5}
\begin{center}
    \begin{tabular}{|l|p{12cm}|}
    \hline
    \multicolumn{2}{|c|}{Caso de uso 5 - Gestionar direcciones y envíos} \\ 
    \hline \hline

    Actores                 &   Cliente, Administrador \\ \hline

    Descripción             &   Permite al cliente gestionar sus direcciones de entrega y recogida, seleccionar una dirección durante la creación de un contrato y consultar las órdenes de logística generadas para los envíos y recogidas. \\  \hline

    Precondiciones          &   El usuario debe estar autenticado para gestionar sus direcciones. Debe existir al menos una dirección para poder asociarla a un contrato. \\  \hline

    Casos de uso            &   CU-5.1. Gestionar direcciones del cliente. \\  
                            &   CU-5.2. Seleccionar dirección de entrega y recogida. \\ 
                            &   CU-5.3. Gestionar órdenes de logística. \\ \hline

    Flujo principal         &   1. El usuario accede a la sección de direcciones en su cuenta. \\ 
                            &   2. El sistema muestra el listado de direcciones registradas. \\ 
                            &   3. El usuario puede crear una nueva dirección o modificar/eliminar una existente. \\ 
                            &   4. Durante la creación de un contrato, el usuario selecciona la dirección de entrega y, en su caso, de recogida. \\ 
                            &   5. El sistema registra la dirección en el contrato y genera la orden de logística correspondiente. \\ \hline

    Flujo alternativo       &   1. El usuario introduce datos de dirección incompletos o no válidos. \\ 
                            &   2. El sistema muestra mensajes de error indicando los campos a corregir. \\ 
                            &   3. Si no existen direcciones válidas, el sistema impide continuar con la creación del contrato hasta que se registre al menos una dirección correcta. \\ \hline    
            \end{tabular}
        \end{center}
    \end{table}

    \subsection{CU-6. Gestionar pagos y facturas}\label{sec:CU-6}

El caso de uso \textit{CU-6. Gestionar pagos y facturas} describe las acciones que realiza el cliente para efectuar el pago de contratos y suscripciones, así como la generación y consulta de las facturas asociadas. Véanse la Tabla \ref{tab:CU-6} y la Figura \ref{fig:Diagrama-CU6}. Este caso de uso está compuesto por los siguientes sub-casos de uso:
\begin{itemize}
    \item CU-6.1. Realizar pago de contrato o suscripción.
    \item CU-6.2. Validar el pago con la pasarela.
    \item CU-6.3. Generar y consultar factura.
\end{itemize}

\begin{figure}[H]
\centering
%\includegraphics[scale=0.75]{img/diagramas/Funcional/CU-6.png}
\caption{CU-6. Gestionar pagos y facturas}\label{fig:Diagrama-CU6}   
\end{figure}

\begin{table}[H]
\caption{CU-6. Gestionar pagos y facturas}\label{tab:CU-6}
\begin{center}
    \begin{tabular}{|l|p{12cm}|}
    \hline
    \multicolumn{2}{|c|}{Caso de uso 6 - Gestionar pagos y facturas} \\ 
    \hline \hline

    Actores                 &   Cliente, Pasarela de pago, Administrador \\ \hline

    Descripción             &   Permite realizar el pago de contratos y suscripciones a través de una pasarela de pago, validar la transacción y generar la factura asociada, que quedará disponible para su consulta y descarga. \\  \hline

    Precondiciones          &   Debe existir un contrato o una suscripción pendiente de pago. La pasarela de pago debe estar disponible. El usuario debe estar autenticado. \\  \hline

    Casos de uso            &   CU-6.1. Realizar pago de contrato o suscripción. \\  
                            &   CU-6.2. Validar el pago con la pasarela. \\ 
                            &   CU-6.3. Generar y consultar factura. \\ \hline

    Flujo principal         &   1. El usuario accede al resumen del contrato o suscripción pendiente de pago. \\ 
                            &   2. Selecciona el método de pago y confirma la operación. \\ 
                            &   3. El sistema envía la información a la pasarela de pago. \\ 
                            &   4. La pasarela de pago procesa la transacción y devuelve el resultado (éxito o fallo). \\ 
                            &   5. Si el pago es aceptado, el sistema marca el contrato o la suscripción como activa y genera la factura en formato PDF. \\ 
                            &   6. El usuario puede consultar y descargar la factura desde su panel. \\ \hline

    Flujo alternativo       &   1. La pasarela de pago rechaza la operación o se produce un error en la transacción. \\ 
                            &   2. El sistema informa al usuario del fallo en el pago. \\ 
                            &   3. El contrato o la suscripción permanecen en estado pendiente y no se activan. \\ 
                            &   4. El usuario puede intentar de nuevo el pago o cancelar la operación. \\ \hline    
            \end{tabular}
        \end{center}
    \end{table}

    \subsection{CU-7. Consultar panel del cliente}\label{sec:CU-7}

El caso de uso \textit{CU-7. Consultar panel del cliente} describe las acciones que realiza el cliente para visualizar un resumen de sus contratos, suscripciones, direcciones y facturas dentro de la plataforma. Véanse la Tabla \ref{tab:CU-7} y la Figura \ref{fig:Diagrama-CU7}. Este caso de uso está compuesto por los siguientes sub-casos de uso:
\begin{itemize}
    \item CU-7.1. Consultar contratos.
    \item CU-7.2. Consultar suscripciones.
    \item CU-7.3. Consultar y descargar facturas.
\end{itemize}

\begin{figure}[H]
\centering
%\includegraphics[scale=0.75]{img/diagramas/Funcional/CU-7.png}
\caption{CU-7. Consultar panel del cliente}\label{fig:Diagrama-CU7}   
\end{figure}

\begin{table}[H]
\caption{CU-7. Consultar panel del cliente}\label{tab:CU-7}
\begin{center}
    \begin{tabular}{|l|p{12cm}|}
    \hline
    \multicolumn{2}{|c|}{Caso de uso 7 - Consultar panel del cliente} \\ 
    \hline \hline

    Actores                 &   Cliente \\ \hline

    Descripción             &   Permite al cliente consultar de forma centralizada la información relacionada con sus contratos, suscripciones, direcciones y facturas, así como acceder al detalle de cada elemento. \\  \hline

    Precondiciones          &   El usuario debe estar autenticado en la plataforma. \\  \hline

    Casos de uso            &   CU-7.1. Consultar contratos. \\  
                            &   CU-7.2. Consultar suscripciones. \\ 
                            &   CU-7.3. Consultar y descargar facturas. \\ \hline

    Flujo principal         &   1. El usuario accede al panel de cliente desde el menú principal. \\ 
                            &   2. El sistema muestra un resumen de sus contratos, suscripciones y facturas. \\ 
                            &   3. El usuario selecciona el apartado que desea consultar (contratos, suscripciones o facturas). \\ 
                            &   4. El sistema muestra el listado detallado del apartado seleccionado. \\ 
                            &   5. El usuario puede acceder al detalle de un elemento concreto o descargar la factura correspondiente, en su caso. \\ \hline

    Flujo alternativo       &   1. El usuario no dispone de contratos, suscripciones o facturas en alguno de los apartados. \\ 
                            &   2. El sistema muestra un mensaje indicando que no existen registros disponibles en esa sección. \\ 
                            &   3. Se ofrecen enlaces o acciones para guiar al usuario (por ejemplo, acceder al catálogo o contratar un plan). \\ \hline    
            \end{tabular}
        \end{center}
    \end{table}

    \subsection{CU-8. Administrar la plataforma}\label{sec:CU-8}

El caso de uso \textit{CU-8. Administrar la plataforma} describe las acciones que realiza el administrador para gestionar los elementos principales del sistema: productos, categorías, planes, usuarios, contratos y órdenes de logística. Véanse la Tabla \ref{tab:CU-8} y la Figura \ref{fig:Diagrama-CU8}. Este caso de uso está compuesto por los siguientes sub-casos de uso:
\begin{itemize}
    \item CU-8.1. Gestionar productos.
    \item CU-8.2. Gestionar categorías y planes.
    \item CU-8.3. Gestionar usuarios y roles.
    \item CU-8.4. Gestionar contratos y órdenes de logística.
\end{itemize}

\begin{figure}[H]
\centering
%\includegraphics[scale=0.75]{img/diagramas/Funcional/CU-8.png}
\caption{CU-8. Administrar la plataforma}\label{fig:Diagrama-CU8}   
\end{figure}

\begin{table}[H]
\caption{CU-8. Administrar la plataforma}\label{tab:CU-8}
\begin{center}
    \begin{tabular}{|l|p{12cm}|}
    \hline
    \multicolumn{2}{|c|}{Caso de uso 8 - Administrar la plataforma} \\ 
    \hline \hline

    Actores                 &   Administrador \\ \hline

    Descripción             &   Permite al administrador gestionar el catálogo de productos y categorías, los planes de suscripción, los usuarios y sus roles, así como revisar y mantener los contratos y las órdenes de logística. \\  \hline

    Precondiciones          &   El usuario debe estar autenticado y disponer de permisos de administrador. \\  \hline

    Casos de uso            &   CU-8.1. Gestionar productos. \\  
                            &   CU-8.2. Gestionar categorías y planes. \\ 
                            &   CU-8.3. Gestionar usuarios y roles. \\
                            &   CU-8.4. Gestionar contratos y órdenes de logística. \\ \hline

    Flujo principal         &   1. El administrador accede al panel de administración de la plataforma. \\ 
                            &   2. El sistema muestra las opciones de gestión disponibles (productos, categorías, planes, usuarios, contratos, logística). \\ 
                            &   3. El administrador selecciona el módulo que desea gestionar. \\ 
                            &   4. El sistema muestra el listado de registros asociados (por ejemplo, productos o usuarios). \\ 
                            &   5. El administrador puede crear, modificar o eliminar registros, según las necesidades de la plataforma. \\ 
                            &   6. El sistema valida los datos introducidos y actualiza la información en la base de datos. \\ \hline

    Flujo alternativo       &   1. El administrador introduce datos incompletos o no válidos al crear o modificar un registro. \\ 
                            &   2. El sistema muestra mensajes de error indicando los campos que deben corregirse. \\ 
                            &   3. Si la eliminación de un registro afecta a otros elementos (por ejemplo, productos asociados a contratos), el sistema puede impedir la operación o solicitar confirmación adicional. \\ \hline    
            \end{tabular}
        \end{center}
    \end{table}


\section{Validación de casos de uso}
La tabla \ref{tab:ValidacionCU} permite comprobar que los casos de uso cubren todos los requisitos funcionales de la aplicación web propuestos en la Sección \ref{sec:requisitos-funcionales}.

\begin{table}[H]
    \centering
    \caption{Tabla de validación casos de uso} \label{tab:ValidacionCU}
    \begin{tabular}{|l|l|}
        \hline
            \textbf{Requisito funcional} & \textbf{Caso de uso} \\ 
        \hline 
        \hline         
            \textbf{RF-01}  & CU-1 \\ % Gestionar cuenta de usuario
        \hline
            \textbf{RF-02}  & CU-1 \\ % Gestionar cuenta de usuario
        \hline
            \textbf{RF-03}  & CU-1, CU-8 \\ % Roles y administración
        \hline
            \textbf{RF-04}  & CU-2 \\ % Consultar catálogo
        \hline
            \textbf{RF-05}  & CU-2 \\ % Ficha de producto
        \hline
            \textbf{RF-06}  & CU-3 \\ % Carrito
        \hline
            \textbf{RF-07}  & CU-3 \\ % Cálculo de precio
        \hline
            \textbf{RF-08}  & CU-3 \\ % Checkout / contrato
        \hline
            \textbf{RF-09}  & CU-4 \\ % Gestionar suscripción
        \hline
            \textbf{RF-10}  & CU-4 \\ % Validación de suscripción
        \hline
            \textbf{RF-11}  & CU-4 \\ % Alta/pausa/cancelación
        \hline
            \textbf{RF-12}  & CU-5 \\ % CRUD direcciones
        \hline
            \textbf{RF-13}  & CU-3, CU-5 \\ % Seleccionar dirección en contrato
        \hline
            \textbf{RF-14}  & CU-5 \\ % Orden logística
        \hline
            \textbf{RF-15}  & CU-6 \\ % Pasarela de pago
        \hline
            \textbf{RF-16}  & CU-3, CU-6 \\ % Contrato solo si pago OK
        \hline
            \textbf{RF-17}  & CU-6 \\ % Factura PDF
        \hline
            \textbf{RF-18}  & CU-7 \\ % Panel cliente
        \hline
            \textbf{RF-19}  & CU-8 \\ % Backoffice admin
        \hline
    \end{tabular}
\end{table}


\newpage 

\section{Diagrama de secuencia}

El diagrama de secuencia es una representación gráfica que pretende dar una visión de las acciones que se realizarán durante la ejecución de alguna operación en el sistema. A continuación, se muestran los diagramas de secuencias para la creación (Figura  \ref{fig:Diagrama de secuencia crear}), búsqueda (Figura \ref{fig:Diagrama de secuencia buscar}), modificación (Figura \ref{fig:Diagrama de secuencia modificar}) y borrado (Figura \ref{fig:Diagrama de secuencia borrar}) de instancias genéricas () que se corresponderán con...

\begin{figure}[H]
    \centering
%        \includegraphics[scale=0.55]{img/diagramas/Secuencia/SEC-1.png}
    \caption{Diagrama de secuencia de crear}
    \label{fig:Diagrama de secuencia crear}
\end{figure}

\begin{figure}[H]
    \centering
%        \includegraphics[scale=0.55]{img/diagramas/Secuencia/SEC-1.png}
    \caption{Diagrama de secuencia de borrar}
    \label{fig:Diagrama de secuencia borrar}
\end{figure}

\begin{figure}[H]
    \centering
%        \includegraphics[scale=0.55]{img/diagramas/Secuencia/SEC-1.png}
    \caption{Diagrama de secuencia de buscar}
    \label{fig:Diagrama de secuencia buscar}
\end{figure}

\begin{figure}[H]
    \centering
%        \includegraphics[scale=0.55]{img/diagramas/Secuencia/SEC-1.png}
    \caption{Diagrama de secuencia de modificar}
    \label{fig:Diagrama de secuencia modificar}
\end{figure}

\part{Diseño}
\chapter{Diseño de datos} \label{cap:diseño_datos}

\section{Introducción}
En este capítulo se definirán las estructuras de datos que conforman el sistema a partir de los elementos identificados durante el análisis de datos del Capítulo 7 (tipos de entidad y tipos de interrelación). Para ello, se llevarán a cabo los siguientes pasos:

\begin{itemize}
    \item Obtención del modelo relacional. Definición de la estructuras (tablas) del modelo de datos.
    \item Normalización del modelo. Refinamiento del modelo, para la eliminación de errores de integridad.
    \item Obtención del esquema relacional.
    \item Diagrama relacional.
\end{itemize}

 \section{Modelo Relacional}\label{sec:modelo-relacional}
A partir del Modelo Entidad - Interrelación descrito en el capítulo 7, se pueden obtener las tablas o relaciones del Modelo Relacional utilizando las reglas de transformación (RTECAR - véase el capítulo 5 de Bases de Datos. Desde Chen hasta Codd con ORACLE.\cite{reglasBD}). En concreto, se han aplicado las siguientes reglas de transformación:

\begin{itemize}
    \item RTECAR-1: Todos los tipos de entidad presentes en el esquema conceptual se transformarán en tablas o relaciones en el esquema relacional manteniendo el número y tipo de atributos, así como la característica de identificador de esos atributos.
    \item RTECAR-3.1: En las relaciones 1:N, la clave primaria de la entidad del lado 1 se convierte en una clave foránea en la tabla de la entidad del lado N.
\end{itemize}

Para cada tabla se mostrará la siguiente información:

\begin{itemize}
    \item \textbf{Descripción}. Se describirá el origen de la tabla, indicando los elementos del modelo Entidad-Interrelación desde los que se ha obtenido.
    \item \textbf{Nombre de la tabla}
    \item \textbf{Atributos}. Se describirán los atributos que componen la tabla, distinguiendo su rol en cada caso con la siguiente notación:
        \begin{itemize}
            \item Clave \underline{primaria}
            \item Clave ALTERNA, si existe
            \item Claves \textbf{foráneas}, si existen
            \item Resto de atributos
        \end{itemize}
    \item \textbf{Esquema relacional}. Definición formal de la tabla, de acuerdo con el Modelo Relacional.
\end{itemize}

\subsection{Tabla Usuario}
\begin{itemize}
    \item \textbf{Descripción}: la tabla \textit{Usuario} se obtiene a partir de la entidad \textit{Usuario}
    (modelo \texttt{User}). Almacena las credenciales y datos básicos necesarios para la autenticación y
    la identificación del usuario dentro del sistema. Además, se relaciona con la entidad \textit{Profesional/Trabajador}
    mediante el atributo \texttt{worker\_id}.

    \item \textbf{Nombre de la tabla}: Usuario (\texttt{users})

    \item \textbf{Atributos}:
    \begin{itemize}
        \item Clave primaria: \underline{id}
        \item Claves alternas: email (único)
        \item Claves foráneas: \textbf{worker\_id} $\rightarrow$ Trabajador(id) \ (si existe la tabla de trabajadores/profesionales)
        \item Resto de atributos: name, password, avatarUrl, email\_verified\_at, remember\_token, created\_at, updated\_at
    \end{itemize}

    \item \textbf{Esquema relacional}:
    Usuario(\textbf{\underline{id}}, name, \textit{email}, password, avatarUrl, \textbf{worker\_id}, email\_verified\_at, remember\_token, created\_at, updated\_at)
\end{itemize}


 \section{Normalización del modelo}\label{sec:normalizacion}
La normalización del modelo descrito en la sección anterior pretende detectar y corregir redundancias e inconsistencias en la información representada, para lo cual se aplicarán las medidas correctoras que garanticen que las tablas obtenidas cumplen las siguientes formas normalizadas \cite{reglasBD}.

\begin{itemize}
    \item La tabla Usuario cumple la FN1 al almacenar valores atómicos en todos sus atributos.
    \item Cumple FN2 y FN3 al depender todos los atributos no clave de forma completa y no transitiva de la clave primaria.
    \item Además, se encuentra en FNBC, ya que los determinantes funcionales relevantes son claves candidatas (id y email).
\end{itemize}

Todas las tablas definidas se encuentran en la Primera Forma Normal, puesto que en ninguna de ellas existen atributos múltiples.

\subsection{Tabla Usuario}
    \begin{itemize}
        \item \textbf{Claves candidatas}: id
        \item \textbf{Normalización}: La tabla Usuario presenta una dependencia funcional formada por la clave primaria y el resto de atributos. La tabla se encuentra en FNBC: el único determinante funcional es la clave primaria, por lo que la dependencia funcional con el resto de atributos es completa.
    \end{itemize}

\begin{figure}[H]
\centering
\includegraphics[scale=0.75]{img/diagramas/Datos/User.png}
\caption{Tabla Usuario en FNBC}\label{fig:Tabla Usuario en FNBC}   
\end{figure}


\subsection{Tabla Trabajador}
\begin{itemize}
    \item \textbf{Descripción}: la tabla \textit{Trabajador} se obtiene a partir de la entidad \textit{Trabajador}
    (modelo \texttt{Worker} de Laravel). Almacena la información laboral necesaria para la planificación de guardias,
    incluyendo rango/categoría, fechas de alta y baja, y la especialidad a la que pertenece.

    \item \textbf{Nombre de la tabla}: Trabajador (\texttt{worker})

    \item \textbf{Atributos}:
    \begin{itemize}
        \item Clave primaria: \underline{id}
        \item Claves alternas: ---
        \item Claves foráneas: \textbf{id\_speciality} $\rightarrow$ Especialidad(id)
        \item Resto de atributos: name, rank, registration\_date, discharge\_date
    \end{itemize}

    \item \textbf{Esquema relacional}:
    Trabajador(\textbf{\underline{id}}, name, rank, registration\_date, discharge\_date, \textbf{id\_speciality})
\end{itemize}

\subsection{Tabla Trabajador}
    \begin{itemize}
        \item \textbf{Claves candidatas}: id
        \item \textbf{Normalización}: La tabla Trabajador presenta una dependencia funcional formada por la clave primaria y el resto de atributos. La tabla se encuentra en FNBC: el único determinante funcional es la clave primaria, por lo que la dependencia funcional con el resto de atributos es completa.
    \end{itemize}

\begin{figure}[H]
\centering
\includegraphics[scale=0.75]{img/diagramas/Datos/trabajador.png}
\caption{Tabla Trabajador en FNBC}\label{fig:Tabla Trabajador en FNBC}   
\end{figure}

\subsection{Tabla Especialidad}
\begin{itemize}
    \item \textbf{Descripción}: la tabla \textit{Especialidad} se obtiene a partir de la entidad \textit{Especialidad}
    (modelo \texttt{Speciality} de Laravel). Almacena las secciones/especialidades del hospital a las que pertenecen
    los trabajadores y sobre las que se organizan las guardias.

    \item \textbf{Nombre de la tabla}: Especialidad (\texttt{speciality})

    \item \textbf{Atributos}:
    \begin{itemize}
        \item Clave primaria: \underline{id}
        \item Claves alternas: ---
        \item Claves foráneas: ---
        \item Resto de atributos: name, active
    \end{itemize}

    \item \textbf{Esquema relacional}:
    Especialidad(\textbf{\underline{id}}, name, active)
\end{itemize}
\subsection{Tabla Especialidad}
    \begin{itemize}
        \item \textbf{Claves candidatas}: id
        \item \textbf{Normalización}: La tabla Especialidad presenta una dependencia funcional formada por la clave primaria y el resto de atributos. La tabla se encuentra en FNBC: el único determinante funcional es la clave primaria, por lo que la dependencia funcional con el resto de atributos es completa.
    \end{itemize}

\begin{figure}[H]
\centering
\includegraphics[scale=0.75]{img/diagramas/Datos/especialidad.png}
\caption{Tabla Especialidad en FNBC}\label{fig:Tabla Especialidad en FNBC}   
\end{figure}

\subsection{Tabla Guardia}
\begin{itemize}
    \item \textbf{Descripción}: la tabla \textit{Guardia} se obtiene a partir de la entidad \textit{Guardia}
    (modelo \texttt{Duty} de Laravel). Representa cada guardia asignada en una fecha concreta, indicando su tipo,
    la especialidad asociada, el trabajador asignado y, cuando corresponda, el trabajador que actúa como jefe de guardia.

    \item \textbf{Nombre de la tabla}: Guardia (\texttt{duties})

    \item \textbf{Atributos}:
    \begin{itemize}
        \item Clave primaria: \underline{id}
        \item Claves alternas: ---
        \item Claves foráneas: \textbf{id\_speciality} $\rightarrow$ Especialidad(id), \textbf{id\_worker} $\rightarrow$ Trabajador(id), \textbf{id\_chief\_worker} $\rightarrow$ Trabajador(id)
        \item Resto de atributos: date, duty\_type
    \end{itemize}

    \item \textbf{Esquema relacional}:
    Guardia(\textbf{\underline{id}}, date, duty\_type, \textbf{id\_speciality}, \textbf{id\_worker}, \textbf{id\_chief\_worker})
\end{itemize}
\subsection{Tabla Guardia}
    \begin{itemize}
        \item \textbf{Claves candidatas}: id
        \item \textbf{Normalización}: La tabla Guardia presenta una dependencia funcional formada por la clave primaria y el resto de atributos. La tabla se encuentra en FNBC: el único determinante funcional es la clave primaria, por lo que la dependencia funcional con el resto de atributos es completa.
    \end{itemize}

\begin{figure}[H]
\centering
\includegraphics[scale=0.75]{img/diagramas/Datos/guardia.png}
\caption{Tabla Guardia en FNBC}\label{fig:Tabla Guardia en FNBC}   
\end{figure}


%%
 \section{Esquema relacional}\label{sec:esquema-relacional}

\begin{table}[H]
\begin{center}
    \begin{tabular}{|l|p{8cm}|}
    \hline
    Tabla                 &   Atributos
    \\ \hline

    Usuario             &   (\textbf{\underline{id}}, name, email, password, avatarUrl, \textbf{worker\_id}, email\_verified)
    \\ \hline

    Trabajador          &   (\textbf{\underline{id}}, name, rank, registration\_date, discharge\_date, \textbf{id\_speciality})
    \\ \hline

    Especialidad        &   (\textbf{\underline{id}}, name, active)
    \\ \hline

    Guardia             &   (\textbf{\underline{id}}, date, duty\_type, \textbf{id\_speciality}, \textbf{id\_worker}, \textbf{id\_chief\_worker})
    \\ \hline

    \end{tabular}
\end{center}
\end{table}



\section{Diagrama relacional}\label{sec:diagrama-relacional}
El diagrama relacional del modelo propuesto se muestra en la figura \ref{fig:Diagrama Relacional}.

\begin{landscape}
\begin{figure}[H]
\centering
%\includegraphics[scale=0.3]{img/diagramas/Datos/Diagrama-relacional.png}
\caption{Diagrama relacional}\label{fig:Diagrama Relacional}   
\end{figure}
\end{landscape}
\chapter{Diseño arquitectónico} \label{cap:arquitectura}

\section{Introducción}

\section{Diagrama de despliegue}

\subsection{Descripción de los nodos}

\subsection{Descripción de los componentes}
\chapter{Diseño de la interfaz} \label{cap:diseño_interfaz}

\section{Introducción}

\section{Características comunes}

\section{Interfaz del módulo Usuario...}
\subsection{Descripción general}
hjk
\subsection{Descripción detallada}
hjk
\subsection{Otros elementos de la interfaz}
hjk

\section{Interfaz del módulo Administrador}
\subsection{Descripción general}
hjk
\subsection{Descripción detallada}
hjk
\subsection{Otros elementos de la interfaz}
hjk








\part{Pruebas}
\chapter{Pruebas}\label{cap:pruebas}

\section{Introducción}

\section{Prueba de usuarios públicos}

\subsection{Pruebas CU-1.1}
 \begin{itemize}
    \item Prueba realizada
    \item Resultado de la prueba
    \item Prueba realizada
    \item Resultado de la prueba
 \end{itemize}




\part{Conclusiones y futuras mejoras}
\chapter{Conclusiones}

\section{Introducción}
En este capítulo se exponen la conclusiones obtenidas sobre el desarrollo del proyecto, y en relación con los objetivos planteados inicialmente en el capítulo 2 frente al resultado final alcanzado tras la finalización del mismo y las pruebas realizadas. 

Con carácter general, el objetivo principal, consistente en el desarrollo de un sistema informático ...

\section{Conclusiones específicas}

\section{Conclusiones personales y profesionales}
A nivel personal, el presente proyecto nos ha permitido reciclar, actualizar ...

En cuanto al ámbito profesional, se han logrado los objetivos propuestos:

\begin{itemize}

\item 

\end{itemize}



\chapter{Futuras mejoras}

\section{Introducción}
El sistema desarrollado en este proyecto es susceptible de mejoras y ampliaciones, como cualquier otro software. De hecho, se identificaron muchas oportunidades durante el desarrollo.

En las siguientes secciones se desglosan las mejoras y ampliaciones propuestas, según cada una de las categoría expuestas.

\section{Mejoras y nuevas funciones}
Basándonos en lo que se ha desarrollado hasta ahora, se identifican una serie de funciones adicionales y mejoras que se pueden realizar. Estas mejoras se describen a continuación, divididas por tipo de usuario y mejoras generales.

\subsection{Mejoras para usuarios}
\begin{itemize}
    \item G
\end{itemize}

\subsection{Mejoras para administrador}
\begin{itemize}
    \item 
\end{itemize}

\subsection{Mejoras de funcionamiento}
\begin{itemize}
    \item 
\end{itemize}

\subsection{Integración con otras aplicaciones}


\subsection{Exportabilidad}



\chapter{ANEXOS}

Este apartado está destinado a incluir todo aquello que queráis añadir de manera adicional al trabajo.

\addcontentsline{toc}{chapter}{BIBLIOGRAFÍA}
\printbibliography

%---------------------------------------------------------------------------------------------
% BORRAR
\chapter{README (BORRAR EN EL main.tex)}

\section{Ejemplos uso Overleaf comunes}

% ---------------------------------------------------------------------------------

Ejemplo cita de la bibliografía \textbf{MEDAC}\cite{medac} es un centro de Formación Profesional...

\cite{arquitecturaMariadb}
\cite{administracionMysql}
\cite{mongodbEnterprise}

% ---------------------------------------------------------------------------------

\begin{itemize}
    \item Áreas de sistemas y departamentos de informática en cualquier sector de actividad.
    \item Sector de servicios tecnológicos y comunicaciones.
    \item Área comercial con gestión de transacciones por Internet.
\end{itemize}

% ---------------------------------------------------------------------------------

\begin{figure}[htp]
    \centering
    \includegraphics[scale=0.9]{img/diagramas/Modelo-Físico-ER.png}
    \caption{Ejemplo de adjuntar diagrama ER}
    \label{fig:ModeloFísicoER}
\end{figure}

% ---------------------------------------------------------------------------------

\textit{Texto escrito en cursiva}
\textup{Texto escrito en letras rectas}
\textsl{Texto roman de estilo inclinado}
\textsc{Texto escrito en mayúsculas pequeñas}


\section{Justificación}

El presente documento describe la guía de trabajo a seguir, para la correcta elaboración del documento sobre el que se soporta el desarrollo del módulo Proyecto, del segundo curso del C.F.G.S de Desarrollo de Aplicaciones Web.

Este módulo profesional complementa la formación de otros módulos profesionales en las funciones de análisis del contexto, diseño y organización de la intervención y planificación de la evaluación de la misma.


Así mismo, teniendo en cuenta que las actividades profesionales asociadas a estas funciones se aplican en:

\begin{itemize}
    \item Áreas de sistemas y departamentos de informática en cualquier sector de actividad.
    \item Sector de servicios tecnológicos y comunicaciones.
    \item Área comercial con gestión de transacciones por Internet.
\end{itemize}
    
Y, por otra parte, las líneas de actuación en el proceso de enseñanza aprendizaje que permiten alcanzar los objetivos del módulo versarán sobre:
\begin{itemize}
    \item La ejecución de trabajos en equipo.
    \item La autoevaluación del trabajo realizado.
    \item La autonomía y la iniciativa.
    \item El uso de las TIC.
\end{itemize}

Por estos motivos se propone un Proyecto Integrado orientado al desarrollo de forma autónoma de nuevos productos y servicios innovadores, así como al autoempleo, con el propósito de que el alumno aplique de forma eficaz y realista, los conocimientos y destrezas adquiridos durante el ciclo. 
Este proyecto deberá seguir el índice de contenidos que se expone en el punto 3.

\section{Formato}
Los alumnos deberán hacer la entrega del proyecto siguiendo la siguiente normativa de manera obligatoria:

\begin{itemize}
    \item Incluir Índice y Portada (disponible en la Plataforma Virtual).  
    \item La Portada debe contener el nombre del proyecto (de forma clara, precisa y representativa del trabajo), nombre del alumno, curso académico, nombre del módulo y nombre del Centro y su logotipo.
\end{itemize}

El índice ha de ser automático, es decir, que ayude a localizar los distintos apartados del trabajo. 

\begin{itemize}
    \item Tras el índice deberá aparecer una página de cortesía.
    \item Enumerar páginas a partir del primer apartado, es decir excluir índice y página de cortesía.
    \item Justificar párrafos.
    \item Enumerar los distintos apartados.
    \item Corregir ortografía y gramática (no cometer faltas de ortografía, cuidar la expresión, etc.).
    \item Referenciar según normas APA.
    \item Utilizar la plantilla Word de Medac (disponible en la Plataforma).
    \item Tipo de letra: Times New Roman 12
    \item Interlineado 1,5.
    \item La extensión del proyecto debe ser de mínimo 18 páginas y máximo 30 páginas (Sin incluir portada, índice y anexos).
\end{itemize}

\section{Orientaciones sobre evaluación}

A parte de los criterios de evaluación dispuestos en la Real Decreto de la titulación, el tutor del proyecto evaluará el trabajo en base a los siguientes criterios de evaluación referentes a la realización del mismo, así como a la actitud del alumno ante su elaboración y trabajo en equipo:

\begin{itemize}
    \item Originalidad de la idea.
    \item Inclusión de los contenidos reflejados en el índice.
    \item Cumplimiento de las normas de formato de entrega, descritas en esta programación.
    \item Coherencia y argumentación del contenido.
    \item Que el proyecto evidencie que el alumno ha adquirido conocimientos propios de su titulación.
    \item Puntualidad y asistencia tanto a las clases teóricas de Proyecto como a las tutorías propuestas por el tutor.
    \item Nivel de responsabilidad observado.
    \item Grado de iniciativa mostrado por el alumno.
    \item Predisposición para el trabajo en equipo.
    \item Compañerismo y respeto con todos los agentes implicados en la elaboración del proyecto: compañeros y tutores.

\end{itemize}

Por último, se detallan a continuación los criterios de evaluación sobre los que se apoyará el tribunal ante la defensa del proyecto:

\begin{itemize}
    \item Originalidad en la presentación.
    \item Correcta expresión oral y corporal.
    \item Dominio de todo el contenido del proyecto.
    \item Indumentaria formal y acorde para la ocasión.
    \item Ajuste al tiempo indicado para la defensa.
    \item Trato respetuoso hacia todos los presentes en el aula durante la defensa.

\end{itemize}

Recordaros que aquellos trabajos en los que se detecte plagio, no serán evaluados, debiendo el alumnado entregar ese trabajo de nuevo en el periodo de recuperaciones correspondiente. La diferencia entre algo que está literalmente copiado y algo que ha sido citado o en lo que el alumnado se ha basado para elaborar su trabajo es clara. MEDAC persigue una evaluación justa y limpia persiguiendo la correcta competencia entre el alumnado y como tal evaluará.

%---------------------------------------------------------------------------------------------

% \renewcommand{\appendixpagename}{Anexos}
% \renewcommand{\appendixtocname}{Anexos}
% \renewcommand{\appendixname}{Anexo}

\end{document}